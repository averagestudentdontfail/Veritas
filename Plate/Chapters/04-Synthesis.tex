% Chapters/04-Synthesis.tex

\chapter{Evidence Synthesis}
\label{cp:synthesis}

\section{Overview of Evidence Landscape}

The systematic review reveals a fundamental characteristic of contemporary Australasian burn care: \gls{multidisciplinary} management has become so universally accepted that no recent studies compare it with traditional single-discipline models. This complete paradigm shift, while validating \gls{mdt} effectiveness through universal adoption, creates challenges for evidence-based evaluation. The 15 selected studies therefore provide convergent rather than comparative evidence, examining \gls{mdt} implementation quality, outcome variations, and optimization strategies rather than questioning the fundamental approach.

\section{Critical Appraisal of Evidence}

\subsection{Registry-Based Population Evidence}

The most comprehensive evidence emerges from \gls{branz} registry analyses encompassing over 31,000 burn admissions. \textcite{Cleland2016} established baseline understanding by documenting that all participating units employ \glspl{mdt}, though with varying implementation completeness. Units with comprehensive \gls{mdt} protocols (daily rounds, unified documentation, formal communication structures) demonstrated superior risk-adjusted outcomes compared to units with less developed team integration.

\textcite{Gong2021} advanced this foundation by developing 23 evidence-based quality indicators for \gls{multidisciplinary} burn care through systematic review and Delphi consensus. These indicators span structure (team composition, meeting frequency), process (assessment timing, communication protocols), and outcome measures (functional recovery, patient satisfaction). Analysis of 31,498 consecutive admissions revealed that units achieving higher compliance with these indicators demonstrated 22\% shorter length of stay (median 8 versus 11 days, p<0.001) and improved functional independence scores at discharge.

\textcite{Tracy2023adherence} examined allied health integration specifically, finding that 97\% of burn patients now receive physiotherapy and occupational therapy assessment within 48 hours of admission. This represents dramatic improvement from historical data suggesting only 60\% received allied health input during entire admissions in the 1990s. Early assessment correlated with reduced \gls{contracture} rates (8\% versus 19\% historically, p<0.001) and improved range of motion preservation.

\subsection{Functional Recovery and Long-term Outcomes}

\textcite{Tracy2025} conducted the first comprehensive feasibility study for centralizing long-term outcome collection across \gls{branz} units. Following 342 burn survivors for two years, they demonstrated that coordinated \gls{mdt} care extending beyond acute admission maintains benefits over time. Patients receiving structured \gls{mdt} follow-up showed superior Burn Specific Health Scale-Brief scores at 24 months compared to those discharged to fragmented community care (mean difference 12.3 points, 95\% \gls{ci} 8.7-15.9).

\textcite{Gabbe2015} established the methodological framework for evaluating long-term \gls{mdt} outcomes, piloting assessment protocols across Victorian burn services. Their work revealed that functional recovery trajectories differ markedly based on \gls{mdt} care coordination quality. Units with formal transition protocols linking acute and community services achieved \gls{rtw} rates of 78\% at 12 months versus 54\% in units without structured handover processes.

\textcite{Tracy2020} examined predictors of chronic pain and itch, demonstrating that early \gls{mdt} pain management protocols significantly influence long-term outcomes. Patients receiving coordinated pain care from admission (including pharmacological, psychological, and physical therapy interventions) reported 40\% lower pain severity scores at 12 months compared to historical cohorts receiving sequential pain management approaches.

\subsection{Cultural Safety Considerations}

\textcite{Hunter2024} provided crucial evidence about \gls{mdt} care for Aboriginal and Torres Strait Islander children through the Coolamon Study. This prospective cohort of 156 Indigenous children with burns revealed that culturally-adapted \gls{mdt} models incorporating Aboriginal Health Workers achieved dramatically improved outcomes. When Indigenous Health Workers participated as core \gls{mdt} members rather than cultural consultants, length of stay decreased by 2.8 days (95\% \gls{ci} 1.2-4.4) and follow-up attendance improved from 52\% to 84\%.

\textcite{Coombes2020} conducted qualitative research with 18 Indigenous families, revealing how standard \gls{mdt} approaches may inadvertently create barriers through overwhelming information delivery, conflicting communication from multiple team members, and failure to accommodate extended family involvement in decision-making. Families reported that First Nations Health Workers serving as cultural brokers and care coordinators transformed their experience from ``frightening and confusing'' to ``supported and understood.''

These studies highlight that effective \gls{mdt} care requires adaptation to cultural contexts rather than universal application of standardized protocols. The evidence suggests that Indigenous patients benefit particularly from \gls{mdt} approaches when \gls{culturalsafety} principles integrate throughout team functioning rather than being addressed as separate considerations.

\subsection{Technology-Based MDT Delivery}

\textcite{Plaza2022} conducted the first Australasian \gls{rct} examining telerehabilitation within \gls{mdt} frameworks. Forty-five patients with burns affecting less than 25\% \gls{tbsa} were randomized to receive either traditional in-person \gls{mdt} rehabilitation or \gls{telehealth}-delivered therapy with remote \gls{mdt} coordination. The study demonstrated non-inferiority for functional outcomes (Lower Limb Functional Index difference 2.1 points, 95\% \gls{ci} -3.4 to 7.6) while reducing travel burden and improving rural access.

\textcite{Matthew2023} examined \gls{telehealth} implementation across remote Australian communities, demonstrating that technology can extend rather than replace \gls{mdt} coordination. Their mixed-methods evaluation of 89 patients revealed that successful remote \gls{mdt} care requires dedicated coordination roles, reliable technology infrastructure, and hybrid models combining periodic in-person assessment with regular virtual contact.

These technology studies prove particularly relevant given Australia's geographic challenges, with many patients living hundreds of kilometers from specialist \glspl{burnunit}. The evidence suggests that \gls{telehealth} can maintain \gls{mdt} care quality while improving accessibility, though it requires systematic implementation rather than ad hoc adoption.

\subsection{Economic Considerations}

\textcite{Cleland2022} analyzed the economic burden of severe burns in Victoria, providing crucial context for \gls{mdt} resource allocation. Examining 331 patients with severe burns, they calculated mean acute care costs of \gls{aud} \$289,000 per patient, with considerable variation based on burn severity and complications. While \gls{mdt} care requires greater upfront resource investment (approximately 18\% higher daily costs), total episode costs prove lower through reduced complications and shorter admissions.

\textcite{McPhail2022} conducted formal economic evaluation of \gls{mdt} scar management protocols, demonstrating cost-effectiveness despite intensive resource requirements. Their analysis of 198 patients revealed that coordinated scar care involving therapists, nurses, and medical staff achieved incremental cost-effectiveness ratio of \gls{aud} \$21,000 per \gls{qaly} gained, well below accepted thresholds for healthcare interventions.

\section{Synthesis Across Studies}

Despite the absence of direct comparative trials, multiple evidence streams converge to support \gls{mdt} effectiveness. Registry data demonstrates universal adoption with associated quality improvements. Implementation studies confirm feasibility across diverse settings. Economic analyses validate resource efficiency despite higher daily costs. Special population research reveals enhanced benefits when \gls{mdt} models adapt to cultural needs.

The consistency of findings across different study designs, populations, and outcome measures provides robust triangulation supporting \gls{mdt} superiority over historical single-discipline models. While we cannot definitively quantify the magnitude of \gls{mdt} benefits without randomized comparison, the complete absence of units maintaining traditional approaches paradoxically provides the strongest possible endorsement: no Australasian burn service considers single-discipline care acceptable practice.

\section{Quality Assessment of Evidence}

Using Oxford Centre for Evidence-Based Medicine criteria, the overall evidence quality supporting \gls{multidisciplinary} burn care rates as Level 2a (systematic reviews of cohort studies with consistent results). While the absence of \glspl{rct} comparing \gls{mdt} to traditional care prevents Level 1 evidence designation, the convergent findings from multiple high-quality observational studies, registry analyses, and the single \gls{rct} examining \gls{telehealth} delivery provide robust support for \gls{mdt} effectiveness.

Methodological strengths include large sample sizes from \gls{branz} registry data, standardized outcome measurement across centers, prospective data collection in several studies, and triangulation across quantitative and qualitative methodologies. Limitations include the inability to establish causation definitively without randomized comparison, potential publication bias favoring positive \gls{mdt} findings, and heterogeneity in \gls{mdt} implementation making precise effect size estimation challenging.

The economic evaluations by \textcite{Cleland2022} and \textcite{McPhail2022} provide Level 2b evidence (individual cohort studies) supporting cost-effectiveness, though comprehensive economic modeling comparing \gls{mdt} to hypothetical single-discipline care remains absent. The qualitative research by \textcite{Coombes2020} adds important context about implementation challenges and cultural considerations that quantitative studies might miss, strengthening the overall evidence synthesis through methodological diversity.