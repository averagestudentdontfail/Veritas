% Chapters/02-Method.tex

\chapter{Methods}
\label{cp:methods}

\section{Search Strategy}

A comprehensive review following \gls{prisma} guidelines was conducted between December 2023 and February 2025. The search strategy aimed to identify all relevant Australasian studies comparing \gls{multidisciplinary} versus traditional burn care models or evaluating \gls{mdt} implementation with historical controls.

\subsection{Database Search}

Primary databases searched included PubMed/MEDLINE (2014-2025), CINAHL Complete (2014-2025), Cochrane Library including Cochrane Database of Systematic Reviews, EMBASE (2014-2025), and the \gls{branz} publications database. Secondary sources included Australian Indigenous HealthInfoNet, Google Scholar (first 200 results), reference lists of included studies, \gls{anzba} conference proceedings, and institutional repositories of major burn centers.

Search terms were combined using Boolean operators:
\begin{itemize}
    \item Population: (burn* OR ``thermal injury'' OR scald*) AND (Australia* OR ``New Zealand'' OR ANZBA OR BRANZ)
    \item Intervention: (multidisciplinary OR interdisciplinary OR ``team-based'' OR ``coordinated care'' OR ``collaborative management'')
    \item Comparison: (traditional OR ``single discipline'' OR sequential OR ``usual care'')
    \item Outcomes: (mortality OR survival OR ``length of stay'' OR function* OR ``quality of life'' OR recovery)
\end{itemize}

\clearpage

\section{Selection Process}

One independent reviewer screened titles and abstracts against predetermined criteria, with full-text review for potentially eligible studies.

\subsection{Inclusion Criteria}
\begin{enumerate}
    \item Studies from Australasian \glspl{burnunit} published January 2014 to February 2025
    \item Adult and/or pediatric burn populations with acute injuries
    \item Evaluation of \gls{multidisciplinary} models or team-based interventions
    \item Clinical outcomes including mortality, length of stay, complications, functional measures, or \gls{qol}
    \item Quantitative or qualitative research designs
    \item English language publication
\end{enumerate}

\subsection{Exclusion Criteria}
\begin{enumerate}
    \item Non-Australasian settings
    \item Case reports with fewer than 10 patients
    \item Opinion pieces without empirical data
    \item Conference abstracts without full publication
    \item Studies exclusively examining chronic burn reconstruction
    \item Animal or laboratory studies
\end{enumerate}

\section{Data Extraction and Quality Assessment}

The standardized data extraction captured study characteristics, population details, intervention descriptions, outcome measures, and results. The quality assessment employed Oxford Centre for Evidence-Based Medicine criteria, evaluating study design, risk of bias, sample size, outcome measurement, and statistical analysis appropriateness.

\section{Flow Diagram}

The search and selection process followed \gls{prisma} guidelines to ensure transparent and reproducible methodology. \autoref{fig:prisma} illustrates the flow of studies through each phase of the review.

\begin{landscapefigure}[p]  % [p] forces it to its own page
    \caption{Search and Study Selection Process}
    \label{fig:prisma}
    \vspace{0.5cm}
    \resizebox{0.7\linewidth}{!}{%  % Scale to 70% of line width
        \begin{tikzpicture}[
            box/.style={rectangle, draw=black, thick, text width=12cm, minimum height=1.2cm, align=center, fill=white},
            smallbox/.style={rectangle, draw=black, thick, text width=6cm, minimum height=1.2cm, align=center, fill=white},
            arrow/.style={thick, ->, >=stealth}]
            
            % Identification phase
            \node[box] (identification) {\textbf{Identification}\\Records identified through database searching\\(n = 287)\\
            \small PubMed (n=89), CINAHL (n=67), Cochrane (n=12), EMBASE (n=94), BRANZ (n=25)};
            
            % After duplicates removed
            \node[box, below=1.5cm of identification] (afterdup) {\textbf{After Duplicates Removed}\\Records after duplicates removed\\(n = 184)\\
            \small 103 duplicates identified and removed};
            
            % Screening phase
            \node[box, below=1.5cm of afterdup] (screened) {\textbf{Screening}\\Records screened by title and abstract\\(n = 184)};
            
            % Excluded at screening
            \node[smallbox, right=4cm of screened] (excluded1) {Records excluded\\(n = 108)\\
            \small Non-Australasian (n=67)\\Wrong population (n=23)\\No relevant outcomes (n=18)};
            
            % Eligibility assessment
            \node[box, below=1.5cm of screened] (fulltext) {\textbf{Eligibility}\\Full-text articles assessed for eligibility\\(n = 76)};
            
            % Excluded at full-text
            \node[smallbox, right=4cm of fulltext] (excluded2) {Full-text articles excluded\\(n = 33)\\
            \small Insufficient outcome data (n=15)\\Conference abstracts only (n=11)\\Wrong focus (n=7)};
            
            % Included for analysis
            \node[box, below=1.5cm of fulltext] (included) {\textbf{Included}\\Studies meeting all inclusion criteria\\(n = 43)\\
            \small Registry analyses (n=12), Cohort studies (n=18),\\Qualitative studies (n=8), RCTs (n=2), Economic evaluations (n=3)};
            
            % Final synthesis
            \node[box, below=1.5cm of included, fill=gray!20] (synthesis) {\textbf{Critical Appraisal}\\Studies selected for detailed critical appraisal\\(n = 15)\\
            \small Highest quality studies based on methodology,\\sample size, and contribution to evidence base};
            
            % Draw arrows
            \draw[arrow] (identification) -- (afterdup);
            \draw[arrow] (afterdup) -- (screened);
            \draw[arrow] (screened) -- (fulltext);
            \draw[arrow] (fulltext) -- (included);
            \draw[arrow] (included) -- (synthesis);
            \draw[arrow] (screened) -- (excluded1);
            \draw[arrow] (fulltext) -- (excluded2);
        \end{tikzpicture}
    }
\end{landscapefigure}

The search identified 287 potentially relevant articles across five databases. After removing 103 duplicates, 184 titles and abstracts underwent screening. This process excluded 108 articles not meeting inclusion criteria (67 non-Australasian, 23 wrong population, 18 no relevant outcomes). Full-text assessment of 76 articles led to exclusion of 33 studies (15 insufficient outcome data, 11 conference abstracts only, 7 wrong focus). The final analysis included 43 studies meeting all criteria, with 15 highest-quality studies selected for detailed critical appraisal based on methodological rigor, sample size, and contribution to the evidence base.