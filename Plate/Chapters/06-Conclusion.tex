% Chapters/06-Conclusion.tex

\chapter{Conclusion}
\label{cp:conclusion}

This critical appraisal reveals a striking reality: coordinated \gls{multidisciplinary} management has achieved such complete acceptance in Australasian burn care that traditional single-discipline comparators no longer exist. While this prevents definitive quantification of \gls{mdt} benefits through randomized trials, the universal adoption itself provides powerful validation. No burn service anywhere in Australia or New Zealand considers returning to surgeon-led sequential consultation models acceptable practice.

The construction worker in our opening scenario would today receive immediate coordinated attention from multiple specialists working as an integrated team. His burn wounds would heal with surgical expertise while physiotherapists preserve his range of motion, dietitians prevent malnutrition, psychologists address traumatic stress, and social workers prepare his family for the recovery journey. This coordinated approach, supported by convergent evidence from registry analyses, implementation studies, and economic evaluations, optimizes his chances for returning to meaningful work and life participation.

The evidence illuminates clear paths forward. Units should implement structured communication protocols, ensure early allied health integration, and adapt coordination approaches for \gls{culturalsafety}. Technology can extend \gls{mdt} benefits to remote communities without compromising quality. Economic investment in coordination resources returns value through improved outcomes and reduced complications.

Most significantly, the Australasian experience demonstrates that complex healthcare challenges demand complex solutions. No single discipline possesses all expertise necessary for optimal burn outcomes. When healthcare professionals truly collaborate through structured protocols, shared decision-making, and unified goals, patient outcomes improve across every measured domain. The question is no longer whether to implement \gls{mdt} care but how to optimize its delivery for every burn patient regardless of location, background, or circumstance.

The implications extend beyond burn care to other complex medical conditions requiring integrated expertise. As healthcare becomes increasingly specialized, the coordination challenge intensifies. The success of \gls{multidisciplinary} burn care provides a blueprint for team-based approaches in trauma, critical care, rehabilitation, and chronic disease management. \textcite{Gong2021}'s quality indicators could be adapted for other conditions, while the implementation strategies documented by \textcite{Hunter2024} for Indigenous populations offer models for addressing health equity across all medical specialties.

For the 6,000-7,000 Australians and New Zealanders who suffer burn injuries annually, this evidence offers hope. Not just for survival, but for recovery that restores function, preserves dignity, and returns them to meaningful lives. The 42-year-old construction worker represents thousands of individuals whose outcomes depend on healthcare systems recognizing that excellence emerges not from individual brilliance but from coordinated expertise working toward shared goals.

The \gls{branz} registry continues documenting outcomes, providing ongoing validation of \gls{mdt} effectiveness while identifying opportunities for improvement. As \textcite{Tracy2025} demonstrated, the benefits of coordinated care extend far beyond hospital discharge, influencing recovery trajectories for years. This long-term perspective reinforces the importance of viewing burn care not as an acute episode but as a continuum requiring sustained coordination across settings and time.

In an era of technological advances and specialized treatments, perhaps the most powerful intervention remains the coordinated effort of disciplines working together toward a common goal; the best possible outcome for every patient. The evidence from Australasian \glspl{burnunit} demonstrates that when we move beyond professional silos to embrace true collaboration, we achieve outcomes that no single discipline could accomplish alone.

The evidence is clear. The implementation challenge remains. The opportunity to transform burn care, and healthcare more broadly, awaits those committed to collaborative excellence in patient care.