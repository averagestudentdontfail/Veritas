\chapter{Introduction}
\label{cp:introduction}

\section{Clinical Scenario}

A 42-year-old construction worker presents to the emergency department following a workplace accident involving hot bitumen, sustaining burns to 35\% total body surface area affecting his chest, abdomen, and both arms. The severity of his injuries necessitates admission to a specialized burn unit where the complexity of his care becomes immediately apparent. Beyond the critical need for fluid resuscitation and wound management, he requires pain control, early mobilization to prevent contractures, nutritional support, psychological assistance for acute stress, and coordination with his family who are struggling to understand the lengthy recovery process ahead. The burn unit team faces a fundamental question: will coordinated multidisciplinary care involving surgeons, nurses, physiotherapists, occupational therapists, dietitians, psychologists, and social workers produce better outcomes than traditional sequential consultation models where each discipline operates independently?

This scenario, replicated thousands of times annually across Australasian burn units, illustrates why burn injury represents one of medicine's most complex challenges. The question of optimal care coordination becomes not merely academic but urgently practical, affecting both immediate survival and long-term quality of life for burn survivors.

\section{Background}

\subsection{The Australasian Burn Care Context}

Burn injury affects approximately 6,000-7,000 Australians and New Zealanders requiring hospitalization annually, with severe burns (>20\% total body surface area) comprising 15\% of admissions. The Burns Registry of Australia and New Zealand (BRANZ), established in 2009, now collects standardized data from 17 specialist burn units, creating one of the world's most comprehensive burn care quality monitoring systems. This infrastructure enables rigorous evaluation of different care models, revealing significant variations in practice and outcomes between centers despite standardized protocols.

The evolution from traditional hierarchical medical care to integrated multidisciplinary approaches reflects broader recognition that burn injury affects multiple body systems simultaneously. A severe burn triggers not only local tissue damage but also systemic inflammatory responses, metabolic derangements, psychological trauma, and social disruption. This complexity suggests that coordinated team-based care might achieve better outcomes than sequential single-discipline interventions, yet empirical evidence specific to the Australasian context has only recently emerged.

\subsection{Defining Multidisciplinary Burn Care}

Multidisciplinary management in burn care extends beyond simple co-location of different specialists. True multidisciplinary care involves structured communication protocols, shared decision-making, coordinated treatment planning, and integrated outcome assessment. The Australian and New Zealand Burn Association (ANZBA) defines optimal multidisciplinary care as requiring regular team meetings, unified documentation systems, coordinated goal-setting with patients and families, and systematic quality improvement processes.

This contrasts sharply with traditional models where surgeons direct medical management while other disciplines provide supplementary services upon request. The fundamental question becomes whether the additional resources required for coordinated multidisciplinary care produce sufficient improvements in patient outcomes to justify the investment.

\section{Focused Clinical Question}

In adults with acute burn injury requiring specialist burn unit admission, does coordinated multidisciplinary team management, compared with traditional single-discipline-led care with sequential consultations, improve clinical outcomes including survival, length of stay, functional recovery, and quality of life?