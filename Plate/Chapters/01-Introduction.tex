\chapter{Introduction}
\label{cp:introduction}

\section{Clinical Scenario}

A 42-year-old construction worker presents to the emergency department following a workplace accident involving hot bitumen, sustaining burns to 35\% total body surface area (TBSA) affecting his chest, abdomen, and both arms. TBSA represents the percentage of body surface affected by burns, typically calculated using either the ``Rule of Nines'' (dividing body surface into sections representing 9\% or multiples thereof) or more precise Lund and Browder charts that account for age-related body proportions. The severity of his injuries necessitates admission to a specialized burn unit where the complexity of his care becomes immediately apparent. Beyond critical fluid resuscitation using the Parkland Formula (4mL/kg/\% TBSA of crystalloid over 24 hours), he requires multimodal pain management, early mobilization to prevent contractures, nutritional support exceeding 30 kcal/kg/day, psychological assistance for acute stress symptoms, and coordination with his family struggling to understand the lengthy recovery ahead.

The burn unit team faces a fundamental question: will coordinated multidisciplinary care involving burn surgeons, intensivists, nurses, physiotherapists, occupational therapists, dietitians, psychologists, and social workers produce better outcomes than traditional sequential consultation models where each discipline operates independently? This scenario, replicated thousands of times annually across Australasian \glspl{burnunit}, illustrates why burn injury represents one of medicine's most complex management challenges.

\section{Background}

\subsection{The Australasian Burn Care Context}

Burn injury affects approximately 6,000-7,000 Australians and New Zealanders requiring hospitalization annually, with severe burns (defined as greater than 20\% TBSA or requiring intensive care admission) comprising 15\% of these admissions. The \gls{branz}, established in 2009, systematically collects standardized clinical data from 17 specialist \glspl{burnunit} across both countries, creating one of the world's most comprehensive burn care quality monitoring systems. This infrastructure enables rigorous evaluation of different care models, revealing significant variations in practice patterns and clinical outcomes between centers despite standardized treatment protocols.

The \gls{anzba} coordinates evidence-based practice standards across the region, establishing minimum criteria for designated burn centers including 24-hour specialized nursing, immediate surgical availability, and access to allied health services. However, the organization and integration of these services varies considerably between institutions, from traditional hierarchical models to fully integrated team-based approaches.

\subsection{Defining Multidisciplinary Burn Care}

\gls{multidisciplinary} in burn care extends beyond simple co-location of different healthcare specialists. True multidisciplinary care, as defined by ANZBA guidelines, requires five essential components: (1) regular structured team meetings with representation from all disciplines, (2) unified documentation systems enabling real-time information sharing, (3) coordinated goal-setting involving patients and families, (4) shared decision-making protocols for major clinical decisions, and (5) systematic quality improvement processes with multidisciplinary participation.

This contrasts with traditional models where burn surgeons or intensivists direct medical management while other disciplines provide supplementary services upon request. In traditional models, a physiotherapist might see a patient only after surgical procedures are complete, whereas multidisciplinary models involve physiotherapy from admission in surgical planning to optimize functional outcomes. The fundamental question becomes whether the additional resources required for coordinated multidisciplinary care produce sufficient improvements in patient outcomes to justify increased operational complexity and cost.

\section{Focused Clinical Question}

In adult and paediatric patients with acute burn injury requiring specialist burn unit admission, does coordinated multidisciplinary team management, compared with traditional single-discipline-led care with sequential consultations, improve clinical outcomes including survival, length of stay, functional recovery, and quality of life?