% Chapters/05-Discussion.tex

\chapter{Discussion}
\label{cp:discussion}

\section{Clinical Bottom Line}

\textbf{Strong convergent evidence supports coordinated \gls{multidisciplinary} management as the optimal care standard for acute burn injury in Australasian settings.} While direct comparative studies with traditional single-discipline care are absent due to universal \gls{mdt} adoption, multiple evidence streams consistently demonstrate:

\begin{enumerate}
    \item \textbf{Universal implementation validates effectiveness}: All 17 specialist \glspl{burnunit} across Australia and New Zealand have adopted \gls{mdt} models, with no services maintaining traditional single-discipline approaches. This complete paradigm shift represents powerful implicit evidence for \gls{mdt} superiority.
    
    \item \textbf{Process indicators demonstrate coordination quality}: 97\% of patients receive allied health assessment within 48 hours \parencite{Tracy2023adherence}, compared to historical rates of 60\% receiving any allied health input. Units with comprehensive \gls{mdt} protocols achieve 22\% shorter length of stay \parencite{Gong2021}.
    
    \item \textbf{Functional outcomes show sustained improvement}: Structured \gls{mdt} follow-up maintains benefits at 24 months post-injury \parencite{Tracy2025}. \Gls{rtw} rates reach 78\% with formal \gls{mdt} transition protocols versus 54\% without \parencite{Gabbe2015}.
    
    \item \textbf{Cultural safety and health equity improves outcomes}: Indigenous children experience 2.8-day shorter admissions when Aboriginal Health Workers integrate as core \gls{mdt} members \parencite{Hunter2024}. Culturally-adapted \gls{mdt} models improve follow-up attendance from 52\% to 84\%.
    
    \item \textbf{Technology successfully extends \gls{mdt} reach}: Telerehabilitation proves non-inferior to in-person \gls{mdt} therapy while improving rural access \parencite{Plaza2022}. Remote communities can access coordinated care through structured \gls{telehealth} programs \parencite{Matthew2023}.
    
    \item \textbf{Economic evaluation supports resource allocation}: Despite 18\% higher daily costs, \gls{mdt} care reduces total episode expenses through fewer complications and shorter admissions \parencite{Cleland2022}. Scar management protocols achieve cost-effectiveness at \gls{aud} \$21,000 per \gls{qaly} gained \parencite{McPhail2022}.
\end{enumerate}

\section{Implications for Practice}

\subsection{Immediate Implementation Priorities}

\Glspl{burnunit} seeking to optimize \gls{multidisciplinary} care should focus on three foundational elements supported by the evidence. First, establish daily structured team rounds using standardized communication frameworks. \textcite{Gong2021} demonstrated that even basic coordination protocols significantly improve outcomes compared to informal communication patterns. Second, ensure allied health assessment within 48 hours of admission, as delays correlate with poorer functional outcomes and increased complications. Third, develop unified documentation systems that enable real-time information sharing between disciplines, reducing communication errors and treatment delays.

The evidence particularly emphasizes early integration rather than sequential consultation. Physiotherapists and occupational therapists should participate in surgical planning from admission rather than commencing therapy after wound closure. Dietitians should establish nutrition protocols immediately rather than responding to developing malnutrition. Psychologists should engage during acute care rather than addressing established \gls{ptsd}.

\subsection{Institutional Requirements}

Implementing effective \gls{mdt} care demands institutional commitment beyond good intentions. Protected time for team meetings proves essential, with successful units allocating minimum 30 minutes daily for structured rounds. Physical spaces that support collaboration, such as conference rooms adjacent to \glspl{burnunit}, facilitate regular team interaction. Electronic health records must enable simultaneous documentation and review by multiple disciplines rather than siloed systems requiring duplicate data entry.

Resource calculations from \textcite{Cleland2022} indicate that a 20-bed \gls{burnunit} requires approximately 2.0 additional full-time equivalent positions across allied health disciplines to support comprehensive \gls{mdt} care compared to traditional models. However, economic analysis confirms return on investment through reduced complications, shorter admissions, and improved long-term outcomes.

\subsection{Cultural Safety and Equity}

The evidence from \textcite{Hunter2024} and \textcite{Coombes2020} demands systematic attention to \gls{culturalsafety} within \gls{mdt} frameworks. Indigenous Health Workers should participate as core team members rather than occasional cultural consultants. Communication protocols must accommodate extended family involvement in decision-making for Indigenous patients. Information delivery requires adaptation to avoid overwhelming families with multiple team members providing fragmented messages.

These principles extend beyond Indigenous populations to all culturally and linguistically diverse communities. Effective \gls{mdt} care requires cultural humility, recognizing that coordination approaches successful in mainstream populations may create barriers for minority groups. Regular cultural competency training for all team members, employment of bicultural workers, and systematic evaluation of equity outcomes prove essential.

\subsection{Technology Integration}

Rural and remote burn care can achieve quality comparable to metropolitan centers through systematic \gls{telehealth} integration. \textcite{Plaza2022} and \textcite{Matthew2023} demonstrate that success requires dedicated coordination roles, reliable technology infrastructure, and hybrid models combining virtual and in-person contact. Investment in high-quality videoconferencing equipment, staff training in virtual consultation skills, and protocols for remote assessment prove essential.

\Gls{telehealth} should supplement rather than replace face-to-face \gls{mdt} coordination for complex cases. The evidence suggests optimal models involve initial in-person assessment when possible, regular virtual team rounds including remote providers, and structured handover processes when patients transfer between virtual and physical care settings.

\section{Future Directions}

\subsection{Research Priorities}

The absence of comparative studies creates clear research imperatives. While randomizing patients to receive suboptimal care raises ethical concerns, natural experiments may arise when resource constraints force temporary \gls{mdt} service reductions. Prospective data collection during such periods could provide comparative evidence currently lacking.

Implementation science methodologies could reveal how to optimize existing \gls{mdt} models rather than whether to adopt them. Questions requiring investigation include optimal team composition for different burn severities, meeting frequency and structure for maximum effectiveness, and coordination approaches that balance comprehensiveness with efficiency.

Long-term outcome research beyond two years remains limited. Understanding how \gls{mdt} care during acute admission influences outcomes at five and ten years could justify increased resource investment. Particular attention to \gls{qol}, community participation, and psychological adjustment would complement existing functional and economic data.

\subsection{System-Level Changes}

The evidence supports policy mandating minimum \gls{mdt} standards for designated burn centers. \Gls{branz} quality indicators provide framework for measurement, but formal accreditation requirements could ensure consistent implementation. Funding models must recognize coordination costs while capturing downstream savings through prevented complications and readmissions.

Medical and nursing education should incorporate \gls{mdt} competencies as core curriculum rather than optional content. Training programs should emphasize communication skills, team dynamics, and systems thinking alongside traditional clinical knowledge. Simulation-based education could prepare health professionals for complex team coordination before entering clinical practice.

\section{Study Limitations}

This critically appraised topic has several limitations. The search was restricted to Australasian studies, potentially missing relevant international evidence. The heterogeneity of \gls{multidisciplinary} definitions across studies limits direct comparison. Publication bias may favor positive findings about team-based care. Finally, the focus on specialist burn centers may limit applicability to smaller or rural facilities with different resource constraints.

Despite these limitations, the evidence synthesis provides robust support for \gls{mdt} implementation as the standard of care for acute burn injury. The consistency of findings across diverse methodologies, populations, and outcome measures strengthens confidence in the conclusions despite the absence of direct comparative trials.