\thispagestyle{plain}

\pdfbookmark[1]{Abstract}{abstract}
\chapter*{Abstract}

\textbf{Background:} Acute burn injury represents one of the most complex medical emergencies, requiring coordinated intervention across multiple healthcare disciplines. While multidisciplinary team approaches have become standard in many Australasian \glspl{burnunit}, empirical evidence comparing these models to traditional single-discipline care remains limited.

\textbf{Objective:} To critically appraise current Australasian evidence examining whether coordinated multidisciplinary team management improves clinical outcomes compared to traditional single-discipline-led care in adults with acute burn injury.

\textbf{Methods:} Systematic literature search of PubMed, CINAHL, Cochrane Library, and EMBASE databases (2014-2025) identified studies comparing multidisciplinary versus traditional care models in Australasian \glspl{burnunit}. Evidence was synthesized using narrative analysis with quality assessment based on Oxford Centre for Evidence-Based Medicine criteria.

\textbf{Results:} Fourteen studies met inclusion criteria, including registry analyses, cohort studies, qualitative research, and one randomized controlled trial. Registry data from 9,441 patients demonstrated 45\% lower mortality in units with established multidisciplinary teams. Coordinated care reduced length of stay by 23-30\%, improved functional independence scores by 40\%, and improved quality of life measures. Implementation studies confirmed feasibility across diverse settings including remote communities via telehealth. Aboriginal and Torres Strait Islander populations showed particular benefit from culturally-integrated multidisciplinary approaches.

\textbf{Conclusions:} Strong evidence supports coordinated multidisciplinary team management as superior to traditional care for acute burn injury in Australasian settings. Benefits span survival, functional recovery, psychological outcomes, and cost-effectiveness. Implementation requires institutional commitment to structured communication protocols, shared decision-making, and integrated care planning.

\keywordsen{burn injury, multidisciplinary care, team composition, clinical outcome, Australia, New Zealand}

\MediaOptionLogicBlank