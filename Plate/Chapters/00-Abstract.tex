\thispagestyle{plain}

\pdfbookmark[1]{Abstract}{abstract}
\chapter*{Abstract}

A 42-year-old construction worker sustaining burns to 35\% total body surface area exemplifies the complex challenges facing Australasian burn units, where coordinated multidisciplinary care has emerged as the gold standard. This critically appraised topic synthesizes evidence from 12 Australasian studies (2014-2024) examining whether coordinated multidisciplinary team management produces superior outcomes compared to traditional single-discipline-led care for acute burn injury requiring specialist unit admission.

A comprehensive literature search across PubMed, CINAHL, Cochrane Library, and EMBASE identified studies comparing multidisciplinary versus single-discipline approaches in Australasian burn units. Selected studies included registry analyses, cohort studies, randomized controlled trials, and implementation research encompassing diverse disciplinary perspectives from medicine, nursing, physiotherapy, psychology, and social work.

Strong evidence demonstrates that coordinated multidisciplinary team management significantly improves clinical outcomes. Mortality reduces by 45-55\% in units with established multidisciplinary protocols (Level 2a evidence), length of stay decreases by 20-30\% through complication prevention (Level 2b evidence), and functional outcomes improve by 35-40\% when rehabilitation disciplines integrate from admission (Level 2b evidence). Psychological outcomes show significant improvement with integrated mental health support (Level 1b evidence), while cost-effectiveness is demonstrated despite higher initial resource requirements (Level 2b evidence).

The synthesis definitively supports coordinated multidisciplinary team management as superior to traditional hierarchical care models for acute burn injury in Australasian settings. Implementation requires institutional commitment to protected team meeting time, shared documentation systems, and cultural transformation from medical hierarchy to collaborative practice. Future research should focus on optimal team composition strategies and long-term outcome evaluation beyond one year.

\keywordsen{Burn injury, multidisciplinary care, team-based management, clinical outcomes, Australasian healthcare, evidence synthesis.}

\MediaOptionLogicBlank