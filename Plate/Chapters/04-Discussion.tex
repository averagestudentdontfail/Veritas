\chapter{Discussion}
\label{cp:discussion}

\section{Clinical Bottom Line}

\textbf{Robust evidence from Australasian burn centers definitively supports coordinated multidisciplinary team management over traditional single-discipline-led care for acute burn injury.} Based on synthesized evidence:

\begin{enumerate}
    \item \textbf{Mortality reduces by 45-55\%} in units with established multidisciplinary protocols, with benefits most pronounced in severe burns (Level 2a evidence from registry analyses)
    
    \item \textbf{Length of stay decreases by 23-30\%} through complication prevention and optimized treatment sequencing rather than premature discharge (Level 2b evidence from multiple cohort studies)
    
    \item \textbf{Functional independence improves by 35-40\%} when rehabilitation disciplines integrate from admission, with benefits persisting at two-year follow-up (Level 2b evidence from prospective cohorts)
    
    \item \textbf{Psychological morbidity reduces by 50-60\%} with integrated mental health support, preventing chronic PTSD development (Level 1b evidence from RCT)
    
    \item \textbf{Return to work rates reach 82\%} at two years with comprehensive team support versus 61\% with fragmented care (Level 2b evidence from longitudinal cohort)
    
    \item \textbf{Cost-effectiveness is clearly demonstrated} with cost per QALY gained of AUD \$28,000, despite higher daily operational costs (Level 2b economic analysis)
    
    \item \textbf{Indigenous populations show particular benefit} when teams incorporate cultural safety principles and Indigenous health workers (Level 2b evidence from prospective cohort)
\end{enumerate}

\section{Implications for Practice}

\subsection{Immediate Implementation Priorities}

\Glspl{burnunit} currently operating traditional hierarchical structures should prioritize three foundational changes. First, establish structured daily multidisciplinary rounds using standardized communication frameworks like ISBAR. Evidence suggests even twice-weekly structured meetings significantly improve outcomes compared to ad hoc communication. Second, designate a clinical coordinator role, typically filled by senior nursing staff, ensuring all disciplines contribute to care planning. Third, implement unified documentation systems enabling real-time information sharing between disciplines.

\subsection{Resource Requirements and Institutional Support}

Implementing effective multidisciplinary care demands institutional commitment beyond good intentions. Protected time for team meetings (minimum 30 minutes daily), shared documentation platforms, and physical spaces supporting collaboration prove essential. The economic evidence demonstrates return on investment through improved outcomes, but initial resource allocation remains challenging. Phillips et al. (2021) calculated that a 20-bed burn unit requires 2.0 additional full-time equivalent positions across disciplines to support comprehensive multidisciplinary care.

\subsection{Training and Culture Change}

Transitioning from hierarchical medical culture to collaborative practice requires systematic training in team communication, shared decision-making, and constructive conflict resolution. ANZBA's competency framework provides structure, but local implementation must address specific institutional cultures. Reeder et al. (2023) emphasized that ``flattening hierarchies'' proves particularly challenging in surgical specialties with strong traditional authority structures.

\subsection{Special Considerations for Rural and Remote Settings}

McWilliams et al. (2021) demonstrated that geographic isolation need not preclude multidisciplinary care. Virtual team rounds, visiting specialist programs, and partnerships with metropolitan centers can extend coordinated care to remote communities. Investment in reliable telecommunications infrastructure and training for local healthcare providers proves essential.

\section{Future Directions}

\subsection{Research Priorities}

Critical knowledge gaps require addressing through targeted research. Optimal team composition for different burn severities remains undefined. Implementation science methodologies could reveal how to transform traditional units into high-functioning teams. Long-term outcomes beyond two years require systematic investigation. Indigenous and culturally diverse populations need specific attention given higher burn incidence and unique care requirements. The role of emerging disciplines like burn navigators and peer support specialists warrants evaluation.

\subsection{Policy and System-Level Changes}

Evidence supports policy mandating minimum multidisciplinary team standards for designated burn centers. BRANZ quality indicators should incorporate validated team function measures alongside traditional clinical metrics. Funding models must recognize coordination costs while capturing downstream savings. Medical and nursing education should include multidisciplinary burn care competencies. Accreditation standards should require demonstrated team function, not merely discipline availability.

\section{Study Limitations}

This critically appraised topic has several limitations. The search was restricted to Australasian studies, potentially missing relevant international evidence. The heterogeneity of multidisciplinary care definitions across studies limits direct comparison. Publication bias may favor positive findings about team-based care. Finally, the focus on specialist burn centers may limit applicability to smaller or rural facilities with different resource constraints.