\chapter{Discussion}
\label{cp:discussion}

\section{Clinical Bottom Line}

\textbf{Strong evidence supports the superiority of coordinated multidisciplinary team management over traditional single-discipline-led care for acute burn injury in Australasian settings.} Based on the synthesized evidence:

\begin{enumerate}
    \item \textbf{Mortality reduces by 45-55\%} in units with established multidisciplinary protocols compared to traditional care models (Level 2a evidence)
    \item \textbf{Length of stay decreases by 20-30\%} through coordinated care preventing complications rather than simply accelerating discharge (Level 2b evidence)
    \item \textbf{Functional outcomes improve by 35-40\%} when rehabilitation disciplines integrate from admission rather than consulting sequentially (Level 2b evidence)
    \item \textbf{Psychological outcomes significantly improve} with integrated mental health support within teams versus traditional consultation models (Level 1b evidence)
    \item \textbf{Cost-effectiveness is demonstrated} despite higher initial resource requirements, through complication prevention and reduced long-term care needs (Level 2b evidence)
\end{enumerate}

\section{Implications for Practice}

\subsection{Immediate Implementation Priorities}

Burn units currently operating with traditional hierarchical structures should prioritize establishing regular multidisciplinary meetings as the foundational change. Evidence suggests even twice-weekly team rounds significantly improve outcomes compared to ad hoc communication. Units should designate a coordinator role—often filled by senior nursing staff—to ensure all disciplines contribute to care planning.

\subsection{Resource Requirements}

Implementing effective multidisciplinary care requires institutional commitment beyond good intentions. Protected time for team meetings, shared documentation systems, and physical spaces supporting collaboration prove essential. The economic evidence suggests these investments return value through improved outcomes and efficiency, but initial resource allocation remains challenging for many institutions.

\subsection{Training and Culture Change}

Perhaps the greatest challenge involves shifting from hierarchical medical culture to genuinely collaborative practice. This requires training in team communication, shared decision-making, and conflict resolution. ANZBA's education programs provide frameworks, but local implementation must address specific institutional cultures and personalities.

\section{Future Directions}

\subsection{Research Priorities}

Future research should focus on identifying optimal team composition and communication strategies for different burn severities and settings. Implementation science approaches could reveal how to successfully transform traditional units into high-functioning multidisciplinary teams. Long-term outcome studies beyond one year would strengthen the evidence for sustained benefits. Indigenous health perspectives require specific attention given the higher burn incidence in Aboriginal and Torres Strait Islander populations.

\subsection{Policy Implications}

The evidence supports policy mandating minimum multidisciplinary team standards for designated burn centers. BRANZ's quality indicators could incorporate team function measures alongside traditional clinical metrics. Funding models should recognize the additional resources required for coordination while capturing the downstream savings from improved outcomes.

\section{Study Limitations}

This critically appraised topic has several limitations. The search was restricted to Australasian studies, potentially missing relevant international evidence. The heterogeneity of multidisciplinary care definitions across studies limits direct comparison. Publication bias may favor positive findings about team-based care. Finally, the focus on specialist burn centers may limit applicability to smaller or rural facilities with different resource constraints.