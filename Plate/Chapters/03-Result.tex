\chapter{Results}
\label{cp:results}

\section{Selected Studies}

Twelve studies met inclusion criteria, representing diverse methodological approaches and disciplinary perspectives. \autoref{tab:included-studies} provides a comprehensive overview of the selected studies.

\begin{table}[!htpb]
    \caption{Summary of Included Studies}
    \label{tab:included-studies}
    \begin{tabularx}{\textwidth}{lXllL}
        \toprule
        \textbf{Study} & \textbf{Design} & \textbf{Discipline} & \textbf{Sample Size} & \textbf{Key Findings} \\
        \midrule
        Cleland et al., 2016 (MJA) & Registry analysis & Medicine/Surgery & 7,184 adults & Multidisciplinary units: 45\% lower mortality \\
        Tracy et al., 2022 & Survey study & Multiple disciplines & 70 specialists & Team decisions reduce mortality risk \\
        Edgar et al., 2018 & Cohort study & Physiotherapy & 234 patients & Early team intervention improves function \\
        Gong et al., 2019 & Quality improvement & Nursing & 156 patients & Coordinated protocols reduce infections \\
        Singer et al., 2020 & RCT & Psychology & 89 patients & Integrated psychological care improves QoL \\
        Phillips et al., 2021 & Economic analysis & Health economics & 450 patients & Multidisciplinary care cost-effective \\
        McWilliams et al., 2021 & Implementation study & Telehealth/OT & 67 patients & Virtual MDT feasible for remote areas \\
        Kornhaber et al., 2018 & Systematic review & Nursing & 32 studies & Strong evidence for team coordination \\
        Foster et al., 2019 & Qualitative study & Social work & 45 families & Family support crucial for outcomes \\
        Brown et al., 2023 & Prospective cohort & Dietetics & 120 patients & Nutritional team integration reduces complications \\
        Lee et al., 2020 & Before-after study & Emergency medicine & 200 patients & Early MDT activation improves survival \\
        Wood et al., 2017 & Innovation report & Surgery/Science & 300 patients & Integrated research-clinical teams advance care \\
        \bottomrule
    \end{tabularx}
\end{table}

\section{Critical Appraisal of Evidence}

\subsection{Mortality and Survival Outcomes}

The most compelling evidence emerges from BRANZ registry analysis by Cleland and colleagues (2016), demonstrating that specialized burn units with established multidisciplinary teams achieved 45\% lower risk-adjusted mortality compared to centers using traditional care models. This population-level evidence, encompassing 7,184 adult admissions over five years, provides robust support for team-based approaches. The mortality benefit persisted after adjusting for injury severity, age, and comorbidity burden, suggesting that care coordination rather than patient selection explains the improved outcomes.

Tracy et al. (2022) surveyed 70 burn specialists across Australian and New Zealand centers, revealing that 94\% reported improved survival when critical decisions involved the full multidisciplinary team rather than individual practitioners. While survey evidence ranks lower than empirical outcomes data, this study provides important insight into the mechanisms through which multidisciplinary care might improve survival—through collective expertise, error reduction, and comprehensive assessment of complex cases.

\subsection{Length of Stay and Complications}

Multiple studies demonstrate reduced length of stay with coordinated multidisciplinary protocols. Gong et al. (2019) implemented nursing-led coordination protocols integrated with medical, therapy, and surgical teams, achieving 23\% reduction in average length of stay and 41\% reduction in wound infection rates. The economic analysis by Phillips et al. (2021) confirmed that while multidisciplinary care requires greater upfront resource investment, total costs decrease through prevention of complications and reduced readmissions.

Brown et al. (2023) specifically examined nutritional support within multidisciplinary frameworks, finding that integrated dietetic involvement from admission reduced septic complications by 34\% compared to traditional consultation models. This exemplifies how each discipline's expertise, when coordinated effectively, contributes to overall outcome improvement.

\subsection{Functional Recovery Outcomes}

Edgar and colleagues (2018) provide compelling evidence that early, coordinated rehabilitation within multidisciplinary teams significantly improves functional outcomes. Their prospective cohort study of 234 burn survivors demonstrated that patients receiving integrated physiotherapy and occupational therapy from admission achieved 40\% better Functional Independence Measure scores at discharge compared to those receiving traditional sequential consultations.

The systematic review by Kornhaber et al. (2018), analyzing 32 studies including 18 from Australasian centers, concluded that functional outcomes consistently improve when rehabilitation disciplines integrate with acute medical care rather than operating in parallel. Return-to-work rates reached 79\% with comprehensive team support versus 52\% with fragmented care approaches.

\subsection{Psychosocial Outcomes}

Singer et al. (2020) conducted the first Australasian randomized controlled trial comparing integrated psychological support within burn teams versus traditional psychiatric consultation models. Patients receiving integrated psychological care showed significantly reduced post-traumatic stress symptoms (Cohen's d = 0.82) and improved quality of life scores at six months. Importantly, early psychological integration also correlated with better engagement in physical rehabilitation, illustrating the interconnected nature of burn recovery.

Foster et al. (2019) qualitative study with 45 families highlighted how social work coordination within multidisciplinary teams addresses the broader systemic challenges facing burn patients—insurance navigation, family support, housing modifications, and return-to-work planning. Families reported feeling ``held'' by the team approach versus ``lost'' in fragmented care systems.

\subsection{Implementation and Feasibility}

McWilliams et al. (2021) demonstrated that multidisciplinary burn care can be successfully delivered via telehealth to remote communities, with virtual team meetings achieving similar coordination benefits to in-person rounds. This innovation proves particularly relevant for Australia and New Zealand's geographically dispersed populations.

The innovation report by Wood et al. (2017) from Royal Perth Hospital illustrates how integration of research scientists within clinical multidisciplinary teams accelerates translation of discoveries like spray-on skin technology into practice. This model, where laboratory and bedside merge within team structures, represents a uniquely successful Australasian contribution to global burn care.

\section{Limitations of Current Evidence}

Several limitations constrain the current evidence base. First, true randomization to multidisciplinary versus single-discipline care proves ethically challenging once benefits become apparent. Most studies employ observational designs susceptible to confounding. Second, defining and measuring ``multidisciplinary care'' varies between studies, making direct comparison difficult. Third, publication bias likely favors positive findings about team-based care. Fourth, most studies focus on major burn centers, limiting generalizability to smaller facilities. Finally, long-term outcomes beyond one year remain understudied.