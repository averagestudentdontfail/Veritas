% Chapters/03-Result.tex

\chapter{Results}
\label{cp:results}

\section{Selected Studies}

Fifteen studies met inclusion criteria for detailed critical appraisal, representing diverse methodological approaches and disciplinary perspectives across Australian and New Zealand burn centers. The studies span from \textcite{Gabbe2015}'s feasibility pilot through to \textcite{Tracy2025}'s comprehensive long-term outcome evaluation.

% Landscape table using landscapemode
\begin{landscapemode}{297mm}{210mm}
\begin{table}[p]
    \caption{Summary of Included Studies for Critical Appraisal}
    \label{tab:included-studies}
    \small
    \setlength{\tabcolsep}{4pt}
    \begin{tabularx}{\linewidth}{p{2.8cm}p{2.5cm}p{3cm}lX}
        \toprule
        \textbf{Study} & \textbf{Design} & \textbf{Setting} & \textbf{Sample} & \textbf{Key Findings} \\
        \midrule
        \textcite{Cleland2016} & Registry analysis & 10 \gls{branz} units & 7,184 adults & Established \gls{mdt} as standard across all units; significant variation in implementation completeness \\
        
        \textcite{Tracy2022adherence} & Registry analysis & 17 \gls{branz} units & 10,884 patients & Allied health assessment within 48hr achieved in 97\% of cases with structured protocols \\
        
        \textcite{Tracy2025} & Prospective cohort & 3 burn centers & 342 patients & \Gls{qol} and \gls{rtw} maintained at 2 years post-injury with coordinated follow-up \\
        
        \textcite{Gong2021} & Quality indicators & 17 \gls{branz} units & 31,498 patients & 23 evidence-based \gls{mdt} quality measures defined and validated across all centers \\
        
        \textcite{Hunter2024} & Prospective cohort & Queensland & 156 Indigenous children & \Gls{culturalsafety} integration reduces length of stay by 2.8 days (95\% \gls{ci} 1.2-4.4) \\
        
        \textcite{Coombes2020} & Qualitative study & Queensland & 18 Indigenous families & First Nations Health Workers critical for \gls{mdt} coordination and family engagement \\
        
        \textcite{Plaza2022} & \Gls{rct} & Adelaide & 45 patients & \Gls{telehealth} non-inferior to in-person \gls{mdt} rehabilitation for functional outcomes \\
        
        \textcite{Kurmis2022} & Cohort study & ANZ centers & 255 major burns & Early nutrition within \gls{mdt} framework reduces complications by 34\% \\
        
        \textcite{Gabbe2015} & Pilot study & Victoria & 150 patients & Framework for long-term \gls{mdt} outcome evaluation established and validated \\
        
        \textcite{Cleland2022} & Economic analysis & Victoria & 331 severe burns & \Gls{mdt} daily costs 18\% higher but total episode costs 22\% lower \\
        
        \textcite{Cassidy2015} & Retrospective cohort & \gls{branz} units & 2,892 patients & Pre-hospital coordination affects \gls{mdt} activation and outcomes significantly \\
        
        \textcite{Singer2022} & Observational & 6 centers & 866 admissions & Out-of-hours \gls{mdt} availability impacts mortality and complication rates \\
        
        \textcite{Tracy2020} & Prospective cohort & \gls{branz} units & 328 patients & \gls{mdt} pain management predicts 12-month pain and itch severity outcomes \\
        
        \textcite{Fitts2023} & Mixed methods & Remote Australia & 89 patients & \Gls{telehealth} enables \gls{mdt} care delivery in remote settings successfully \\
        
        \textcite{McPhail2022} & Economic evaluation & Queensland & 198 patients & \Gls{mdt} scar management cost-effective at \gls{aud} \$21,000 per \gls{qaly} gained \\
        \bottomrule
    \end{tabularx}
\end{table}
\end{landscapemode}

\section{Geographic Distribution of Burn Services}

The \gls{branz} registry encompasses 17 specialist \glspl{burnunit} distributed across Australia and New Zealand, serving populations across vast geographic areas. The distribution reflects population density patterns while ensuring reasonable access for remote communities through hub-and-spoke models and \gls{telehealth} services.

% Accurate geographic distribution map using landscapemode
\begin{landscapemode}{297mm}{210mm}
\begin{figure}[p]
    \centering
    \caption{Geographic distribution of specialist \glspl{burnunit} participating in \gls{branz} (2024)}
    \label{fig:burncenters}
    \begin{tikzpicture}[scale=1.2]
        % Draw more accurate Australia outline
        \draw[thick, fill=gray!10] 
            % Western coast
            (0,3) .. controls (0.5,4) and (1,4.5) .. (2,5) % Northwest
            .. controls (3,5.2) and (4,5.3) .. (5,5) % North coast
            .. controls (6,4.8) and (7,4.9) .. (8,5.2) % Northeast
            .. controls (8.5,4.5) and (8.3,3.5) .. (8,3) % East coast
            .. controls (7.8,2) and (7.5,1) .. (7,0.5) % Southeast
            .. controls (6,0.3) and (5,0.2) .. (4,0.3) % South coast
            .. controls (3,0.4) and (2,0.5) .. (1,1) % Southwest
            .. controls (0.3,2) and (0,2.5) .. (0,3); % West coast back to start
        
        % Draw Tasmania
        \draw[thick, fill=gray!10] (6.5,-0.8) ellipse (0.3 and 0.25);
        
        % Draw more accurate New Zealand
        \draw[thick, fill=gray!10] 
            % North Island
            (10,4.5) .. controls (10.2,4.8) and (10.3,4.6) .. (10.4,4.3)
            .. controls (10.5,3.8) and (10.3,3.4) .. (10.2,3.2)
            .. controls (10,3) and (9.9,3.1) .. (9.8,3.3)
            .. controls (9.7,3.8) and (9.8,4.2) .. (10,4.5);
        
        \draw[thick, fill=gray!10]
            % South Island
            (9.7,2.8) .. controls (9.9,2.9) and (10,2.7) .. (10.1,2.4)
            .. controls (10.2,1.8) and (10,1.2) .. (9.8,1)
            .. controls (9.6,0.9) and (9.4,1.1) .. (9.3,1.4)
            .. controls (9.2,2) and (9.4,2.6) .. (9.7,2.8);
        
        % Australian burn centers with accurate locations
        % Western Australia
        \node[circle,fill=red!80,inner sep=3pt] (perth1) at (1.2,2.2) {};
        \node[below=2pt of perth1, font=\footnotesize] {Perth Children's};
        \node[circle,fill=red!80,inner sep=3pt] (perth2) at (1.2,1.8) {};
        \node[below=2pt of perth2, font=\footnotesize] {Fiona Stanley};
        
        % Northern Territory
        \node[circle,fill=red!80,inner sep=3pt] (darwin) at (4.5,4.2) {};
        \node[right=2pt of darwin, font=\footnotesize] {Royal Darwin};
        
        % Queensland
        \node[circle,fill=red!80,inner sep=3pt] (brisbane1) at (7.5,2.5) {};
        \node[right=2pt of brisbane1, font=\footnotesize] {RBWH};
        \node[circle,fill=red!80,inner sep=3pt] (brisbane2) at (7.5,2.2) {};
        \node[right=2pt of brisbane2, font=\footnotesize] {QCH};
        \node[circle,fill=red!80,inner sep=3pt] (townsville) at (7.2,3.8) {};
        \node[left=2pt of townsville, font=\footnotesize] {Townsville};
        
        % New South Wales
        \node[circle,fill=red!80,inner sep=3pt] (sydney1) at (7.3,1.5) {};
        \node[right=2pt of sydney1, font=\footnotesize] {RPA};
        \node[circle,fill=red!80,inner sep=3pt] (sydney2) at (7.3,1.2) {};
        \node[right=2pt of sydney2, font=\footnotesize] {Concord};
        \node[circle,fill=red!80,inner sep=3pt] (sydney3) at (7.3,0.9) {};
        \node[right=2pt of sydney3, font=\footnotesize] {Westmead Kids};
        
        % Victoria
        \node[circle,fill=red!80,inner sep=3pt] (melbourne1) at (6.2,0.6) {};
        \node[below=2pt of melbourne1, font=\footnotesize] {Alfred};
        \node[circle,fill=red!80,inner sep=3pt] (melbourne2) at (5.9,0.6) {};
        \node[below=2pt of melbourne2, font=\footnotesize] {RCH};
        
        % South Australia
        \node[circle,fill=red!80,inner sep=3pt] (adelaide1) at (4.8,1.2) {};
        \node[below=2pt of adelaide1, font=\footnotesize] {RAH};
        \node[circle,fill=red!80,inner sep=3pt] (adelaide2) at (4.5,1.2) {};
        \node[below=2pt of adelaide2, font=\footnotesize] {WCH};
        
        % Tasmania
        \node[circle,fill=red!80,inner sep=3pt] (hobart) at (6.5,-0.8) {};
        \node[below=2pt of hobart, font=\footnotesize] {Royal Hobart};
        
        % New Zealand burn centers
        \node[circle,fill=blue!80,inner sep=3pt] (auckland) at (10.1,4.1) {};
        \node[right=2pt of auckland, font=\footnotesize] {Middlemore};
        \node[circle,fill=blue!80,inner sep=3pt] (hamilton) at (10,3.7) {};
        \node[right=2pt of hamilton, font=\footnotesize] {Waikato};
        \node[circle,fill=blue!80,inner sep=3pt] (hutt) at (10,3.3) {};
        \node[right=2pt of hutt, font=\footnotesize] {Hutt Valley};
        \node[circle,fill=blue!80,inner sep=3pt] (christchurch) at (9.9,1.6) {};
        \node[right=2pt of christchurch, font=\footnotesize] {Christchurch};
        
        % Legend
        \node[draw, rectangle, minimum width=4cm, minimum height=1.5cm] at (2.5,-1.5) {
            \begin{tabular}{cl}
                \tikz\fill[red!80] circle (3pt); & Australian Centers (n=13) \\
                \tikz\fill[blue!80] circle (3pt); & New Zealand Centers (n=4)
            \end{tabular}
        };
        
        % Scale indicator
        \draw[|-|] (0,-2) -- (2,-2);
        \node[below] at (1,-2) {1000 km};
    \end{tikzpicture}
\end{figure}
\end{landscapemode}

The geographic distribution reveals concentration of services in major population centers along the eastern seaboard of Australia and both islands of New Zealand. Western Australia's two centers in Perth serve the entire western third of the continent, while single centers in Darwin and Townsville provide services to Australia's tropical north. The hub-and-spoke model enables these centers to coordinate care across vast distances through retrieval services and \gls{telehealth} partnerships with regional hospitals.