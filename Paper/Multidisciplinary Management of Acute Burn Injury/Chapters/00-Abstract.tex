% Chapters/00-Abstract.tex

\thispagestyle{plain}

\pdfbookmark[1]{Abstract}{abstract}
\chapter*{Abstract}

\textbf{Background:} Acute burn injury represents one of the most complex medical emergencies, requiring coordinated intervention across multiple healthcare disciplines. While \gls{multidisciplinary} approaches have become standard in Australasian \glspl{burnunit} since the early 2000s, the empirical evidence base comparing these models to traditional single-discipline care requires careful evaluation.

\textbf{Objective:} To critically appraise current Australasian evidence examining whether coordinated \gls{multidisciplinary} management improves clinical outcomes compared to traditional single-discipline-led care in adults and children with acute burn injury.

\textbf{Methods:} Comprehensive literature search following \gls{prisma} guidelines across PubMed, CINAHL, Cochrane Library, EMBASE, and \gls{branz} publications (January 2014-February 2025). Search strategy identified 287 articles, with 184 screened after duplicate removal, 76 full-text articles assessed, and 43 meeting inclusion criteria. Evidence synthesis focused on 15 highest-quality studies, with assessment based on Oxford Centre for Evidence-Based Medicine criteria.

\textbf{Results:} No studies directly compared \gls{multidisciplinary} versus traditional single-discipline care, as \gls{mdt} management has become universal standard practice across all 17 Australasian \glspl{burnunit}. Registry data from 31,498 patients demonstrates that units with comprehensive \gls{mdt} protocols achieve superior outcomes: 97\% of patients receive allied health assessment within 48 hours, functional independence scores improve by 35-40\%, and long-term \gls{qol} measures show sustained benefits. Implementation studies confirm feasibility across diverse settings including remote communities via \gls{telehealth}. Aboriginal and Torres Strait Islander populations show particular benefit when \gls{culturalsafety} principles integrate within \gls{mdt} frameworks.

\textbf{Conclusions:} While direct comparative evidence is absent due to universal \gls{mdt} adoption, convergent evidence from registry analyses, quality improvement studies, and implementation research strongly supports coordinated \gls{multidisciplinary} management as optimal care standard for acute burn injury. The absence of contemporary single-discipline comparators paradoxically validates \gls{mdt} effectiveness through its complete acceptance across Australasian burn services.

\keywordsen{burn injury, multidisciplinary care, team-based management, Australia, New Zealand, BRANZ, Indigenous health}

\MediaOptionLogicBlank
