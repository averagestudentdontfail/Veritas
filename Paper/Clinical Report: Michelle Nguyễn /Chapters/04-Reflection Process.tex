% Chapters/03-Reflection Process.tex

\chapter{Reflection Process}
\label{cp:reflection process}

\section{Areas of Strength}

During this initial assessment, I successfully established rapport with Michelle despite her stated reluctance to attend therapy. My opening approach acknowledging she had not seen a psychologist before and explaining confidentiality in accessible terms helped reduce initial resistance and established a foundation of transparency. I demonstrated cultural sensitivity by allowing Michelle to share her perspective on family dynamics without immediately pathologizing Vietnamese parenting styles, recognizing these as acculturation tensions rather than dysfunction requiring correction.

\section{Areas of Professional Development}

Reflecting on this session, I recognize several areas requiring development in my clinical practice. I could have explored Michelle's strengths more systematically throughout the assessment. While she mentioned work success, I did not fully investigate what specific skills or qualities contribute to this achievement, representing a missed opportunity for strength-based intervention planning that could build self-efficacy and hope.

My exploration of protective factors was insufficient for comprehensive risk assessment. I should have asked more about her friendship network dynamics, specific coping strategies beyond alcohol use, and cultural strengths within her family system that might serve as resources for intervention. When Michelle mentioned friends with "progressive" parents, I could have explored these alternative family models as potential sources of support or mentorship.

I notice I did not adequately assess Michelle's readiness for change or explore her own goals beyond wanting to be "heard" by her family. Questions exploring what specific changes she would like to see and what she might be willing to do differently could have elicited more collaborative treatment planning and increased engagement. Additionally, I could have been more curious about her experience as a Vietnamese-Australian teenager navigating dual cultural identities, exploring sources of pride and connection rather than focusing primarily on conflicts.

The risk assessment, while covering immediate safety concerns, could have explored protective factors more thoroughly to provide a more balanced clinical picture. I also recognize that my closing felt somewhat rushed, and I could have spent more time ensuring Michelle felt heard and validating her courage in attending despite her reluctance and ambivalence about therapy.

