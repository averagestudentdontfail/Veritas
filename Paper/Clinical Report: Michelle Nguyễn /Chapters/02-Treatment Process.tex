% Chapters/02-Treatment Process.tex

\chapter{Treatment Process}
\label{cp:treatment process}

\section{Clinical Impression}

Michelle provisionally presents with Adjustment Disorder with mixed anxiety and depressed mood (F43.23) in the context of significant family dysfunction and acculturation stress. Her difficulties reflect systemic family problems rather than individual pathology, indicating family-based intervention as the primary treatment approach. Differential considerations include monitoring for potential Alcohol Use Disorder development and Social Anxiety Disorder features.

\section{Evidence-Based Treatment}

The following intervention plan draws from the California Evidence-Based Clearinghouse for Child Welfare (CEBC) registry and follows the Exploration, Preparation, Implementation, and Sustainment (EPIS) framework as per \textcite{Aaron2010}.

\begin{landscapelongtable}
\begin{longtable}{@{}p{1.5cm}p{3cm}p{3.5cm}p{4cm}p{3.5cm}@{}}
    \caption{Five-Session Intervention Plan Using Brief Strategic Family Therapy}
    \label{tab:intervention-plan}\\
   
    % First header
    \toprule
    \textbf{Session} & \textbf{Participants} & \textbf{Objectives} & \textbf{Key Activities} & \textbf{Rationale} \\
    \midrule
    \endfirsthead
   
    % Continued header for subsequent pages
    \multicolumn{5}{c}{\tablename\ \thetable{} -- \textit{Continued from previous page}} \\
    \toprule
    \textbf{Session} & \textbf{Participants} & \textbf{Objectives} & \textbf{Key Activities} & \textbf{Rationale} \\
    \midrule
    \endhead
   
    % Footer for pages except last
    \midrule
    \multicolumn{5}{r}{\textit{Continued on next page}} \\
    \endfoot
   
    % Footer for last page
    \bottomrule
    \endlastfoot
   
    % Table content
    2 & Michelle (30 min), then parents (30 min) & Complete assessment; Provide psychoeducation; Establish alliance & Administer DASS-21 and Acculturative Stress Scale; Psychoeducation on anxiety-alcohol connection; Introduce BSFT framework to parents & Following BSFT joining phase, builds alliance while gathering baseline data and reducing blame through education \\
    
    3 & Full family & Identify interaction patterns; Map family structure & Conduct enactment exercise observing family discussion; Create family genogram with immigration history; Identify repetitive problematic sequences & Direct observation provides richer data than reports; Understanding patterns precedes change attempts \\
    
    4 & Michelle and parents & Restructure communication; Address cultural gaps & Practice adapted I-statements for hierarchical context; Develop cultural compromise contracts; Role-play validation before limit-setting & Targets specific interactions while respecting cultural values and preserving parental authority \\
    
    5 & Michelle (20 min), family (40 min) & Consolidate changes; Address substance use; Safety planning & Develop anxiety management strategies; Collaborative problem-solving exercise; Create family safety plan; Schedule maintenance sessions & Ensures changes transfer beyond therapy while addressing risk factors within family context \\
    
    6 & Full family & Review progress; Adjust interventions; Plan sustainment & Review contract compliance; Adjust agreements based on outcomes; Develop long-term maintenance plan & Promotes sustained change and prevents relapse to previous patterns \\
    
\end{longtable}
\end{landscapelongtable}