\chapter{Treatment Process}
\label{cp:treatment process}

\section{Clinical Impression}

Michelle presents with Adjustment Disorder with Mixed Anxiety and Depressed Mood (F43.23) within the context of family dysfunction and acculturation stress. Differential considerations include monitoring for potential Alcohol Use Disorder development, and for Personality, Trauma and Anxiety features. Her difficulties appear to reflect systemic family problems requiring family-based intervention.

\section{Evidence-Based Treatment Plan}

The intervention follows Brief Strategic Family Therapy (BSFT) principles, selected from the California Evidence-Based Clearinghouse for Child Welfare registry. BSFT addresses family interaction patterns while respecting cultural values.

\textbf{Session 2:} Individual assessment with Michelle (30 minutes) followed by parent engagement (30 minutes). Objectives include administering DASS-21 and Acculturative Stress Scale, providing psychoeducation about anxiety-alcohol connections, and introducing BSFT framework to parents. This follows BSFT joining phase principles to establish therapeutic alliance while gathering baseline data.

\textbf{Session 3:} Full family participation focusing on identifying interaction patterns through enactment exercises observing family discussion. Activities include creating a family genogram incorporating immigration history and identifying repetitive problematic sequences. Direct observation provides richer assessment data than self-reports alone.

\textbf{Session 4:} Michelle and parents together to restructure communication patterns and address cultural gaps. Activities include practicing adapted I-statements appropriate for hierarchical family context, developing cultural compromise contracts, and role-playing validation techniques before limit-setting. This targets specific problematic interactions while maintaining respect for Vietnamese cultural values and parental authority.

\textbf{Session 5:} Brief individual check-in with Michelle (20 minutes) followed by family session (40 minutes). Focus includes consolidating changes, addressing substance use directly, and safety planning. Activities involve developing anxiety management strategies, conducting collaborative problem-solving exercises, creating a family safety plan, and scheduling maintenance sessions.

\textbf{Session 6:} Full family review session to assess progress, adjust interventions based on outcomes, and develop long-term maintenance strategies. Activities include reviewing contract compliance, adjusting agreements as needed, and creating sustainment plans to prevent relapse to previous patterns.