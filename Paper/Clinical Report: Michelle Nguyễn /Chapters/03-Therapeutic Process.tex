\chapter{Therapeutic Process}
\label{cp:therapeutic process}

\section{Cultural Compromise Contract}

The Cultural Compromise Contract represents a structured negotiation exercise addressing the core tension between Michelle's developmental autonomy needs and Vietnamese family values. This intervention transforms abstract cultural conflicts into concrete, manageable agreements.

\textbf{Objective:} Create specific, measurable agreements balancing Michelle's independence with parental values, reducing daily conflicts while preserving family harmony and cultural identity.

\textbf{Rationale:} Research demonstrates youth successfully integrating dual cultures show better psychological adjustment than those forced to choose between cultural frameworks. The written contract format appeals to Vietnamese preference for formal agreements while the negotiation process reflects Australian democratic values.

\subsection{Implementation Protocol}

The session begins with a 10-minute introduction framing Vietnamese values (family harmony, respect, academic achievement) and Australian values (independence, self-expression, peer relationships) as complementary rather than contradictory—a "both/and" rather than "either/or" situation.

During the 15-minute values clarification phase, each family member identifies their top three values regarding the negotiated issue. Parents might prioritize safety, reputation, and academic focus; Michelle might emphasize trust, social connection, and independence.

The proposal development phase (10 minutes) involves each party writing specific proposals. For social outings, Michelle might propose Friday nights until 11:00 PM with friends. Parents might counter with homework completion first, 9:00 PM return, and arrival/departure calls.

The negotiation phase (15 minutes) utilizes therapist mediation to find middle ground—perhaps Friday nights until 10:00 PM with completed homework, monthly 11:00 PM special events, and bi-hourly text updates.

The final phase (10 minutes) involves drafting a formal contract specifying agreements, trial period duration, review date, violation consequences, and compliance rewards.

\subsection{Developmental and Cultural Considerations}

This approach provides scaffolded independence appropriate for 16-year-olds through expanding boundaries over time. The concrete format suits adolescent cognitive development. Parents maintain authority as contract signatories while Michelle gains voice in term creation, with the therapist serving as culturally-informed mediator.