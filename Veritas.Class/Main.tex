%%% Veritas Pandoc Template %%%
%%% Converts Markdown to LaTeX using Veritas class
%%%
%%% YAML Frontmatter Variables:
%%%   title: Document title
%%%   subtitle: Document subtitle (optional)
%%%   author: Author name or list
%%%   date: Publication date
%%%   lang: en or de
%%%   chapterstyle: minimal, classic, fancy, modern
%%%   abstract: Abstract text (optional)

\documentclass[
    language=en,
    chapterstyle=minimal,
    media=screen,
    doctype=article
]{Veritas}

%%% Metadata from YAML %%%
\Title{Cyclotomic Polynomials}
\Subtitle{A Complete Treatment}
\FirstAuthor{}
\Date{December 2025}

%%% Pandoc Compatibility %%%
\providecommand{\tightlist}{%
  \setlength{\itemsep}{0pt}\setlength{\parskip}{0pt}}

% CSL References for Pandoc citeproc
\newlength{\cslhangindent}
\setlength{\cslhangindent}{1.5em}
\newlength{\csllabelwidth}
\setlength{\csllabelwidth}{3em}
\newenvironment{CSLReferences}[2]%
  {\clearpage%
   \section*{References}%
   \addcontentsline{toc}{section}{References}%
   \setlength{\parindent}{0pt}%
   \everypar{\setlength{\hangindent}{\cslhangindent}}\ignorespaces}%
  {\par}
\newcommand{\CSLBlock}[1]{\hfill\break#1\hfill\break}
\newcommand{\CSLLeftMargin}[1]{\parbox[t]{\csllabelwidth}{\strut#1\strut}}
\newcommand{\CSLRightInline}[1]{\parbox[t]{\dimexpr\linewidth-\csllabelwidth\relax}{\strut#1\strut}}
\newcommand{\CSLIndent}[1]{\hspace{\cslhangindent}#1}

\begin{document}

%%% Title Page %%%
\pagenumbering{gobble}
%%% Matter/Title-Page.tex %%%
%%% Unified Title Page for Veritas Template
%%%
%%% FIRST PRINCIPLES: All fields are optional.
%%% Only defined metadata will be displayed.

\newcommand\BackgroundPicTitlePage{%
    \put(0,0){%
    \parbox[b][\paperheight]{\paperwidth}{%
    \vfill
    \centering
    \IfFileExists{Figures/Theme/Front-Page-BG.pdf}{%
        \includegraphics[width=\paperwidth,height=\paperheight,keepaspectratio]{Figures/Theme/Front-Page-BG.pdf}%
    }{}%
    \vfill
}}}
\AddToShipoutPictureBG*{\BackgroundPicTitlePage}

\newgeometry{margin=2.5cm, top=3cm, bottom=2.5cm}
\begin{titlepage}
    \miniondisplayfont
    \color{frontpagedark}
    
    \vspace*{2\baselineskip}
    
    %%% Title (if defined) %%%
    \ifdefined\GetTitle
        \noindent
        \makebox[\textwidth][l]{%
            \parbox{\dimexpr\textwidth-1cm\relax}{%
                \setstretch{1.05}%
                \raggedright\bfseries\fontsize{24}{30}\selectfont\GetTitle
            }%
        }
        \vspace{0.8\baselineskip}
    \fi
    
    %%% Subtitle (if defined) %%%
    \ifdefined\GetSubtitle
        \noindent
        \makebox[\textwidth][l]{%
            \parbox{\dimexpr\textwidth-3cm\relax}{%
                \setstretch{1.03}%
                \raggedright\fontsize{14}{18}\selectfont\itshape\GetSubtitle
            }%
        }
        \vspace{1.5\baselineskip}
    \fi
    
    \vspace{1\baselineskip}
    
    %%% Authors (if defined) %%%
    \ifdefined\GetFirstAuthor
        {\noindent\fontsize{14}{18}\selectfont\GetFirstAuthor}
        
        % Show student number only if explicitly defined
        \ifdefined\GetFirstAuthorNumber
            \ifdefined\ShowStudentNumber
                \\[2pt]
                {\noindent\fontsize{10}{12}\selectfont\itshape\GetFirstAuthorNumber}
            \fi
        \fi
        \vspace{8pt}
    \fi
    
    \ifdefined\GetSecondAuthor
        \\[4pt]
        {\noindent\fontsize{14}{18}\selectfont\GetSecondAuthor}
    \fi
    
    \ifdefined\GetThirdAuthor
        \\[4pt]
        {\noindent\fontsize{14}{18}\selectfont\GetThirdAuthor}
    \fi
    
    %%% Supervisor Section (if any supervisor defined) %%%
    \ifdefined\GetSupervisor
        \vspace{2\baselineskip}
        {
        \noindent
        \fontsize{10}{12}\selectfont
        \renewcommand{\arraystretch}{0.1}
        \hspace*{-2.5pt}\begin{tabular}{@{}r@{\hspace{5pt}}>{\raggedright\arraybackslash}m{6cm}@{}}
            \textbf{Supervisor:} & \GetSupervisor \\ [-.7ex]
            \ifdefined\GetSupervisorTitle
                & \setstretch{0.9}{\fontsize{8}{10}\selectfont\itshape \GetSupervisorTitle} \\ [2ex]
            \fi
            
            \ifdefined\GetCoSupervisor
                \textbf{Co-supervisor:} & \GetCoSupervisor \\ [-.7ex]
                \ifdefined\GetCoSupervisorTitle
                    & \setstretch{0.9}{\fontsize{8}{10}\selectfont\itshape \GetCoSupervisorTitle} \\ [.5ex]
                \fi
            \fi

            \ifdefined\GetSecCoSupervisor        
                & \GetSecCoSupervisor \\ [-.7ex]
                \ifdefined\GetSecCoSupervisorTitle
                    & \setstretch{0.9}{\fontsize{8}{10}\selectfont\itshape \GetSecCoSupervisorTitle} \\
                \fi
            \fi
        \end{tabular}
        }
    \fi
    
    \vfill
    
    %%% Institution Section (if any defined) %%%
    \ifdefined\GetUniversity
        {\noindent\fontsize{10}{12}\selectfont\GetUniversity}\\[2pt]
    \fi
    
    \ifdefined\GetSchool
        {\noindent\fontsize{10}{12}\selectfont\GetSchool}\\[2pt]
    \fi
    
    \ifdefined\GetDepartment
        {\noindent\fontsize{10}{12}\selectfont\GetDepartment}\\[2pt]
    \fi
    
    \ifdefined\GetDegree
        {\noindent\fontsize{10}{12}\selectfont\GetDegree}\\[2pt]
    \fi

    \ifdefined\GetCourse
        {\noindent\fontsize{10}{12}\selectfont\GetCourse}
    \fi

    %%% Document Type (if defined) %%%
    \ifdefined\GetThesisType
        \vspace{1.5\baselineskip}
        {\noindent\fontsize{10}{12}\itshape\selectfont\GetThesisType}
    \fi

    %%% Working Draft Version (only in working stage) %%%
    \ifthenelse{\equal{\DocStageOption}{working}}{%
        \vspace{1.5\baselineskip}
        \ifdefined\GetDocumentVersion
            {\noindent\fontsize{10}{12}\selectfont\textcolor{burgundy}{\GetDocumentVersion}}
        \fi
    }{}

    %%% Date (if defined) %%%
    \ifdefined\GetDate
        \vspace{1.5\baselineskip}
        {\noindent\fontsize{10}{12}\selectfont\GetDate}
    \fi

    \vspace{2\baselineskip}
\end{titlepage}
\restoregeometry
\MediaOptionLogicBlank


%%% Abstract (if present) %%%

%%% Main Content %%%
\pagenumbering{arabic}
\pagestyle{mainmatter}

% Fix section numbering for article mode (remove chapter prefix)
\renewcommand{\thesection}{\arabic{section}}
\renewcommand{\thesubsection}{\thesection.\arabic{subsection}}
\renewcommand{\thesubsubsection}{\thesubsection.\arabic{subsubsection}}

\hypertarget{preliminaries}{%
\section{Preliminaries}\label{preliminaries}}

\textbf{Definition 1.1 (Primitive Root of Unity).} Let n ≥ 1 be a
positive integer. A complex number ζ is a \emph{primitive n-th root of
unity} if ζⁿ = 1 and ζᵏ ≠ 1 for all 1 ≤ k \textless{} n.~Equivalently, ζ
is primitive if and only if ord(ζ) = n in the multiplicative group ℂˣ.

The n-th roots of unity are precisely ζₙᵏ := e\^{}\{2πik/n\} for k = 0,
1, \ldots, n−1. Among these, ζₙᵏ is primitive if and only if gcd(k, n) =
1. The number of primitive n-th roots of unity is therefore φ(n),
Euler's totient function.

\textbf{Definition 1.2 (Cyclotomic Polynomial).} The \emph{n-th
cyclotomic polynomial} is defined as:

\[\Phi_n(x) := \prod_{\substack{1 \leq k \leq n \\ \gcd(k,n) = 1}} \left( x - \zeta_n^k \right)\]

where ζₙ = e\^{}\{2πi/n\}. Equivalently, Φₙ(x) is the minimal polynomial
over ℚ of any primitive n-th root of unity.

By construction, Φₙ(x) is a monic polynomial of degree φ(n) with roots
precisely the primitive n-th roots of unity.

\textbf{Index convention.} For a divisor d \textbar{} n, each n-th root
of unity ζₙᵏ is a primitive d-th root of unity for exactly one d, namely
d = n/gcd(k, n). This partitions the roots of xⁿ − 1 by their primitive
order.

\begin{center}\rule{0.5\linewidth}{0.5pt}\end{center}

\hypertarget{the-fundamental-factorisation}{%
\section{The Fundamental
Factorisation}\label{the-fundamental-factorisation}}

\textbf{Theorem 2.1 (Divisor Product Identity).} For all positive
integers n:

\[x^n - 1 = \prod_{d \mid n} \Phi_d(x)\]

\emph{Proof.} Both sides are monic polynomials of degree n.~The left
side has roots \{ζₙᵏ : 0 ≤ k ≤ n−1\}. Each such root ζₙᵏ has
multiplicative order d := n/gcd(k, n), which divides n.~Thus ζₙᵏ is a
primitive d-th root of unity and hence a root of Φ\_d(x).

Conversely, every primitive d-th root of unity (for d \textbar{} n)
satisfies ζᵈ = 1, hence ζⁿ = (ζᵈ)\^{}\{n/d\} = 1, so it is an n-th root
of unity.

This establishes a bijection between the roots of xⁿ − 1 and the union
⋃\_\{d \textbar{} n\} \{roots of Φ\_d(x)\}. Since the Φ\_d(x) have
pairwise disjoint root sets (a primitive d-th root has order exactly d),
and since

\[\sum_{d \mid n} \deg \Phi_d(x) = \sum_{d \mid n} \phi(d) = n,\]

the factorisation follows from unique factorisation in ℂ{[}x{]}. ∎

\textbf{Corollary 2.2.} The cyclotomic polynomial admits the recursive
formula:

\[\Phi_n(x) = \frac{x^n - 1}{\displaystyle\prod_{\substack{d \mid n \\ d < n}} \Phi_d(x)}\]

\begin{center}\rule{0.5\linewidth}{0.5pt}\end{center}

\hypertarget{integrality-of-coefficients}{%
\section{Integrality of
Coefficients}\label{integrality-of-coefficients}}

\textbf{Lemma 3.1 (Monic Division in ℤ{[}x{]}).} Let f(x), g(x) ∈
ℤ{[}x{]} with g(x) monic. If g(x) \textbar{} f(x) in ℚ{[}x{]}, then g(x)
\textbar{} f(x) in ℤ{[}x{]}; that is, f(x)/g(x) ∈ ℤ{[}x{]}.

\emph{Proof.} By the division algorithm in ℚ{[}x{]}, write f(x) =
g(x)q(x) + r(x) where q(x), r(x) ∈ ℚ{[}x{]} and deg r \textless{} deg g.
Since g \textbar{} f in ℚ{[}x{]}, we have r(x) = 0, so f(x) = g(x)q(x).

We claim q(x) ∈ ℤ{[}x{]}. Suppose not; let q(x) = ∑\_\{i=0\}\^{}\{m\}
qᵢxⁱ with some qᵢ ∉ ℤ. Write qᵢ = aᵢ/bᵢ in lowest terms. Let p be a
prime dividing some denominator bⱼ, and let j be maximal such that p
\textbar{} bⱼ.

Consider the coefficient of x\^{}\{j + deg g\} in f(x) = g(x)q(x). Since
g is monic of degree d := deg g, this coefficient is:

\[q_j + \sum_{i > j} g_{d-(i-j)} q_i\]

where we set gₖ = 0 for k \textless{} 0. By maximality of j, each qᵢ
with i \textgreater{} j has denominator coprime to p, and g\_\{d−(i−j)\}
∈ ℤ. Thus the sum ∑\emph{\{i\textgreater j\} g}\{d−(i−j)\} qᵢ has
denominator coprime to p.~But qⱼ has p in its denominator, so the total
cannot be an integer, contradicting f ∈ ℤ{[}x{]}. ∎

\textbf{Theorem 3.2 (Integrality).} For all n ≥ 1, Φₙ(x) ∈ ℤ{[}x{]}.

\emph{Proof.} We proceed by strong induction on n.

\emph{Base case.} Φ₁(x) = x − 1 ∈ ℤ{[}x{]}.

\emph{Inductive step.} Assume Φ\_d(x) ∈ ℤ{[}x{]} for all d \textless{}
n.~Define:

\[D_n(x) := \prod_{\substack{d \mid n \\ d < n}} \Phi_d(x)\]

By the inductive hypothesis, each factor lies in ℤ{[}x{]}, hence Dₙ(x) ∈
ℤ{[}x{]}. Moreover, Dₙ(x) is monic (being a product of monic
polynomials).

By Corollary 2.2:

\[\Phi_n(x) = \frac{x^n - 1}{D_n(x)}\]

Both xⁿ − 1 ∈ ℤ{[}x{]} and Dₙ(x) ∈ ℤ{[}x{]} is monic. By Theorem 2.1,
this division is exact in ℂ{[}x{]}. Since ℚ{[}x{]} ⊆ ℂ{[}x{]},
polynomial division of xⁿ − 1 by Dₙ(x) in ℚ{[}x{]} yields zero
remainder, so the quotient lies in ℚ{[}x{]}. By Lemma 3.1, Φₙ(x) ∈
ℤ{[}x{]}. ∎

\begin{center}\rule{0.5\linewidth}{0.5pt}\end{center}

\hypertarget{the-muxf6bius-inversion-formula}{%
\section{The Möbius Inversion
Formula}\label{the-muxf6bius-inversion-formula}}

The divisor product identity admits an explicit inversion via the Möbius
function.

\textbf{Definition 4.1 (Möbius Function).} The \emph{Möbius function} μ:
ℕ → \{−1, 0, 1\} is defined by:

\[\mu(n) = \begin{cases} 1 & \text{if } n = 1 \\ (-1)^k & \text{if } n = p_1 p_2 \cdots p_k \text{ for distinct primes } p_i \\ 0 & \text{if } p^2 \mid n \text{ for some prime } p \end{cases}\]

The fundamental property is:

\[\sum_{d \mid n} \mu(d) = \begin{cases} 1 & \text{if } n = 1 \\ 0 & \text{if } n > 1 \end{cases}\]

\textbf{Theorem 4.2 (Möbius Inversion for Cyclotomic Polynomials).} For
all n ≥ 1:

\[\Phi_n(x) = \prod_{d \mid n} \left( x^{n/d} - 1 \right)^{\mu(d)} = \prod_{d \mid n} \left( x^d - 1 \right)^{\mu(n/d)}\]

\emph{Proof.} Taking formal logarithms of the divisor product identity
(Theorem 2.1):

\[\log(x^n - 1) = \sum_{d \mid n} \log \Phi_d(x)\]

Define f(n) := log Φₙ(x) and g(n) := log(xⁿ − 1). The identity states
g(n) = ∑\_\{d \textbar{} n\} f(d). By Möbius inversion on the divisor
poset:

\[f(n) = \sum_{d \mid n} \mu(n/d) \, g(d) = \sum_{d \mid n} \mu(n/d) \log(x^d - 1)\]

Exponentiating:

\[\Phi_n(x) = \exp\left( \sum_{d \mid n} \mu(n/d) \log(x^d - 1) \right) = \prod_{d \mid n} (x^d - 1)^{\mu(n/d)}\]

The substitution d ↦ n/d yields the equivalent form ∏\_\{d \textbar{}
n\} (x\^{}\{n/d\} − 1)\^{}\{μ(d)\}.

Although the right-hand side is a priori in ℚ(x), it equals Φₙ(x) by
inversion of Theorem 2.1, hence lies in ℤ{[}x{]} by Theorem 3.2. ∎

\textbf{Remark 4.3 (Validity of the formal argument).} The logarithmic
manipulation is justified in the ring of formal power series
ℚ{[}{[}x⁻¹{]}{]}. Writing xⁿ − 1 = xⁿ(1 − x⁻ⁿ), the expression log(1 −
x⁻ⁿ) = −∑\_\{k≥1\} x⁻ⁿᵏ/k is a well-defined element of ℚ{[}{[}x⁻¹{]}{]}.
The identity holds in this ring, and the resulting polynomial identity
can be verified by observing that both sides are polynomials (by Theorem
3.2) agreeing as formal Laurent series.

\begin{center}\rule{0.5\linewidth}{0.5pt}\end{center}

\hypertarget{explicit-formulae-for-special-cases}{%
\section{Explicit Formulae for Special
Cases}\label{explicit-formulae-for-special-cases}}

\textbf{Proposition 5.1 (Prime Index).} For prime p:

\[\Phi_p(x) = \frac{x^p - 1}{x - 1} = x^{p-1} + x^{p-2} + \cdots + x + 1 = \sum_{k=0}^{p-1} x^k\]

\emph{Proof.} The divisors of p are 1 and p.~Thus xᵖ − 1 = Φ₁(x)Φₚ(x) =
(x−1)Φₚ(x). ∎

\textbf{Proposition 5.2 (Prime Power Index).} For prime p and k ≥ 1:

\[\Phi_{p^k}(x) = \Phi_p(x^{p^{k-1}}) = \sum_{j=0}^{p-1} x^{j \cdot p^{k-1}}\]

In particular, deg Φ\_\{pᵏ\} = φ(pᵏ) = p\^{}\{k−1\}(p−1).

\emph{Proof.} The divisors of pᵏ are 1, p, p², \ldots, pᵏ. Among these,
μ(pʲ) ≠ 0 only for j ∈ \{0, 1\}, with μ(1) = 1 and μ(p) = −1. By Theorem
4.2:

\[\Phi_{p^k}(x) = \frac{(x^{p^k} - 1)^{\mu(1)}}{(x^{p^{k-1}} - 1)^{-\mu(p)}} = \frac{x^{p^k} - 1}{x^{p^{k-1}} - 1}\]

The substitution y = x\textsuperscript{\{p}\{k−1\}\} gives:

\[\Phi_{p^k}(x) = \frac{y^p - 1}{y - 1} = \Phi_p(y) = \Phi_p(x^{p^{k-1}}) \qquad \blacksquare\]

\textbf{Proposition 5.3 (Product of Two Distinct Primes).} For distinct
primes p \textless{} q:

\[\Phi_{pq}(x) = \frac{(x^{pq} - 1)(x - 1)}{(x^p - 1)(x^q - 1)}\]

\emph{Proof.} The divisors of pq are \{1, p, q, pq\} with Möbius values
μ(1) = 1, μ(p) = μ(q) = −1, μ(pq) = 1. The formula follows from Theorem
4.2. ∎

\textbf{Proposition 5.4 (The Case 2p for Odd Prime p).} For odd prime p:

\[\Phi_{2p}(x) = \Phi_p(-x) = x^{p-1} - x^{p-2} + x^{p-3} - \cdots - x + 1 = \sum_{k=0}^{p-1} (-1)^{p-1-k} x^k\]

\emph{Proof.} By Proposition 5.3 with the pair (2, p):

\[\Phi_{2p}(x) = \frac{(x^{2p} - 1)(x - 1)}{(x^2 - 1)(x^p - 1)} = \frac{(x^{2p} - 1)(x - 1)}{(x-1)(x+1)(x^p - 1)} = \frac{x^{2p} - 1}{(x+1)(x^p - 1)}\]

Observe that x\^{}\{2p\} − 1 = (xᵖ)² − 1 = (xᵖ − 1)(xᵖ + 1). Thus:

\[\Phi_{2p}(x) = \frac{x^p + 1}{x + 1}\]

Now Φₚ(−x) = ((−x)ᵖ − 1)/((−x) − 1) = (−xᵖ − 1)/(−x − 1) = (xᵖ + 1)/(x +
1), using that p is odd. ∎

\begin{center}\rule{0.5\linewidth}{0.5pt}\end{center}

\hypertarget{reduction-formulae}{%
\section{Reduction Formulae}\label{reduction-formulae}}

\textbf{Theorem 6.1 (Reduction Formulae).} Let n \textgreater{} 1 and
let p be a prime dividing n.~Write n = pᵃm where gcd(p, m) = 1.

\begin{enumerate}
\def\labelenumi{(\roman{enumi})}
\tightlist
\item
  If a = 1 (so n = pm):
\end{enumerate}

\[\Phi_{pm}(x) = \frac{\Phi_m(x^p)}{\Phi_m(x)}\]

\begin{enumerate}
\def\labelenumi{(\roman{enumi})}
\setcounter{enumi}{1}
\tightlist
\item
  If a ≥ 2 (so p² \textbar{} n):
\end{enumerate}

\[\Phi_n(x) = \Phi_{n/p}(x^p)\]

\emph{Proof of (i).} The divisors of pm partition as \{d : d \textbar{}
m\} ∪ \{pd : d \textbar{} m\}. By Theorem 4.2:

\[\Phi_{pm}(x) = \prod_{d \mid pm} (x^d - 1)^{\mu(pm/d)}\]

Splitting by whether p divides the divisor:

\[= \prod_{d \mid m} (x^d - 1)^{\mu(pm/d)} \cdot \prod_{d \mid m} (x^{pd} - 1)^{\mu(m/d)}\]

Since gcd(p, m) = 1, for d \textbar{} m we have μ(pm/d) = μ(p)μ(m/d) =
−μ(m/d). Thus:

\[\Phi_{pm}(x) = \prod_{d \mid m} (x^d - 1)^{-\mu(m/d)} \cdot \prod_{d \mid m} (x^{pd} - 1)^{\mu(m/d)} = \frac{\Phi_m(x^p)}{\Phi_m(x)}\]

where the final identification uses Theorem 4.2 applied to m. ∎

\emph{Proof of (ii).} Write n = pᵃm with a ≥ 2 and gcd(p, m) = 1. By
Theorem 4.2:

\[\Phi_n(x) = \prod_{d \mid n} (x^d - 1)^{\mu(n/d)}\]

For μ(n/d) ≠ 0, we require n/d to be squarefree. If p ∤ d, then p²
\textbar{} (n/d), so n/d is not squarefree, hence μ(n/d) = 0. Thus only
divisors d with p \textbar{} d contribute.

Writing d = pe for e \textbar{} (n/p), and noting that n/d = n/(pe) =
(n/p)/e:

\[\Phi_n(x) = \prod_{e \mid (n/p)} (x^{pe} - 1)^{\mu((n/p)/e)} = \prod_{e \mid (n/p)} ((x^p)^e - 1)^{\mu((n/p)/e)}\]

The final product is precisely Φ\_\{n/p\}(x\^{}p) by Theorem 4.2 applied
to n/p.~∎

\begin{center}\rule{0.5\linewidth}{0.5pt}\end{center}

\hypertarget{the-general-closed-form}{%
\section{The General Closed Form}\label{the-general-closed-form}}

\textbf{Theorem 7.1 (Canonical Form).} Let n = p₁\^{}\{a₁\} p₂\^{}\{a₂\}
⋯ pᵣ\^{}\{aᵣ\} be the prime factorisation of n \textgreater{} 1. Define
the \emph{radical} rad(n) := p₁p₂⋯pᵣ. Then:

\[\Phi_n(x) = \Phi_{\mathrm{rad}(n)}\!\left( x^{n/\mathrm{rad}(n)} \right)\]

For the squarefree case n = p₁p₂⋯pᵣ:

\[\Phi_n(x) = \prod_{S \subseteq \{1,\ldots,r\}} \left( x^{n/\prod_{i \in S} p_i} - 1 \right)^{(-1)^{|S|}}\]

Equivalently:

\[\Phi_n(x) = \prod_{d \mid n} (x^d - 1)^{\mu(n/d)}\]

\emph{Proof.} For the first identity, apply Theorem 6.1(ii) repeatedly
for each prime p with exponent a ≥ 2 in n, reducing the exponent by one
at each step until all exponents equal one, yielding rad(n).

For the squarefree expansion, note that μ(d) ≠ 0 precisely when d is
squarefree. For n squarefree with r prime factors, the divisors d
\textbar{} n biject with subsets S ⊆ \{1, \ldots, r\} via d = ∏\emph{\{i
∈ S\} pᵢ. Then μ(d) = (−1)\^{}\{\textbar S\textbar\} and n/d = ∏}\{i ∉
S\} pᵢ. ∎

\begin{center}\rule{0.5\linewidth}{0.5pt}\end{center}

\hypertarget{application-explicit-computation-of-ux3c6ux2081ux2082x}{%
\section{Application: Explicit Computation of
Φ₁₂(x)}\label{application-explicit-computation-of-ux3c6ux2081ux2082x}}

Setting n = 12 = 2² · 3, we have rad(12) = 6 and 12/rad(12) = 2.

\textbf{Step 1.} Compute Φ₆(x). Since 6 = 2 · 3, by Proposition 5.3:

\[\Phi_6(x) = \frac{(x^6 - 1)(x - 1)}{(x^2 - 1)(x^3 - 1)}\]

Factoring: x⁶ − 1 = (x³ − 1)(x³ + 1) and x² − 1 = (x−1)(x+1). Thus:

\[\Phi_6(x) = \frac{(x^3 - 1)(x^3 + 1)(x - 1)}{(x-1)(x+1)(x^3 - 1)} = \frac{x^3 + 1}{x + 1} = x^2 - x + 1\]

\textbf{Step 2.} Apply Theorem 7.1:

\[\Phi_{12}(x) = \Phi_6(x^2) = (x^2)^2 - (x^2) + 1 = x^4 - x^2 + 1\]

\textbf{Verification.} The degree is φ(12) = φ(4)φ(3) = 2 · 2 = 4. ✓

The roots are e\^{}\{2πik/12\} for gcd(k, 12) = 1, i.e., k ∈ \{1, 5, 7,
11\}. ✓

\begin{center}\rule{0.5\linewidth}{0.5pt}\end{center}

\hypertarget{table-of-cyclotomic-polynomials}{%
\section{Table of Cyclotomic
Polynomials}\label{table-of-cyclotomic-polynomials}}

\begin{longtable}[]{@{}lll@{}}
\toprule\noalign{}
n & Φₙ(x) & deg Φₙ = φ(n) \\
\midrule\noalign{}
\endhead
\bottomrule\noalign{}
\endlastfoot
1 & x − 1 & 1 \\
2 & x + 1 & 1 \\
3 & x² + x + 1 & 2 \\
4 & x² + 1 & 2 \\
5 & x⁴ + x³ + x² + x + 1 & 4 \\
6 & x² − x + 1 & 2 \\
7 & x⁶ + x⁵ + x⁴ + x³ + x² + x + 1 & 6 \\
8 & x⁴ + 1 & 4 \\
9 & x⁶ + x³ + 1 & 6 \\
10 & x⁴ − x³ + x² − x + 1 & 4 \\
12 & x⁴ − x² + 1 & 4 \\
15 & x⁸ − x⁷ + x⁵ − x⁴ + x³ − x + 1 & 8 \\
\end{longtable}

\begin{center}\rule{0.5\linewidth}{0.5pt}\end{center}

\hypertarget{values-at-special-points}{%
\section{Values at Special Points}\label{values-at-special-points}}

\textbf{Lemma 10.0.} For all odd m \textgreater{} 1:

\[\Phi_{2m}(x) = \Phi_m(-x)\]

\emph{Proof.} By Theorem 6.1(i) with p = 2 and gcd(2, m) = 1:

\[\Phi_{2m}(x) = \frac{\Phi_m(x^2)}{\Phi_m(x)}\]

We show this equals Φₘ(−x). By Theorem 4.2:

\[\frac{\Phi_m(x^2)}{\Phi_m(x)} = \prod_{d \mid m} \frac{(x^{2d} - 1)^{\mu(m/d)}}{(x^d - 1)^{\mu(m/d)}} = \prod_{d \mid m} \left( \frac{x^{2d} - 1}{x^d - 1} \right)^{\mu(m/d)} = \prod_{d \mid m} (x^d + 1)^{\mu(m/d)}\]

For the right-hand side, since m is odd, every divisor d \textbar{} m is
odd, so (−x)\^{}d = −x\^{}d.~Thus:

\[\Phi_m(-x) = \prod_{d \mid m} ((-x)^d - 1)^{\mu(m/d)} = \prod_{d \mid m} (-x^d - 1)^{\mu(m/d)} = \prod_{d \mid m} (-(x^d + 1))^{\mu(m/d)}\]

Since μ(m/d) ∈ \{−1, 0, 1\}, each factor contributes either 1 or −1
accordingly. The total sign is (−1)\^{}\{∑\emph{\{d \textbar{} m\}
μ(m/d)\} = (−1)\^{}0 = 1, since ∑}\{d \textbar{} m\} μ(m/d) = 0 for m
\textgreater{} 1. Hence Φ\_\{2m\}(x) = Φₘ(−x). ∎

\textbf{Proposition 10.1.}

\begin{enumerate}
\def\labelenumi{(\roman{enumi})}
\tightlist
\item
  For n \textgreater{} 1:
\end{enumerate}

\[\Phi_n(1) = \begin{cases} p & \text{if } n = p^k \text{ for some prime } p \\ 1 & \text{otherwise} \end{cases}\]

\begin{enumerate}
\def\labelenumi{(\roman{enumi})}
\setcounter{enumi}{1}
\tightlist
\item
  For n ≥ 1:
\end{enumerate}

\[\Phi_n(-1) = \begin{cases} -2 & \text{if } n = 1 \\ 0 & \text{if } n = 2 \\ 2 & \text{if } n = 2^k \text{ for } k \geq 2 \\ p & \text{if } n = 2p^k \text{ for an odd prime } p, \, k \geq 1 \\ 1 & \text{otherwise} \end{cases}\]

\emph{Proof of (i).} From Theorem 2.1, xⁿ − 1 = ∏\_\{d \textbar{} n\}
Φ\_d(x). Differentiating and evaluating at x = 1:

\[n = \left. \frac{d}{dx}(x^n - 1) \right|_{x=1} = \sum_{d \mid n} \Phi_d'(1) \prod_{\substack{e \mid n \\ e \neq d}} \Phi_e(1)\]

Since Φ₁(1) = 0, only the term d = 1 survives, yielding:

\[n = \Phi_1'(1) \cdot \prod_{\substack{d \mid n \\ d > 1}} \Phi_d(1) = 1 \cdot \prod_{\substack{d \mid n \\ d > 1}} \Phi_d(1)\]

Thus ∏\_\{d \textbar{} n, d \textgreater{} 1\} Φ\_d(1) = n.~We prove the
closed form by strong induction on n.

For n = p prime, the only divisor greater than 1 is p itself, so Φₚ(1) =
p.

For n = pᵏ with k ≥ 2, the divisors greater than 1 are p, p², \ldots,
pᵏ. By induction, Φ\_\{pʲ\}(1) = p for j \textless{} k. Thus:

\[\prod_{j=1}^{k} \Phi_{p^j}(1) = p^k \implies p^{k-1} \cdot \Phi_{p^k}(1) = p^k \implies \Phi_{p^k}(1) = p\]

For n with at least two distinct prime factors, write n = pᵃm with
gcd(p, m) = 1 and m \textgreater{} 1. The divisors of n greater than 1
include all divisors of m greater than 1, all divisors of pᵃ greater
than 1, and mixed divisors. By the multiplicative structure:

\[\prod_{\substack{d \mid n \\ d > 1}} \Phi_d(1) = n = p^a \cdot m\]

The divisors pʲ for 1 ≤ j ≤ a contribute pᵃ (by induction). The divisors
of m greater than 1 contribute m (by induction on m). The remaining
divisors (those involving both p and primes of m) must therefore
contribute 1. Each such Φ\_d(1) is a positive integer: it lies in ℤ by
Theorem 3.2, and Φ\_d(1) \textgreater{} 0 because pairing conjugate
roots gives Φ\_d(1) = ∏ \textbar1 − ζ\textbar² \textgreater{} 0. Since
their product is 1 and each factor is a positive integer, each equals 1.
In particular, Φₙ(1) = 1. ∎

\emph{Proof of (ii).} Write n = 2ᵃm with m odd.

\textbf{Case 1: a = 0 (n odd).} For n = 1, direct computation gives
Φ₁(−1) = −2. For odd n \textgreater{} 1, Lemma 10.0 gives Φ\_\{2n\}(x) =
Φₙ(−x), so Φₙ(−1) = Φ\_\{2n\}(1). Because n \textgreater{} 1 is odd, it
has an odd prime divisor p, so 2n is divisible by both 2 and p.~Hence 2n
is not a prime power, and Φ\_\{2n\}(1) = 1 by part (i).

\textbf{Case 2: a = 1 (n = 2m with m odd).} For m = 1, direct
computation gives Φ₂(−1) = 0. For m \textgreater{} 1, Lemma 10.0 gives
Φ\_\{2m\}(x) = Φₘ(−x), hence Φ\_\{2m\}(−1) = Φₘ(1). By part (i), Φₘ(1) =
p if m = pᵏ for some odd prime p, and Φₘ(1) = 1 otherwise. This yields
Φ\_\{2pᵏ\}(−1) = p for odd primes p, and Φ\_\{2m\}(−1) = 1 for other odd
m \textgreater{} 1.

\textbf{Case 3: a ≥ 2 (4 \textbar{} n).} Apply Theorem 6.1(ii)
repeatedly with p = 2:

\[\Phi_{2^a m}(x) = \Phi_{2m}(x^{2^{a-1}})\]

Evaluating at x = −1:

\[\Phi_{2^a m}(-1) = \Phi_{2m}\left((-1)^{2^{a-1}}\right) = \Phi_{2m}(1)\]

since 2\^{}\{a−1\} ≥ 2 implies (−1)\textsuperscript{\{2}\{a−1\}\} = 1.
If m = 1, this gives Φ\_\{2ᵃ\}(−1) = Φ₂(1) = 2. If m \textgreater{} 1,
then 2m is not a prime power (having both 2 and an odd prime as
factors), so Φ\_\{2m\}(1) = 1 by part (i), yielding Φ\_\{2ᵃm\}(−1) = 1.
∎

\begin{center}\rule{0.5\linewidth}{0.5pt}\end{center}

\hypertarget{generalisations}{%
\section{Generalisations}\label{generalisations}}

\hypertarget{cyclotomic-polynomials-over-finite-fields}{%
\subsection{Cyclotomic Polynomials over Finite
Fields}\label{cyclotomic-polynomials-over-finite-fields}}

For a finite field 𝔽\_q with gcd(q, n) = 1, the polynomial Φₙ(x) ∈
ℤ{[}x{]} reduces to Φ̄ₙ(x) ∈ 𝔽\_q{[}x{]}. This reduced polynomial factors
into irreducible factors of equal degree d, where d is the
multiplicative order of q modulo n.~The number of irreducible factors is
φ(n)/d.

\hypertarget{generalised-cyclotomic-polynomials}{%
\subsection{Generalised Cyclotomic
Polynomials}\label{generalised-cyclotomic-polynomials}}

For integers a, b with gcd(a, b) = 1, define:

\[\Phi_n(a, b) := \prod_{\substack{1 \leq k \leq n \\ \gcd(k,n) = 1}} \left( a - \zeta_n^k b \right)\]

This is a homogeneous polynomial in a, b of degree φ(n) with integer
coefficients. The substitution a = x, b = 1 recovers Φₙ(x).

\begin{center}\rule{0.5\linewidth}{0.5pt}\end{center}

\hypertarget{concluding-remarks}{%
\section{Concluding Remarks}\label{concluding-remarks}}

The cyclotomic polynomial Φₙ(x) admits a closed form via Möbius
inversion:

\[\Phi_n(x) = \prod_{d \mid n} (x^d - 1)^{\mu(n/d)}\]

The integrality Φₙ(x) ∈ ℤ{[}x{]} follows from strong induction using
monic polynomial division. The derivation rests on three principles:

\begin{enumerate}
\def\labelenumi{\arabic{enumi}.}
\item
  \textbf{Divisor partition.} The n-th roots of unity partition by
  primitive order into disjoint sets indexed by d \textbar{} n.
\item
  \textbf{Möbius inversion.} The divisor sum g(n) = ∑\emph{\{d
  \textbar{} n\} f(d) inverts to f(n) = ∑}\{d \textbar{} n\} μ(n/d)
  g(d).
\item
  \textbf{Monic divisibility.} Division by monic polynomials preserves
  integrality in ℤ{[}x{]}.
\end{enumerate}

The reduction Φₙ(x) = Φ\_\{rad(n)\}(x\^{}\{n/rad(n)\}) shows that
computation of Φₙ(x) reduces to the squarefree case, where the Möbius
formula involves 2ʳ terms for r distinct prime factors.

\begin{center}\rule{0.5\linewidth}{0.5pt}\end{center}

\newpage

\hypertarget{appendix-algorithmic-summary}{%
\section{Appendix: Algorithmic
Summary}\label{appendix-algorithmic-summary}}

To compute Φₙ(x):

\begin{enumerate}
\def\labelenumi{\arabic{enumi}.}
\item
  Compute the prime factorisation n = p₁\^{}\{a₁\} ⋯ pᵣ\^{}\{aᵣ\}.
\item
  Compute m := rad(n) = p₁ ⋯ pᵣ and e := n/m = p₁\^{}\{a₁−1\} ⋯
  pᵣ\^{}\{aᵣ−1\}.
\item
  Compute Φₘ(x) via:
\end{enumerate}

\[\Phi_m(x) = \prod_{S \subseteq \{1, \ldots, r\}} \left( x^{m / \prod_{i \in S} p_i} - 1 \right)^{(-1)^{|S|}}\]

\begin{enumerate}
\def\labelenumi{\arabic{enumi}.}
\setcounter{enumi}{3}
\tightlist
\item
  Return Φₙ(x) = Φₘ(xᵉ).
\end{enumerate}

For implementation, the product in step 3 is computed iteratively:
initialise P(x) := 1 (empty product), then for each subset S, multiply
or divide by x\^{}\{m/d\_S\} − 1 according to the parity of
\textbar S\textbar.

\end{document}
