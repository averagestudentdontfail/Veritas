%%% Veritas Pandoc Template %%%
%%% Converts Markdown to LaTeX using Veritas class
%%%
%%% YAML Frontmatter Variables:
%%%   title: Document title
%%%   subtitle: Document subtitle (optional)
%%%   author: Author name or list
%%%   date: Publication date
%%%   lang: en or de
%%%   chapterstyle: minimal, classic, fancy, modern
%%%   abstract: Abstract text (optional)

\documentclass[
    language=en,
    chapterstyle=minimal,
    media=screen,
    doctype=article
]{Veritas}

%%% Metadata from YAML %%%
\Title{Clinical Report: Bill Wynsky}
\Subtitle{Diagnosis, Formulation, and Treatment Processes}
\FirstAuthor{Kiran Nath}
\Date{2025}

%%% Pandoc Compatibility %%%
\providecommand{\tightlist}{%
  \setlength{\itemsep}{0pt}\setlength{\parskip}{0pt}}

% CSL References for Pandoc citeproc
\newlength{\cslhangindent}
\setlength{\cslhangindent}{1.5em}
\newlength{\csllabelwidth}
\setlength{\csllabelwidth}{3em}
\newenvironment{CSLReferences}[2]%
  {\clearpage%
   \section*{References}%
   \addcontentsline{toc}{section}{References}%
   \setlength{\parindent}{0pt}%
   \everypar{\setlength{\hangindent}{\cslhangindent}}\ignorespaces}%
  {\par}
\newcommand{\CSLBlock}[1]{\hfill\break#1\hfill\break}
\newcommand{\CSLLeftMargin}[1]{\parbox[t]{\csllabelwidth}{\strut#1\strut}}
\newcommand{\CSLRightInline}[1]{\parbox[t]{\dimexpr\linewidth-\csllabelwidth\relax}{\strut#1\strut}}
\newcommand{\CSLIndent}[1]{\hspace{\cslhangindent}#1}

\begin{document}

%%% Title Page %%%
\pagenumbering{gobble}
%%% Matter/Title-Page.tex %%%
%%% Unified Title Page for Veritas Template
%%%
%%% FIRST PRINCIPLES: All fields are optional.
%%% Only defined metadata will be displayed.

\newcommand\BackgroundPicTitlePage{%
    \put(0,0){%
    \parbox[b][\paperheight]{\paperwidth}{%
    \vfill
    \centering
    \IfFileExists{Figures/Theme/Front-Page-BG.pdf}{%
        \includegraphics[width=\paperwidth,height=\paperheight,keepaspectratio]{Figures/Theme/Front-Page-BG.pdf}%
    }{}%
    \vfill
}}}
\AddToShipoutPictureBG*{\BackgroundPicTitlePage}

\newgeometry{margin=2.5cm, top=3cm, bottom=2.5cm}
\begin{titlepage}
    \miniondisplayfont
    \color{frontpagedark}
    
    \vspace*{2\baselineskip}
    
    %%% Title (if defined) %%%
    \ifdefined\GetTitle
        \noindent
        \makebox[\textwidth][l]{%
            \parbox{\dimexpr\textwidth-1cm\relax}{%
                \setstretch{1.05}%
                \raggedright\bfseries\fontsize{24}{30}\selectfont\GetTitle
            }%
        }
        \vspace{0.8\baselineskip}
    \fi
    
    %%% Subtitle (if defined) %%%
    \ifdefined\GetSubtitle
        \noindent
        \makebox[\textwidth][l]{%
            \parbox{\dimexpr\textwidth-3cm\relax}{%
                \setstretch{1.03}%
                \raggedright\fontsize{14}{18}\selectfont\itshape\GetSubtitle
            }%
        }
        \vspace{1.5\baselineskip}
    \fi
    
    \vspace{1\baselineskip}
    
    %%% Authors (if defined) %%%
    \ifdefined\GetFirstAuthor
        {\noindent\fontsize{14}{18}\selectfont\GetFirstAuthor}
        
        % Show student number only if explicitly defined
        \ifdefined\GetFirstAuthorNumber
            \ifdefined\ShowStudentNumber
                \\[2pt]
                {\noindent\fontsize{10}{12}\selectfont\itshape\GetFirstAuthorNumber}
            \fi
        \fi
        \vspace{8pt}
    \fi
    
    \ifdefined\GetSecondAuthor
        \\[4pt]
        {\noindent\fontsize{14}{18}\selectfont\GetSecondAuthor}
    \fi
    
    \ifdefined\GetThirdAuthor
        \\[4pt]
        {\noindent\fontsize{14}{18}\selectfont\GetThirdAuthor}
    \fi
    
    %%% Supervisor Section (if any supervisor defined) %%%
    \ifdefined\GetSupervisor
        \vspace{2\baselineskip}
        {
        \noindent
        \fontsize{10}{12}\selectfont
        \renewcommand{\arraystretch}{0.1}
        \hspace*{-2.5pt}\begin{tabular}{@{}r@{\hspace{5pt}}>{\raggedright\arraybackslash}m{6cm}@{}}
            \textbf{Supervisor:} & \GetSupervisor \\ [-.7ex]
            \ifdefined\GetSupervisorTitle
                & \setstretch{0.9}{\fontsize{8}{10}\selectfont\itshape \GetSupervisorTitle} \\ [2ex]
            \fi
            
            \ifdefined\GetCoSupervisor
                \textbf{Co-supervisor:} & \GetCoSupervisor \\ [-.7ex]
                \ifdefined\GetCoSupervisorTitle
                    & \setstretch{0.9}{\fontsize{8}{10}\selectfont\itshape \GetCoSupervisorTitle} \\ [.5ex]
                \fi
            \fi

            \ifdefined\GetSecCoSupervisor        
                & \GetSecCoSupervisor \\ [-.7ex]
                \ifdefined\GetSecCoSupervisorTitle
                    & \setstretch{0.9}{\fontsize{8}{10}\selectfont\itshape \GetSecCoSupervisorTitle} \\
                \fi
            \fi
        \end{tabular}
        }
    \fi
    
    \vfill
    
    %%% Institution Section (if any defined) %%%
    \ifdefined\GetUniversity
        {\noindent\fontsize{10}{12}\selectfont\GetUniversity}\\[2pt]
    \fi
    
    \ifdefined\GetSchool
        {\noindent\fontsize{10}{12}\selectfont\GetSchool}\\[2pt]
    \fi
    
    \ifdefined\GetDepartment
        {\noindent\fontsize{10}{12}\selectfont\GetDepartment}\\[2pt]
    \fi
    
    \ifdefined\GetDegree
        {\noindent\fontsize{10}{12}\selectfont\GetDegree}\\[2pt]
    \fi

    \ifdefined\GetCourse
        {\noindent\fontsize{10}{12}\selectfont\GetCourse}
    \fi

    %%% Document Type (if defined) %%%
    \ifdefined\GetThesisType
        \vspace{1.5\baselineskip}
        {\noindent\fontsize{10}{12}\itshape\selectfont\GetThesisType}
    \fi

    %%% Working Draft Version (only in working stage) %%%
    \ifthenelse{\equal{\DocStageOption}{working}}{%
        \vspace{1.5\baselineskip}
        \ifdefined\GetDocumentVersion
            {\noindent\fontsize{10}{12}\selectfont\textcolor{burgundy}{\GetDocumentVersion}}
        \fi
    }{}

    %%% Date (if defined) %%%
    \ifdefined\GetDate
        \vspace{1.5\baselineskip}
        {\noindent\fontsize{10}{12}\selectfont\GetDate}
    \fi

    \vspace{2\baselineskip}
\end{titlepage}
\restoregeometry
\MediaOptionLogicBlank


%%% Abstract (if present) %%%

%%% Main Content %%%
\pagenumbering{arabic}
\pagestyle{mainmatter}

% Fix section numbering for article mode (remove chapter prefix)
\renewcommand{\thesection}{\arabic{section}}
\renewcommand{\thesubsection}{\thesection.\arabic{subsection}}
\renewcommand{\thesubsubsection}{\thesubsection.\arabic{subsubsection}}

\hypertarget{diagnoses}{%
\section{DIAGNOSES}\label{diagnoses}}

\hypertarget{provisional-diagnoses}{%
\subsection{Provisional Diagnoses}\label{provisional-diagnoses}}

Bill's clinical presentation meets criteria for multiple DSM-5-TR
diagnoses requiring careful differential assessment . Post-Traumatic
Stress Disorder {[}309.81{]} emerges as the primary diagnosis, with
Criterion A unequivocally met through direct combat exposure and
witnessing his closest friend's death via gunshot to the head in
Afghanistan. Bill demonstrates all required symptom clusters with
clinically significant severity: intrusive symptoms manifest through
vivid combat-themed nightmares occurring biweekly, unwanted trauma
memories intruding during waking hours, and physiological reactivity to
trauma cues (touching friend's dog tag); persistent avoidance evidenced
by steadfast refusal discussing military experiences, emotional numbing
during trauma recounting, and avoidance of reminders including military
media; negative cognition and mood alterations encompassing persistent
negative beliefs (``I am weak,'' ``I'm a bad person''), persistent shame
and guilt, markedly diminished interest in previously enjoyed activities
(soccer, socialising), emotional detachment from family members, and
constricted positive affect; marked arousal and reactivity alterations
including irritability culminating in property destruction (punching
walls), hypervigilance, exaggerated startle responses, concentration
impairment affecting daily functioning, and severe sleep disturbance
with new-onset parasomnias. Symptoms persist beyond one month with
profound functional impairment across interpersonal, occupational, and
social domains . Major Depressive Disorder, Moderate {[}296.22{]}
diagnosis appears warranted given persistent depressed mood described as
feeling ``low'' and ``empty,'' anhedonia affecting previously
pleasurable activities, excessive worthlessness and inappropriate guilt
unrelated to trauma, diminished concentration capacity, and recurrent
death thoughts with passive suicidal ideation (visualising throwing
himself under buses), persisting several months representing marked
change from premorbid functioning . Alcohol Use Disorder, Mild
{[}305.00{]} reflects problematic consumption patterns demonstrating
tolerance development (increased quantities: 20--40g weeknights, 100g
weekends versus previous social drinking), unsuccessful control efforts
despite recognition of problems, continued use despite persistent
interpersonal consequences (relationship breakdown), and psychological
dependence using alcohol for stress management, meeting three DSM-5-TR
criteria .

\hypertarget{differential-diagnoses}{%
\subsection{Differential Diagnoses}\label{differential-diagnoses}}

Complex PTSD (ICD-11) warrants primary differential consideration given
comprehensive symptom presentation meeting all diagnostic requirements:
core PTSD symptoms plus self-organisation disturbances encompassing
affect dysregulation (anger outbursts, emotional numbing), negative
self-concept (``bad person,'' persistent shame), and interpersonal
difficulties (detachment, relationship failures), though this diagnosis
exists outside DSM-5-TR nosology . Persistent Complex Bereavement
Disorder remains differential given traumatic loss circumstances, though
broader symptomatology exceeds bereavement-specific criteria .
Adjustment Disorder with Mixed Anxiety and Depressed Mood {[}309.28{]}
appears insufficient given symptom severity, duration exceeding six
months, and specific PTSD criteria fulfilment. REM Sleep Behavior
Disorder necessitates polysomnography evaluation given new-onset
sleepwalking potentially representing dream enactment behaviour
associated with combat nightmares .

\hypertarget{dsm-5-diagnostic-challenges}{%
\subsection{DSM-5 Diagnostic
Challenges}\label{dsm-5-diagnostic-challenges}}

The DSM-5-TR categorical framework presents substantial limitations
capturing Bill's complex phenomenology. Artificial diagnostic boundaries
separate interconnected symptoms; alcohol use, depression, and PTSD
manifestations likely represent integrated affect regulation attempts
rather than discrete, independent disorders, risking fragmented
treatment approaches . The framework inadequately conceptualises
developmental trauma's profound personality organisation impact; Bill's
childhood domestic violence witnessing, perceived maternal protection
failure during psychiatric hospitalisation, and immigration-related
cultural dislocation created foundational vulnerabilities that military
trauma subsequently reactivated, yet this developmental trajectory
receives minimal diagnostic weight within current nosology .
Additionally, DSM-5-TR fails adequately addressing moral injury: Bill's
profound survivor guilt and witnessing civilian abuse without
intervening represents violated deeply-held moral beliefs requiring
different therapeutic approaches than fear-based traumatic stress .

\hypertarget{formulation}{%
\section{FORMULATION}\label{formulation}}

\hypertarget{biopsychosocial-framework}{%
\subsection{Biopsychosocial Framework}\label{biopsychosocial-framework}}

Bill's presentation reflects complex developmental vulnerability and
military trauma interactions within an integrated biopsychosocial
framework requiring multifaceted conceptualisation.

\hypertarget{predisposing-vulnerabilities}{%
\subsubsection{Predisposing
Vulnerabilities}\label{predisposing-vulnerabilities}}

Multiple childhood experiences created intersecting vulnerability
pathways. Immigration from South Africa aged six coincided with parental
conflict escalation during critical attachment formation periods,
disrupting secure base establishment essential for emotional regulation
development . Witnessing father's alcohol-fuelled violence toward mother
established trauma exposure templates, dysregulating neurobiological
stress response systems and creating hypervigilance patterns, emotional
dysregulation vulnerabilities, and maladaptive coping strategies
modelling substance use for distress management . Mother's psychiatric
hospitalisation when Bill was eight represented critical attachment
rupture, with expressed shame regarding protection failure (``couldn't
protect her'') establishing enduring schemas of personal inadequacy,
excessive responsibility, and fundamental helplessness subsequently
reactivated by military experiences. Parental separation at twelve
created additional losses and divided loyalties (``guilty leaving his
father''). School bullying targeting perceived intellectual deficits
(``dumb'') compounded negative self-concept development, with early
educational discontinuation limiting vocational opportunities beyond
military service. These accumulated adversities created what
\textcite{briere2015} conceptualise as ``complex developmental trauma'':
disrupted attachment patterns, impaired self-concept formation, emotion
dysregulation vulnerabilities, and interpersonal difficulties increasing
susceptibility to subsequent traumatic stress.

\hypertarget{precipitating-factors}{%
\subsubsection{Precipitating Factors}\label{precipitating-factors}}

Military service initially provided compensatory experiences addressing
earlier developmental deficits: structure, belonging, identity, and
meaningful peer connections (``real friends for the first time''). This
environment temporarily scaffolded self-organisation capacities
compromised by developmental trauma. However, Afghanistan combat
exposure overwhelmed these compensatory mechanisms through multiple
pathways. Witnessing his closest friend's death activated profound
survivor guilt whilst shattering assumptions about predictability,
control, and fairness. Forced retreat leaving friend's body triggered
abandonment schemas established during childhood maternal protection
failures. Equally significant, witnessing civilian abuse by service
members without intervening created moral injury: deep psychological
wounds resulting from perpetrating, witnessing, or failing preventing
acts violating core moral beliefs, generating shame, self-condemnation,
and existential crisis distinct from fear-based responses . Subsequent
medical discharge following knee injury removed military identity
scaffolding, precipitating psychological decompensation as underlying
vulnerabilities re-emerged without environmental supports.

\hypertarget{perpetuating-mechanisms}{%
\subsubsection{Perpetuating Mechanisms}\label{perpetuating-mechanisms}}

Multiple interconnected factors maintain Bill's difficulties through
self-reinforcing cycles. Cognitive factors include persistent negative
trauma-related cognitions (``I should have saved him,'' ``Good people
die while bad people survive'') creating information processing biases
selectively attending to confirmatory evidence whilst dismissing
disconfirmatory information . Behavioural patterns perpetuate
dysfunction: alcohol provides temporary numbing but prevents processing
whilst creating additional shame (``becoming like father''), social
withdrawal maintains disconnection preventing corrective experiences,
and avoidance prevents habituation. Neurobiological alterations sustain
symptoms through chronically dysregulated stress systems evidenced by
hypervigilance, exaggerated startle, and anger dyscontrol, suggesting
altered amygdala-hippocampal-prefrontal circuitry characteristic of PTSD
. Environmental factors including temporary accommodation, employment
uncertainty, and family separation create ongoing instability
maintaining threat perception.

\hypertarget{protective-factors}{%
\subsubsection{Protective Factors}\label{protective-factors}}

Despite severity, Bill demonstrates important strengths: help-seeking
despite reluctance indicates motivation; family connections provide
potential support; military identity offers belonging; future goals
suggest hope; previous adaptive functioning indicates recovery capacity;
absence of active suicide planning reduces immediate risk. \emph{(See
Appendix A for formulation schematic)}

\hypertarget{intervention}{%
\section{INTERVENTION}\label{intervention}}

\hypertarget{risk-and-safety-considerations}{%
\subsection{Risk and Safety
Considerations}\label{risk-and-safety-considerations}}

Three critical safety domains require immediate stabilisation preceding
trauma-focused intervention. Suicidal ideation management represents
highest priority given passive ideation with specific method
contemplation. Implementation requires collaborative safety planning
using Stanley-Brown Safety Planning Intervention: identifying personal
warning signs (increased isolation, hopelessness), internal coping
strategies (distraction techniques, self-soothing), social distractions,
family/friend crisis contacts, professional resources, and means
restriction including medication security and avoiding high-risk
locations . Weekly Columbia Suicide Severity Rating Scale administration
ensures systematic monitoring with clear escalation protocols . Alcohol
use stabilisation necessitates immediate intervention given escalating
patterns potentially compromising treatment engagement and increasing
impulsivity. Motivational interviewing explores ambivalence, developing
discrepancy between current consumption and valued goals (military
service, future family) whilst supporting self-efficacy for change .
Psychoeducation addresses bidirectional trauma-alcohol relationships,
introducing self-medication concepts whilst highlighting perpetuation of
symptoms. AUDIT-C provides validated monitoring throughout treatment.
Anger and behavioural dysregulation manifesting through property
destruction requires immediate skill development preventing
interpersonal violence escalation. Dialectical Behaviour Therapy
distress tolerance modules offer concrete strategies: TIPP (Temperature
change, Intense exercise, Paced breathing, Paired muscle relaxation) for
acute crises; ACCEPTS (Activities, Contributing, Comparisons, Emotions,
Pushing away, Thoughts, Sensations) for sustained distress without
destructive behaviour .

\hypertarget{social-and-cultural-considerations}{%
\subsection{Social and Cultural
Considerations}\label{social-and-cultural-considerations}}

Bill's treatment requires careful attention to intersecting
sociocultural factors. South African immigration during childhood family
dysfunction suggests acculturation stress affecting identity formation
and belonging . Treatment should explore how cultural dislocation
compounded trauma impacts, potentially incorporating narrative therapy
examining cultural identity stories. Military culture's emphasis on
strength, self-reliance, and stoicism conflicts with vulnerability
required for trauma processing; reframing treatment using
military-consistent language (``operational readiness,'' ``psychological
fitness,'' ``mission planning'') whilst acknowledging service meaning
may enhance engagement . Masculine socialisation creates additional
emotional expression barriers; psychoeducation normalising
neurobiological trauma responses rather than character weakness, using
medical analogies comparing psychological to physical injuries, may
reduce shame-based resistance .

\hypertarget{treatment-interventions}{%
\subsection{Treatment Interventions}\label{treatment-interventions}}

Comprehensive treatment requires coordinated multidisciplinary
intervention. Psychiatric evaluation for evidence-based pharmacotherapy:
SSRIs (sertraline 50--200mg or paroxetine 20--60mg daily) demonstrating
efficacy for military PTSD; prazosin (1--15mg nocte) specifically
targeting trauma nightmares through noradrenergic blockade . Sleep
medicine consultation for comprehensive polysomnography evaluating
parasomnia presentation differentiating PTSD-related disturbance from
primary sleep disorders. Occupational therapy assessing functional
capacity, vocational rehabilitation needs, and return-to-duty fitness .
Three evidence-based therapeutic interventions demonstrate strong
empirical support for military PTSD. Cognitive Processing Therapy (CPT)
directly targets maladaptive cognitions maintaining PTSD through
systematic examination of ``stuck points'' where traumatic experiences
conflict with pre-existing beliefs . Bill's survivor guilt, moral
injury, and negative self-concept represent cognitive maintenance
factors CPT specifically addresses through Socratic dialogue and
cognitive restructuring. Prolonged Exposure (PE) facilitates emotional
processing through systematic confrontation of avoided memories and
situations, demonstrating robust military PTSD efficacy . Bill's marked
avoidance and emotional numbing suggest habituation-based intervention
could reduce symptoms through fear structure modification . Skills
Training in Affective and Interpersonal Regulation/Modified Prolonged
Exposure (STAIR-MPE) provides phased treatment prioritising emotion
regulation before trauma processing, particularly suited for complex
presentations with developmental trauma .

\hypertarget{cognitive-processing-therapy-protocol}{%
\subsection{Cognitive Processing Therapy
Protocol}\label{cognitive-processing-therapy-protocol}}

CPT follows manualised twelve-session protocol adapted for military
populations . Sessions 1--2: Psychoeducation establishing cognitive
model; impact statement exploring trauma's belief effects. Session 3:
ABC worksheets teaching thought-emotion connections; Socratic
questioning challenging initial stuck points. Sessions 4--5: Written
trauma accounts facilitating emotional processing whilst identifying
maintaining cognitions. Sessions 6--7: Challenging Questions Worksheets
systematically examining evidence, alternatives, thinking patterns.
Sessions 8--12: Five thematic modules comprising Safety (threat
assessment), Trust (rebuilding capacity), Power/Control (accepting
limitations, identifying genuine control), Esteem (balanced self-worth),
Intimacy (connection capacity despite trauma). Effectiveness monitoring:
PCL-5 weekly (10-point reduction indicates clinically significant
change); PHQ-9 tracking depression; Posttraumatic Cognitions Inventory
assessing mechanism change .

\hypertarget{reflection}{%
\section{REFLECTION}\label{reflection}}

\hypertarget{strengths-and-limitations}{%
\subsection{Strengths and Limitations}\label{strengths-and-limitations}}

I believe this report comprehensively integrates complex trauma
presentations within evidence-based frameworks, acknowledging
developmental vulnerabilities without deterministic conclusions. I was
selective in the intervention selection, and was careful to balance
empirical support with practical engagement considerations addressing
military culture. However, I do admit that neurobiological factors
including potential traumatic brain injury received insufficient
consideration. Social determinants (housing instability, employment
uncertainty) warranted greater therapeutic planning emphasis. There were
also systemic military institutional factors perpetuating tigma that
deserved deeper critical analysis.

\hypertarget{personal-clinical-challenges}{%
\subsection{Personal Clinical
Challenges}\label{personal-clinical-challenges}}

Bill's moral injury narratives would certaintly activate my personal
distress regarding institutional failures permitting atrocities. I think
maintaining therapeutic neutrality whilst validating legitimate moral
concerns requires careful self-monitoring preventing either dismissing
ethical violations or reinforcing paralysing guilt which i would find
difficult. My cognitive intervention preference might overshadow
necessary emotional processing given Bill's defensive numbing. His
suicidal ideation would trigger personal anxiety potentially leading to
overly cautious risk management compromising therapeutic alliance. I
would certaintly need extensive supervision to explore
countertransference, and would need to rely on my peers to help me
manage the disclosure of such traumas, and of course, personal therapy
to process any vicarious traumatisation.

\end{document}
