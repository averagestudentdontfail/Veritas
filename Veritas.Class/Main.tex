%%% Veritas Pandoc Template %%%
%%% Converts Markdown to LaTeX using Veritas class
%%%
%%% YAML Frontmatter Variables:
%%%   title: Document title
%%%   subtitle: Document subtitle (optional)
%%%   author: Author name or list
%%%   date: Publication date
%%%   lang: en or de
%%%   chapterstyle: minimal, classic, fancy, modern
%%%   abstract: Abstract text (optional)

\documentclass[
    language=en,
    chapterstyle=minimal,
    media=screen,
    doctype=article
]{Veritas}

%%% Metadata from YAML %%%
\Title{The Insolvability of the Quintic}
\Subtitle{A Complete Treatment}
\FirstAuthor{}
\Date{December 2025}

%%% Pandoc Compatibility %%%
\providecommand{\tightlist}{%
  \setlength{\itemsep}{0pt}\setlength{\parskip}{0pt}}

% CSL References for Pandoc citeproc
\newlength{\cslhangindent}
\setlength{\cslhangindent}{1.5em}
\newlength{\csllabelwidth}
\setlength{\csllabelwidth}{3em}
\newenvironment{CSLReferences}[2]%
  {\clearpage%
   \section*{References}%
   \addcontentsline{toc}{section}{References}%
   \setlength{\parindent}{0pt}%
   \everypar{\setlength{\hangindent}{\cslhangindent}}\ignorespaces}%
  {\par}
\newcommand{\CSLBlock}[1]{\hfill\break#1\hfill\break}
\newcommand{\CSLLeftMargin}[1]{\parbox[t]{\csllabelwidth}{\strut#1\strut}}
\newcommand{\CSLRightInline}[1]{\parbox[t]{\dimexpr\linewidth-\csllabelwidth\relax}{\strut#1\strut}}
\newcommand{\CSLIndent}[1]{\hspace{\cslhangindent}#1}

\begin{document}

%%% Title Page %%%
\pagenumbering{gobble}
%%% Matter/Title-Page.tex %%%
%%% Unified Title Page for Veritas Template
%%%
%%% FIRST PRINCIPLES: All fields are optional.
%%% Only defined metadata will be displayed.

\newcommand\BackgroundPicTitlePage{%
    \put(0,0){%
    \parbox[b][\paperheight]{\paperwidth}{%
    \vfill
    \centering
    \IfFileExists{Figures/Theme/Front-Page-BG.pdf}{%
        \includegraphics[width=\paperwidth,height=\paperheight,keepaspectratio]{Figures/Theme/Front-Page-BG.pdf}%
    }{}%
    \vfill
}}}
\AddToShipoutPictureBG*{\BackgroundPicTitlePage}

\newgeometry{margin=2.5cm, top=3cm, bottom=2.5cm}
\begin{titlepage}
    \miniondisplayfont
    \color{frontpagedark}
    
    \vspace*{2\baselineskip}
    
    %%% Title (if defined) %%%
    \ifdefined\GetTitle
        \noindent
        \makebox[\textwidth][l]{%
            \parbox{\dimexpr\textwidth-1cm\relax}{%
                \setstretch{1.05}%
                \raggedright\bfseries\fontsize{24}{30}\selectfont\GetTitle
            }%
        }
        \vspace{0.8\baselineskip}
    \fi
    
    %%% Subtitle (if defined) %%%
    \ifdefined\GetSubtitle
        \noindent
        \makebox[\textwidth][l]{%
            \parbox{\dimexpr\textwidth-3cm\relax}{%
                \setstretch{1.03}%
                \raggedright\fontsize{14}{18}\selectfont\itshape\GetSubtitle
            }%
        }
        \vspace{1.5\baselineskip}
    \fi
    
    \vspace{1\baselineskip}
    
    %%% Authors (if defined) %%%
    \ifdefined\GetFirstAuthor
        {\noindent\fontsize{14}{18}\selectfont\GetFirstAuthor}
        
        % Show student number only if explicitly defined
        \ifdefined\GetFirstAuthorNumber
            \ifdefined\ShowStudentNumber
                \\[2pt]
                {\noindent\fontsize{10}{12}\selectfont\itshape\GetFirstAuthorNumber}
            \fi
        \fi
        \vspace{8pt}
    \fi
    
    \ifdefined\GetSecondAuthor
        \\[4pt]
        {\noindent\fontsize{14}{18}\selectfont\GetSecondAuthor}
    \fi
    
    \ifdefined\GetThirdAuthor
        \\[4pt]
        {\noindent\fontsize{14}{18}\selectfont\GetThirdAuthor}
    \fi
    
    %%% Supervisor Section (if any supervisor defined) %%%
    \ifdefined\GetSupervisor
        \vspace{2\baselineskip}
        {
        \noindent
        \fontsize{10}{12}\selectfont
        \renewcommand{\arraystretch}{0.1}
        \hspace*{-2.5pt}\begin{tabular}{@{}r@{\hspace{5pt}}>{\raggedright\arraybackslash}m{6cm}@{}}
            \textbf{Supervisor:} & \GetSupervisor \\ [-.7ex]
            \ifdefined\GetSupervisorTitle
                & \setstretch{0.9}{\fontsize{8}{10}\selectfont\itshape \GetSupervisorTitle} \\ [2ex]
            \fi
            
            \ifdefined\GetCoSupervisor
                \textbf{Co-supervisor:} & \GetCoSupervisor \\ [-.7ex]
                \ifdefined\GetCoSupervisorTitle
                    & \setstretch{0.9}{\fontsize{8}{10}\selectfont\itshape \GetCoSupervisorTitle} \\ [.5ex]
                \fi
            \fi

            \ifdefined\GetSecCoSupervisor        
                & \GetSecCoSupervisor \\ [-.7ex]
                \ifdefined\GetSecCoSupervisorTitle
                    & \setstretch{0.9}{\fontsize{8}{10}\selectfont\itshape \GetSecCoSupervisorTitle} \\
                \fi
            \fi
        \end{tabular}
        }
    \fi
    
    \vfill
    
    %%% Institution Section (if any defined) %%%
    \ifdefined\GetUniversity
        {\noindent\fontsize{10}{12}\selectfont\GetUniversity}\\[2pt]
    \fi
    
    \ifdefined\GetSchool
        {\noindent\fontsize{10}{12}\selectfont\GetSchool}\\[2pt]
    \fi
    
    \ifdefined\GetDepartment
        {\noindent\fontsize{10}{12}\selectfont\GetDepartment}\\[2pt]
    \fi
    
    \ifdefined\GetDegree
        {\noindent\fontsize{10}{12}\selectfont\GetDegree}\\[2pt]
    \fi

    \ifdefined\GetCourse
        {\noindent\fontsize{10}{12}\selectfont\GetCourse}
    \fi

    %%% Document Type (if defined) %%%
    \ifdefined\GetThesisType
        \vspace{1.5\baselineskip}
        {\noindent\fontsize{10}{12}\itshape\selectfont\GetThesisType}
    \fi

    %%% Working Draft Version (only in working stage) %%%
    \ifthenelse{\equal{\DocStageOption}{working}}{%
        \vspace{1.5\baselineskip}
        \ifdefined\GetDocumentVersion
            {\noindent\fontsize{10}{12}\selectfont\textcolor{burgundy}{\GetDocumentVersion}}
        \fi
    }{}

    %%% Date (if defined) %%%
    \ifdefined\GetDate
        \vspace{1.5\baselineskip}
        {\noindent\fontsize{10}{12}\selectfont\GetDate}
    \fi

    \vspace{2\baselineskip}
\end{titlepage}
\restoregeometry
\MediaOptionLogicBlank


%%% Abstract (if present) %%%

%%% Main Content %%%
\pagenumbering{arabic}
\pagestyle{mainmatter}

% Fix section numbering for article mode (remove chapter prefix)
\renewcommand{\thesection}{\arabic{section}}
\renewcommand{\thesubsection}{\thesection.\arabic{subsection}}
\renewcommand{\thesubsubsection}{\thesubsection.\arabic{subsubsection}}

\hypertarget{preliminaries-groups-and-permutations}{%
\section{Preliminaries: Groups and
Permutations}\label{preliminaries-groups-and-permutations}}

\textbf{Definition 1.1 (Symmetric Group).} For a positive integer \(n\),
the symmetric group \(S_n\) is the group of all bijections from
\(\{1, 2, \ldots, n\}\) to itself, with composition as the group
operation. The order of \(S_n\) is \(n!\).

\textbf{Convention 1.2 (Composition Order).} Throughout this document,
composition of permutations is performed right-to-left: for
\(\sigma, \tau \in S_n\), the product \(\sigma\tau\) means ``first apply
\(\tau\), then apply \(\sigma\)''. That is,
\((\sigma\tau)(x) = \sigma(\tau(x))\).

\textbf{Definition 1.3 (Transposition and Cycle).} A transposition is a
permutation that exchanges exactly two elements and fixes all others. A
\(k\)-cycle \((a_1 \, a_2 \, \cdots \, a_k)\) is the permutation sending
\(a_1 \to a_2 \to \cdots \to a_k \to a_1\) and fixing all other
elements.

\textbf{Definition 1.4 (Even and Odd Permutations).} A permutation
\(\sigma \in S_n\) is even if it can be expressed as a product of an
even number of transpositions, and odd otherwise. The parity is
well-defined.

\textbf{Definition 1.5 (Alternating Group).} The alternating group
\(A_n\) is the subgroup of \(S_n\) consisting of all even permutations.
The order of \(A_n\) is \(n!/2\) for \(n \geq 2\).

\textbf{Definition 1.6 (Support).} The support of a permutation
\(\sigma\) is
\(\operatorname{supp}(\sigma) = \{x : \sigma(x) \neq x\}\).

\textbf{Lemma 1.7 (Generation by 3-Cycles).} For \(n \geq 3\), the
alternating group \(A_n\) is generated by 3-cycles.

\emph{Proof.} Every even permutation is a product of an even number of
transpositions. It suffices to show that any product of two
transpositions is a product of 3-cycles.

\emph{Case 1:} \((a \, b)(a \, b) = e\), the identity.

\emph{Case 2:} \((a \, b)(b \, c)\) with \(a, b, c\) distinct.
Verification that \((a \, b)(b \, c) = (a \, b \, c)\): -
\(a \to (b \, c)\) fixes \(a \to (a \, b)\) sends \(a \to b\). Result:
\(a \to b\). ✓ - \(b \to (b \, c)\) sends \(b \to c \to (a \, b)\) fixes
\(c\). Result: \(b \to c\). ✓ - \(c \to (b \, c)\) sends
\(c \to b \to (a \, b)\) sends \(b \to a\). Result: \(c \to a\). ✓

\emph{Case 3:} \((a \, b)(c \, d)\) with \(a, b, c, d\) distinct. We
claim \((a \, b)(c \, d) = (a \, c \, b)(a \, c \, d)\). Verification: -
\(a \to (a \, c \, d)\) sends \(a \to c \to (a \, c \, b)\) sends
\(c \to b\). Result: \(a \to b\). ✓ - \(b \to (a \, c \, d)\) fixes
\(b \to (a \, c \, b)\) sends \(b \to a\). Result: \(b \to a\). ✓ -
\(c \to (a \, c \, d)\) sends \(c \to d \to (a \, c \, b)\) fixes \(d\).
Result: \(c \to d\). ✓ - \(d \to (a \, c \, d)\) sends
\(d \to a \to (a \, c \, b)\) sends \(a \to c\). Result: \(d \to c\). ✓
\(\blacksquare\)

\begin{center}\rule{0.5\linewidth}{0.5pt}\end{center}

\hypertarget{normal-subgroups-and-quotients}{%
\section{Normal Subgroups and
Quotients}\label{normal-subgroups-and-quotients}}

\textbf{Definition 2.1 (Normal Subgroup).} A subgroup \(N\) of \(G\) is
normal, written \(N \trianglelefteq G\), if \(gNg^{-1} = N\) for all
\(g \in G\).

\textbf{Definition 2.2 (Quotient Group).} For \(N \trianglelefteq G\),
the quotient \(G/N\) is the set of cosets \(\{gN : g \in G\}\) with
multiplication \((gN)(hN) := (gh)N\).

\textbf{Definition 2.3 (Simple Group).} A group \(G\) is simple if
\(|G| > 1\) and the only normal subgroups are \(\{e\}\) and \(G\).

\textbf{Lemma 2.4 (Conjugacy of 3-Cycles).} In \(S_n\) for \(n \geq 3\),
any two 3-cycles are conjugate.

\emph{Proof.} Given 3-cycles \((a \, b \, c)\) and \((d \, e \, f)\),
choose \(\sigma \in S_n\) with \(\sigma(a) = d\), \(\sigma(b) = e\),
\(\sigma(c) = f\). Then
\(\sigma(a \, b \, c)\sigma^{-1} = (d \, e \, f)\) by direct
verification on each element. \(\blacksquare\)

\textbf{Lemma 2.5 (Commutator in Normal Subgroups).} If
\(N \trianglelefteq G\) and \(\sigma \in N\), then
\([\sigma, \tau] := \sigma\tau\sigma^{-1}\tau^{-1} \in N\) for all
\(\tau \in G\).

\emph{Proof.} We have
\([\sigma, \tau] = \sigma(\tau\sigma^{-1}\tau^{-1})\). Since \(N\) is
normal, \(\tau\sigma^{-1}\tau^{-1} \in N\), so \([\sigma, \tau] \in N\).
\(\blacksquare\)

\begin{center}\rule{0.5\linewidth}{0.5pt}\end{center}

\hypertarget{simplicity-of-the-alternating-group}{%
\section{Simplicity of the Alternating
Group}\label{simplicity-of-the-alternating-group}}

\textbf{Lemma 3.1 (Conjugation of Cycles).} For any permutation
\(\sigma\) and cycle \((x \, y \, z)\), we have
\(\sigma(x \, y \, z)\sigma^{-1} = (\sigma(x) \, \sigma(y) \, \sigma(z))\).

\emph{Proof.} For any element \(w\): if \(w = \sigma(x)\), then
\(\sigma(x \, y \, z)\sigma^{-1}(w) = \sigma(x \, y \, z)(x) = \sigma(y)\).
Similarly for \(\sigma(y)\) and \(\sigma(z)\). If
\(w \notin \{\sigma(x), \sigma(y), \sigma(z)\}\), then
\(\sigma^{-1}(w) \notin \{x, y, z\}\), so \((x \, y \, z)\) fixes
\(\sigma^{-1}(w)\), and \(\sigma(x \, y \, z)\sigma^{-1}(w) = w\).
\(\blacksquare\)

\textbf{Lemma 3.2 (Reduction from Long Cycles).} Let \(\sigma \in A_n\)
contain a cycle of length \(r \geq 4\), with \(\sigma(a_1) = a_2\),
\(\sigma(a_2) = a_3\), \(\sigma(a_3) = a_4\). Let
\(\tau = (a_1 \, a_2 \, a_3)\). Then
\([\sigma, \tau] = (a_1 \, a_4 \, a_2)\), a 3-cycle.

\emph{Proof.} By Lemma 3.1,
\(\sigma\tau\sigma^{-1} = (\sigma(a_1) \, \sigma(a_2) \, \sigma(a_3)) = (a_2 \, a_3 \, a_4)\).

Computing \([\sigma, \tau] = (a_2 \, a_3 \, a_4)(a_1 \, a_3 \, a_2)\)
where \(\tau^{-1} = (a_1 \, a_3 \, a_2)\): - \(a_1\):
\((a_1 \, a_3 \, a_2)\) sends \(a_1 \to a_3\); \((a_2 \, a_3 \, a_4)\)
sends \(a_3 \to a_4\). Result: \(a_1 \to a_4\). - \(a_2\):
\((a_1 \, a_3 \, a_2)\) sends \(a_2 \to a_1\); \((a_2 \, a_3 \, a_4)\)
fixes \(a_1\). Result: \(a_2 \to a_1\). - \(a_3\):
\((a_1 \, a_3 \, a_2)\) sends \(a_3 \to a_2\); \((a_2 \, a_3 \, a_4)\)
sends \(a_2 \to a_3\). Result: \(a_3 \to a_3\) (fixed). - \(a_4\):
\((a_1 \, a_3 \, a_2)\) fixes \(a_4\); \((a_2 \, a_3 \, a_4)\) sends
\(a_4 \to a_2\). Result: \(a_4 \to a_2\).

Hence \([\sigma, \tau] = (a_1 \, a_4 \, a_2)\), a 3-cycle with
\(\operatorname{supp}([\sigma, \tau]) = \{a_1, a_2, a_4\}\).
\(\blacksquare\)

\textbf{Lemma 3.3 (Reduction from Multiple Transpositions to Double
Transposition).} Let \(\sigma = (a \, b)(c \, d)\sigma'\) where
\(\sigma'\) is a product of transpositions disjoint from
\(\{a, b, c, d\}\). Let \(\tau = (a \, b \, c)\). Then
\([\sigma, \tau] = (a \, c)(b \, d)\).

\emph{Proof.} Since \(\sigma'\) is disjoint from \(\{a, b, c\}\), it
commutes with \(\tau = (a \, b \, c)\). Hence

\[[\sigma, \tau] = [(a \, b)(c \, d)\sigma', \tau] = [(a \, b)(c \, d), \tau].\]

(We do \textbf{not} cancel \((c \, d)\), since it does not commute with
\(\tau\).)

We now compute \([(a \, b)(c \, d), (a \, b \, c)]\) directly. For
\(\sigma_0 = (a \, b)(c \, d)\) and \(\tau = (a \, b \, c)\), note that
\(\sigma_0^{-1} = \sigma_0\) (since \(\sigma_0\) is a product of
disjoint transpositions). We find \(\sigma_0\tau\sigma_0^{-1}\) by
tracking each element: - \(a\): \(\sigma_0(a) = b\), \(\tau(b) = c\),
\(\sigma_0^{-1}(c) = d\). So \(\sigma_0\tau\sigma_0^{-1}(a) = d\). -
\(b\): \(\sigma_0(b) = a\), \(\tau(a) = b\), \(\sigma_0^{-1}(b) = a\).
So \(\sigma_0\tau\sigma_0^{-1}(b) = a\). - \(c\): \(\sigma_0(c) = d\),
\(\tau(d) = d\), \(\sigma_0^{-1}(d) = c\). So
\(\sigma_0\tau\sigma_0^{-1}(c) = c\). - \(d\): \(\sigma_0(d) = c\),
\(\tau(c) = a\), \(\sigma_0^{-1}(a) = b\). So
\(\sigma_0\tau\sigma_0^{-1}(d) = b\).

So \(\sigma_0\tau\sigma_0^{-1}\): \(a \to d\), \(b \to a\), \(c \to c\),
\(d \to b\). This is \((a \, d \, b)\).

Now \([\sigma_0, \tau] = (a \, d \, b)(a \, c \, b)\) where
\(\tau^{-1} = (a \, c \, b)\): - \(a\): \((a \, c \, b)(a) = c\),
\((a \, d \, b)(c) = c\). Result: \(a \to c\). - \(b\):
\((a \, c \, b)(b) = a\), \((a \, d \, b)(a) = d\). Result: \(b \to d\).
- \(c\): \((a \, c \, b)(c) = b\), \((a \, d \, b)(b) = a\). Result:
\(c \to a\). - \(d\): \((a \, c \, b)(d) = d\),
\((a \, d \, b)(d) = b\). Result: \(d \to b\).

Hence \([\sigma, \tau] = (a \, c)(b \, d)\), a double transposition.
\(\blacksquare\)

\textbf{Lemma 3.4 (From Double Transposition to 3-Cycle).} Let
\(n \geq 5\) and let \(\delta = (a \, c)(b \, d)\). Choose
\(e \in \{1,\ldots,n\} \setminus \{a,b,c,d\}\) and let
\(\rho = (a \, c \, e)\). Then \([\delta, \rho] = (a \, c \, e)\), a
3-cycle.

\emph{Proof.} We compute \(\delta\rho\delta^{-1}\) directly. Since
\(\delta = \delta^{-1}\), we compute \(\delta\rho\delta\) by tracking
each element: - \(a\): \(\delta(a) = c\), \(\rho(c) = e\),
\(\delta(e) = e\). So \(\delta\rho\delta(a) = e\). - \(c\):
\(\delta(c) = a\), \(\rho(a) = c\), \(\delta(c) = a\). So
\(\delta\rho\delta(c) = a\). - \(e\): \(\delta(e) = e\),
\(\rho(e) = a\), \(\delta(a) = c\). So \(\delta\rho\delta(e) = c\). -
\(b\): \(\delta(b) = d\), \(\rho(d) = d\), \(\delta(d) = b\). So
\(\delta\rho\delta(b) = b\). - \(d\): \(\delta(d) = b\),
\(\rho(b) = b\), \(\delta(b) = d\). So \(\delta\rho\delta(d) = d\).

Hence \(\delta\rho\delta^{-1} = (a \, e \, c)\).

Now we compute
\([\delta, \rho] = (\delta\rho\delta^{-1})\rho^{-1} = (a \, e \, c)(a \, e \, c)\),
since \(\rho^{-1} = (a \, e \, c)\): - \(a\): \((a \, e \, c)(a) = e\),
\((a \, e \, c)(e) = c\). Result: \(a \to c\). - \(c\):
\((a \, e \, c)(c) = a\), \((a \, e \, c)(a) = e\). Result: \(c \to e\).
- \(e\): \((a \, e \, c)(e) = c\), \((a \, e \, c)(c) = a\). Result:
\(e \to a\).

Hence \([\delta, \rho] = (a \, e \, c)^2 = (a \, c \, e)\), a 3-cycle.
\(\blacksquare\)

\textbf{Lemma 3.5 (Two Disjoint 3-Cycles to 5-Cycle).} Let
\(\sigma = (a \, b \, c)(d \, e \, f)\sigma'\) where \(\sigma'\) is
disjoint from \(\{a,b,c,d,e,f\}\). Let \(\tau = (a \, b \, d)\). Then
\([\sigma, \tau] = (a \, d \, c \, e \, b)\), a 5-cycle.

\emph{Proof.} Since \(\sigma'\) is disjoint from \(\{a, b, d\}\), it
commutes with \(\tau\) and cancels in the commutator.

By Lemma 3.1,
\(\sigma\tau\sigma^{-1} = (\sigma(a) \, \sigma(b) \, \sigma(d)) = (b \, c \, e)\).

Computing \([\sigma, \tau] = (b \, c \, e)(a \, d \, b)\) where
\(\tau^{-1} = (a \, d \, b)\): - \(a\): \((a \, d \, b)(a) = d\),
\((b \, c \, e)(d) = d\). Result: \(a \to d\). - \(b\):
\((a \, d \, b)(b) = a\), \((b \, c \, e)(a) = a\). Result: \(b \to a\).
- \(c\): \((a \, d \, b)(c) = c\), \((b \, c \, e)(c) = e\). Result:
\(c \to e\). - \(d\): \((a \, d \, b)(d) = b\),
\((b \, c \, e)(b) = c\). Result: \(d \to c\). - \(e\):
\((a \, d \, b)(e) = e\), \((b \, c \, e)(e) = b\). Result: \(e \to b\).

Hence \([\sigma, \tau] = (a \, d \, c \, e \, b)\), a 5-cycle.
\(\blacksquare\)

\textbf{Theorem 3.6 (Simplicity of \(A_n\)).} For \(n \geq 5\), the
alternating group \(A_n\) is simple.

\emph{Proof.} Let \(N \trianglelefteq A_n\) with \(N \neq \{e\}\). We
show \(N = A_n\).

\textbf{Step 1: \(N\) contains a 3-cycle.}

Choose \(\sigma \in N \setminus \{e\}\) with
\(|\operatorname{supp}(\sigma)|\) minimal. We show
\(|\operatorname{supp}(\sigma)| = 3\) by contradiction.

Suppose \(|\operatorname{supp}(\sigma)| > 3\). We derive a contradiction
by producing a non-identity element in \(N\) with strictly smaller
support.

\emph{Case A: \(\sigma\) contains a cycle of length \(\geq 4\).}

By Lemma 3.2, \([\sigma, \tau]\) is a 3-cycle in \(N\). Since a 3-cycle
is non-identity with
\(|\operatorname{supp}| = 3 < |\operatorname{supp}(\sigma)|\), this
contradicts minimality.

\emph{Case B: \(\sigma\) is a product of disjoint 3-cycles, with at
least two.}

Then \(|\operatorname{supp}(\sigma)| \geq 6\). By Lemma 3.5,
\([\sigma, \tau]\) is a 5-cycle in \(N\). Since a 5-cycle has length
\(\geq 4\), Lemma 3.2 applies and yields a 3-cycle in \(N\) with
\(|\operatorname{supp}| = 3 < 6 \leq |\operatorname{supp}(\sigma)|\),
contradicting minimality.

\emph{Case C: \(\sigma\) is a product of disjoint transpositions.}

Since \(\sigma \in A_n\), \(\sigma\) has at least 2 transpositions, so
\(|\operatorname{supp}(\sigma)| \geq 4\).

\emph{Subcase C1:} \(|\operatorname{supp}(\sigma)| > 4\) (at least 3
transpositions).

By Lemma 3.3, \([\sigma, \tau] = (a \, c)(b \, d) \in N\). Since
\((a \, c)(b \, d) \neq e\) (it moves four elements), this is a
non-identity element of \(N\) with
\(|\operatorname{supp}| = 4 < |\operatorname{supp}(\sigma)|\),
contradicting minimality.

\emph{Subcase C2:} \(|\operatorname{supp}(\sigma)| = 4\) (exactly a
double transposition).

Write \(\sigma = (a \, c)(b \, d)\). By Lemma 3.4,
\([\sigma, \rho] = (a \, c \, e) \in N\). Since \((a \, c \, e) \neq e\)
(it is a 3-cycle), this is a non-identity element of \(N\) with
\(|\operatorname{supp}| = 3 < 4\), contradicting minimality.

\emph{Exhaustiveness:} In the disjoint cycle decomposition of
\(\sigma\), if any cycle has length \(\geq 4\), we are in Case A.
Otherwise all nontrivial cycles have length 2 or 3. If there are at
least two 3-cycles, we are in Case B. If there are no 3-cycles, we are
in Case C.

Since all cases yield contradictions,
\(|\operatorname{supp}(\sigma)| \leq 3\). Since \(\sigma \neq e\) and
\(\sigma \in A_n\), we have \(|\operatorname{supp}(\sigma)| \geq 2\). A
non-identity even permutation with \(|\operatorname{supp}(\sigma)| = 2\)
would be a single transposition, which is odd---contradiction. Thus
\(|\operatorname{supp}(\sigma)| = 3\).

An even permutation with support exactly 3 must be a 3-cycle. Therefore
\(\sigma\) is a 3-cycle, and \(N\) contains a 3-cycle.

\textbf{Step 2: \(N\) contains all 3-cycles.}

Let \((a \, b \, c) \in N\). For any 3-cycle \((d \, e \, f)\), by Lemma
2.4 there exists \(\sigma \in S_n\) with
\(\sigma(a \, b \, c)\sigma^{-1} = (d \, e \, f)\).

If \(\sigma \in A_n\), then \((d \, e \, f) \in N\) by normality.

If \(\sigma \notin A_n\), choose distinct
\(p, q \in \{1,\ldots,n\} \setminus \{a,b,c\}\). Such \(p, q\) exist
since \(n \geq 5\). Let \(\rho = (p \, q)\). Then \(\sigma\rho \in A_n\)
(product of two odd permutations). Since \(\rho\) is disjoint from
\((a \, b \, c)\), we have
\(\rho(a \, b \, c)\rho^{-1} = (a \, b \, c)\). Thus:

\[(\sigma\rho)(a \, b \, c)(\sigma\rho)^{-1} = \sigma\rho(a \, b \, c)\rho^{-1}\sigma^{-1} = \sigma(a \, b \, c)\sigma^{-1} = (d \, e \, f) \in N.\]

By Lemma 1.7, \(N = A_n\). \(\blacksquare\)

\begin{center}\rule{0.5\linewidth}{0.5pt}\end{center}

\hypertarget{solvable-groups}{%
\section{Solvable Groups}\label{solvable-groups}}

\textbf{Definition 4.1 (Solvable Group).} A group \(G\) is solvable if
there exists a chain
\(\{e\} = G_0 \trianglelefteq G_1 \trianglelefteq \cdots \trianglelefteq G_k = G\)
with each \(G_{i+1}/G_i\) abelian.

\textbf{Lemma 4.2 (Non-Abelian Simple \(\Rightarrow\) Not Solvable).} A
non-abelian simple group is not solvable.

\emph{Proof.} Let \(G\) be simple and non-abelian. In any solvable
series \(\{e\} = G_0 \trianglelefteq \cdots \trianglelefteq G_k = G\),
simplicity forces \(G_{k-1} \in \{\{e\}, G\}\). If \(G_{k-1} = G\), the
term is redundant. Removing redundancies, we reach
\(\{e\} \trianglelefteq G\), requiring \(G/\{e\} \cong G\) to be
abelian, contradicting non-abelianity. \(\blacksquare\)

\textbf{Lemma 4.3 (Subgroups of Solvable Groups).} If \(G\) is solvable
and \(H \leq G\), then \(H\) is solvable.

\emph{Proof.} Given a solvable series
\(\{e\} = G_0 \trianglelefteq \cdots \trianglelefteq G_k = G\), define
\(H_i = G_i \cap H\). Then \(H_i \trianglelefteq H_{i+1}\), and the map
\(H_{i+1} \to G_{i+1}/G_i\) given by \(h \mapsto hG_i\) has kernel
\(H_i\). By the first isomorphism theorem, \(H_{i+1}/H_i\) embeds into
the abelian group \(G_{i+1}/G_i\), hence is abelian. \(\blacksquare\)

\textbf{Theorem 4.4 (\(S_n\) Not Solvable for \(n \geq 5\)).} For
\(n \geq 5\), \(S_n\) is not solvable.

\emph{Proof.} By Theorem 3.6, \(A_n\) is simple. It is non-abelian:
\((1 \, 2 \, 3)(1 \, 2 \, 4) = (1 \, 3)(2 \, 4) \neq (1 \, 4)(2 \, 3) = (1 \, 2 \, 4)(1 \, 2 \, 3)\).
By Lemma 4.2, \(A_n\) is not solvable. Since \(A_n \leq S_n\), Lemma 4.3
implies \(S_n\) is not solvable. \(\blacksquare\)

\begin{center}\rule{0.5\linewidth}{0.5pt}\end{center}

\hypertarget{field-extensions}{%
\section{Field Extensions}\label{field-extensions}}

\textbf{Definition 5.1 (Field Extension).} A field extension \(L/K\) is
an inclusion \(K \subseteq L\). The degree \([L : K]\) is \(\dim_K(L)\).

\textbf{Definition 5.2 (Algebraic Element).} \(\alpha \in L\) is
algebraic over \(K\) if it satisfies some non-zero \(f(x) \in K[x]\).
The minimal polynomial is the unique monic irreducible polynomial in
\(K[x]\) with \(\alpha\) as a root.

\textbf{Definition 5.3 (Splitting Field).} A splitting field of
\(f(x) \in K[x]\) over \(K\) is an extension \(L\) where \(f\) factors
completely as \(f(x) = c\prod(x - \alpha_i)\) and
\(L = K(\alpha_1,\ldots,\alpha_n)\).

\textbf{Lemma 5.4 (Tower Law).} For \(K \subseteq M \subseteq L\) with
finite degrees, \([L : K] = [L : M][M : K]\).

\emph{Proof.} If \(\{u_1,\ldots,u_m\}\) is a \(K\)-basis for \(M\) and
\(\{v_1,\ldots,v_n\}\) is an \(M\)-basis for \(L\), then \(\{u_i v_j\}\)
is a \(K\)-basis for \(L\). \(\blacksquare\)

\begin{center}\rule{0.5\linewidth}{0.5pt}\end{center}

\hypertarget{galois-theory}{%
\section{Galois Theory}\label{galois-theory}}

\textbf{Definition 6.1 (Galois Group).} For \(L/K\), the Galois group
\(\operatorname{Gal}(L/K)\) is the group of \(K\)-automorphisms of
\(L\).

\textbf{Lemma 6.2 (Automorphisms Permute Roots).} Let \(f(x) \in K[x]\)
have roots \(\alpha_1,\ldots,\alpha_n\) in \(L\). Every
\(\sigma \in \operatorname{Gal}(L/K)\) permutes
\(\{\alpha_1,\ldots,\alpha_n\}\).

\emph{Proof.} For \(f(x) = \sum a_i x^i\) with \(a_i \in K\):
\(f(\sigma(\alpha)) = \sum a_i \sigma(\alpha)^i = \sum \sigma(a_i)\sigma(\alpha)^i = \sigma(\sum a_i \alpha^i) = \sigma(0) = 0\).
\(\blacksquare\)

\textbf{Definition 6.3 (Separable Polynomial).} A polynomial
\(f \in K[x]\) is separable if it has no repeated roots in any extension
of \(K\). In characteristic 0, every irreducible polynomial is
separable.

\textbf{Theorem 6.4 (Characterisation of Galois Extensions).} A finite
extension \(L/K\) is Galois (meaning
\(|\operatorname{Gal}(L/K)| = [L : K]\)) if and only if \(L\) is the
splitting field of a separable polynomial over \(K\).

\textbf{Theorem 6.5 (Galois Correspondence).} For a finite Galois
extension \(L/K\) with \(G = \operatorname{Gal}(L/K)\), there is an
inclusion-reversing bijection between intermediate fields
\(K \subseteq M \subseteq L\) and subgroups \(H \leq G\). Moreover,
\(M/K\) is Galois iff \(\operatorname{Gal}(L/M) \trianglelefteq G\), in
which case \(\operatorname{Gal}(M/K) \cong G/\operatorname{Gal}(L/M)\).

\begin{center}\rule{0.5\linewidth}{0.5pt}\end{center}

\hypertarget{symmetric-functions-and-the-generic-polynomial}{%
\section{Symmetric Functions and the Generic
Polynomial}\label{symmetric-functions-and-the-generic-polynomial}}

\textbf{Definition 7.1 (Elementary Symmetric Polynomials).} For
indeterminates \(x_1,\ldots,x_n\), define \(e_1 = \sum x_i\),
\(e_2 = \sum_{i<j} x_i x_j\), \(\ldots\), \(e_n = x_1 \cdots x_n\), so
that \(\prod(t - x_i) = t^n - e_1 t^{n-1} + \cdots + (-1)^n e_n\).

\textbf{Theorem 7.2 (Fundamental Theorem of Symmetric Functions).} Let
\(F = k(x_1,\ldots,x_n)\) with \(S_n\) acting by permuting variables.
Then \(F^{S_n} = k(e_1,\ldots,e_n)\).

\emph{Proof.} Let \(E = k(e_1,\ldots,e_n)\). Clearly
\(E \subseteq F^{S_n}\).

For the reverse: let \(g \in k[x_1,\ldots,x_n]\) be symmetric. Among all
monomials of \(g\), choose one with maximal exponent sequence
\((a_1,\ldots,a_n)\) in lexicographic order. Since \(g\) is symmetric
and this monomial is maximal, we have
\(a_1 \geq a_2 \geq \cdots \geq a_n\) (otherwise permuting would yield a
larger monomial).

The polynomial \(e_1^{a_1-a_2} e_2^{a_2-a_3} \cdots e_n^{a_n}\) has
leading monomial \(x_1^{a_1} \cdots x_n^{a_n}\). Subtracting an
appropriate scalar multiple from \(g\) reduces the maximal monomial. By
induction on the well-ordered set of monomial sequences,
\(g \in k[e_1,\ldots,e_n]\).

For \(f/g \in F^{S_n}\) with \(f,g \in k[x_1,\ldots,x_n]\) and
\(g \neq 0\): the product \(\prod_{\sigma \in S_n}(\sigma \cdot g)\) is
symmetric (applying any \(\tau \in S_n\) permutes the factors), hence
lies in \(k[e_1,\ldots,e_n]\). The numerator
\(f \cdot \prod_{\sigma \neq e}(\sigma \cdot g)\) equals
\((f/g) \cdot \prod_{\sigma \in S_n}(\sigma \cdot g)\), which is the
product of the \(S_n\)-invariant element \(f/g\) with a symmetric
polynomial, hence symmetric, hence in \(k[e_1,\ldots,e_n]\). Thus
\(f/g \in k(e_1,\ldots,e_n)\). \(\blacksquare\)

\textbf{Theorem 7.3 (Galois Group of the Generic Polynomial).} Let \(k\)
have characteristic 0, \(E = k(e_1,\ldots,e_n)\),
\(F = k(x_1,\ldots,x_n)\). Then:

\begin{enumerate}
\def\labelenumi{(\roman{enumi})}
\item
  \(F\) is the splitting field of
  \(P(t) = t^n - e_1 t^{n-1} + \cdots + (-1)^n e_n\) over \(E\).
\item
  \(F/E\) is Galois with \(\operatorname{Gal}(F/E) \cong S_n\).
\item
  \([F : E] = n!\).
\end{enumerate}

\emph{Proof.}

\begin{enumerate}
\def\labelenumi{(\roman{enumi})}
\item
  By definition, \(P(t) = \prod(t - x_i)\), so the roots are
  \(x_1,\ldots,x_n \in F\), and \(F = E(x_1,\ldots,x_n)\).
\item
  The roots \(x_1,\ldots,x_n\) are distinct elements of \(F\) (they are
  independent indeterminates). Over characteristic 0,
  \(P(t) = \prod(t - x_i)\) is therefore separable. Thus \(F/E\) is
  Galois by Theorem 6.4.
\end{enumerate}

Each \(\sigma \in S_n\) induces an \(E\)-automorphism \(\varphi_\sigma\)
of \(F\) by
\(\varphi_\sigma(f(x_1,\ldots,x_n)) = f(x_{\sigma(1)},\ldots,x_{\sigma(n)})\).
This defines a homomorphism
\(\varphi: S_n \to \operatorname{Gal}(F/E)\).

\emph{Injectivity:} If \(\sigma \neq \tau\), then
\(\sigma(i) \neq \tau(i)\) for some \(i\), so
\(\varphi_\sigma(x_i) \neq \varphi_\tau(x_i)\).

\emph{Surjectivity:} Any \(\psi \in \operatorname{Gal}(F/E)\) permutes
\(\{x_1,\ldots,x_n\}\) by Lemma 6.2. If \(\psi(x_i) = x_{\sigma(i)}\),
then \(\psi = \varphi_\sigma\) since \(F = E(x_1,\ldots,x_n)\).

\begin{enumerate}
\def\labelenumi{(\roman{enumi})}
\setcounter{enumi}{2}
\tightlist
\item
  Since \(F/E\) is Galois,
  \([F : E] = |\operatorname{Gal}(F/E)| = |S_n| = n!\). \(\blacksquare\)
\end{enumerate}

\begin{center}\rule{0.5\linewidth}{0.5pt}\end{center}

\hypertarget{radical-extensions}{%
\section{Radical Extensions}\label{radical-extensions}}

\textbf{Definition 8.1 (Radical Extension).} \(L/K\) is a radical
extension if there exists
\(K = K_0 \subseteq K_1 \subseteq \cdots \subseteq K_r = L\) with
\(K_{i+1} = K_i(\alpha_i)\) and \(\alpha_i^{n_i} \in K_i\).

\textbf{Definition 8.2 (Solvable by Radicals).} \(f(x) \in K[x]\) is
solvable by radicals if some radical extension of \(K\) contains all
roots of \(f\).

\textbf{Lemma 8.3 (Radical Expressions Lie in Radical Extensions).} Any
expression built from elements of \(K\) using \(+, -, \times, \div\) and
extraction of \(n\)-th roots lies in some radical extension of \(K\).

\emph{Proof.} Such an expression is constructed in finitely many steps.
Each arithmetic operation stays within the current field. Each \(n\)-th
root extraction \(K_i(\alpha)\) with \(\alpha^n \in K_i\) is a simple
radical extension. The composition of finitely many simple radical
extensions is a radical extension. \(\blacksquare\)

\textbf{Lemma 8.4 (Cyclic Galois Groups from \(n\)-th Roots).} Let \(K\)
contain a primitive \(n\)-th root of unity \(\zeta\), let \(a \in K\),
and let \(L = K(\alpha)\) where \(\alpha^n = a\). Then \(L/K\) is Galois
with cyclic Galois group of order dividing \(n\).

\emph{Proof.} The roots of \(x^n - a\) are
\(\alpha, \zeta\alpha, \zeta^2\alpha, \ldots, \zeta^{n-1}\alpha\). Since
\(\zeta \in K \subseteq L\), all roots lie in \(L\), so \(L\) is the
splitting field of \(x^n - a\) over \(K\). In characteristic 0,
\(x^n - a\) is separable. By Theorem 6.4, \(L/K\) is Galois.

For \(\sigma \in \operatorname{Gal}(L/K)\), we have
\(\sigma(\alpha)^n = a\), so \(\sigma(\alpha) = \zeta^{k_\sigma}\alpha\)
for some \(k_\sigma\). The map \(\sigma \mapsto k_\sigma \pmod{n}\) is
an injective homomorphism
\(\operatorname{Gal}(L/K) \to \mathbb{Z}/n\mathbb{Z}\). Thus
\(\operatorname{Gal}(L/K)\) is cyclic of order dividing \(n\).
\(\blacksquare\)

\textbf{Lemma 8.5 (Roots of Unity).} For \(K\) of characteristic 0, the
splitting field of \(x^n - 1\) over \(K\) has abelian Galois group.

\emph{Proof.} Let \(L = K(\zeta)\) for a primitive \(n\)-th root of
unity. Each \(\sigma \in \operatorname{Gal}(L/K)\) satisfies
\(\sigma(\zeta) = \zeta^{a_\sigma}\) for some
\(a_\sigma \in (\mathbb{Z}/n\mathbb{Z})^\times\). The map
\(\sigma \mapsto a_\sigma\) embeds \(\operatorname{Gal}(L/K)\) into the
abelian group \((\mathbb{Z}/n\mathbb{Z})^\times\). \(\blacksquare\)

\textbf{Lemma 8.6 (Compositum of Abelian Extensions).} Let
\(L_1/K, \ldots, L_m/K\) be finite Galois extensions with abelian Galois
groups, all contained in some field \(\Omega\). The compositum
\(L = L_1 \cdots L_m\) satisfies: \(L/K\) is Galois, and
\(\operatorname{Gal}(L/K)\) is abelian.

\emph{Proof.} Each \(L_i\) is the splitting field of a separable
polynomial \(f_i\) over \(K\). Then \(L\) is the splitting field of
\(f_1 \cdots f_m\) over \(K\), hence Galois.

Define
\(\varphi: \operatorname{Gal}(L/K) \to \prod \operatorname{Gal}(L_i/K)\)
by \(\varphi(\sigma) = (\sigma|_{L_1}, \ldots, \sigma|_{L_m})\). This is
injective: if \(\sigma|_{L_i} = \mathrm{id}\) for all \(i\), then
\(\sigma\) fixes \(L\). Thus \(\operatorname{Gal}(L/K)\) embeds into an
abelian group, hence is abelian. \(\blacksquare\)

\textbf{Theorem 8.7 (Solvable by Radicals \(\Rightarrow\) Solvable
Galois Group).} Let \(K\) have characteristic 0. If \(f(x) \in K[x]\) is
solvable by radicals, then \(\operatorname{Gal}(f/K)\) is solvable.

\emph{Proof.} Let \(L\) be the splitting field of \(f\), and let
\(M \supseteq L\) be a radical extension of \(K\) with tower
\(K = K_0 \subseteq K_1 \subseteq \cdots \subseteq K_r = M\), where
\(K_{i+1} = K_i(\alpha_i)\) with \(\alpha_i^{n_i} \in K_i\).

\textbf{Step 1: Adjoin roots of unity.} Let
\(N = \operatorname{lcm}(n_0,\ldots,n_{r-1})\) and let \(\zeta\) be a
primitive \(N\)-th root of unity. Define \(K' = K(\zeta)\) and
\(K'_i = K_i(\zeta)\). The tower
\(K' \subseteq K'_1 \subseteq \cdots \subseteq M' = M(\zeta)\) still has
\(K'_{i+1} = K'_i(\alpha_i)\) with \(\alpha_i^{n_i} \in K'_i\).

\textbf{Step 2: Each step is Galois with cyclic group.} Since \(K'_i\)
contains a primitive \(n_i\)-th root of unity, Lemma 8.4 implies
\(K'_{i+1}/K'_i\) is Galois with cyclic Galois group.

\textbf{Step 3: Pass to normal closures.} Let \(M''\) be the normal
closure of \(M'\) over \(K\). For each \(i\), let \(L_i\) be the normal
closure of \(K'_i\) over \(K\) within \(M''\).

\textbf{Step 4: Build solvable series.} The extension \(L_{i+1}/L_i\) is
generated by conjugates of \(\alpha_i\) over \(K\). Each conjugate
\(\beta\) satisfies \(\beta^{n_i} \in L_i\), and since \(L_i\) contains
all \(n_i\)-th roots of unity, \(L_i(\beta)/L_i\) is Galois with cyclic
Galois group by Lemma 8.4.

The extension \(L_{i+1}/L_i\) is the compositum of these cyclic
extensions. By Lemma 8.6, \(\operatorname{Gal}(L_{i+1}/L_i)\) is
abelian.

\textbf{Step 5: Conclude solvability.} The chain
\(K \subseteq K(\zeta) = L_0 \subseteq L_1 \subseteq \cdots \subseteq L_r = M''\)
has \(\operatorname{Gal}(L_0/K)\) abelian by Lemma 8.5 and each
\(\operatorname{Gal}(L_{i+1}/L_i)\) abelian by Step 4. Thus
\(G = \operatorname{Gal}(M''/K)\) has a subnormal series with abelian
quotients, so \(G\) is solvable.

\textbf{Step 6: Conclude for \(f\).} We have
\(K \subseteq L \subseteq M''\). By the Galois correspondence,
\(\operatorname{Gal}(L/K) \cong G/\operatorname{Gal}(M''/L)\). A
quotient of a solvable group is solvable. Thus
\(\operatorname{Gal}(L/K)\) is solvable. \(\blacksquare\)

\begin{center}\rule{0.5\linewidth}{0.5pt}\end{center}

\hypertarget{the-main-theorem}{%
\section{The Main Theorem}\label{the-main-theorem}}

\textbf{Theorem 9.1 (Abel--Ruffini).} For \(n \geq 5\), the generic
polynomial of degree \(n\) is not solvable by radicals.

\emph{Proof.} Let \(k\) have characteristic 0, let
\(E = k(e_1,\ldots,e_n)\), and let
\(P(t) = t^n - e_1 t^{n-1} + \cdots + (-1)^n e_n\).

By Theorem 7.3, \(\operatorname{Gal}(P/E) \cong S_n\).

By Theorem 4.4, \(S_n\) is not solvable for \(n \geq 5\).

If \(P\) were solvable by radicals, Theorem 8.7 would imply \(S_n\) is
solvable, a contradiction.

Therefore \(P\) is not solvable by radicals. \(\blacksquare\)

\textbf{Corollary 9.2 (No General Algebraic Formula).} There is no
algebraic formula expressing the roots of a general polynomial of degree
\(n \geq 5\) in terms of its coefficients using only
\(+, -, \times, \div\) and extraction of radicals.

\emph{Proof.} Such a formula, applied to \(P(t)\) with indeterminate
coefficients \(e_1,\ldots,e_n\), would express the roots
\(x_1,\ldots,x_n \in k(x_1,\ldots,x_n)\) in terms of \(e_1,\ldots,e_n\)
using radicals. By Lemma 8.3, the roots would then lie in a radical
extension of \(E = k(e_1,\ldots,e_n)\), meaning \(P\) is solvable by
radicals. This contradicts Theorem 9.1. \(\blacksquare\)

\end{document}
