\chapter{Assessment Process}
\label{cp:assessment process}

\textit{Client: Michelle Nguyễn}

\textit{Provisional Psychologist: Kiran Nath}

\textit{Supervising Psychologist: Dr. John Smith}

\textit{Date of Assessment: Friday, 5th of September, 2025}

\textit{Time of Assessment: 4:40 PM}

\textit{Duration of Assessment: 53 minutes, 40 seconds}

\textit{Format of Assessment: Telehealth via Zoom}

\textit{Referral Source: Rebecca Wise, School Counsellor, Kingswood Park High School}

\section{Presenting Problem}

Michelle, a 16-year-old Vietnamese-Australian female, was referred for persistent anxiety symptoms, declining academic performance, and increasing family conflict. The referral letter noted verbal altercations with peers, classroom disruptions, and incomplete homework. Michelle stated her parents "forced" her to come but acknowledged wanting "my parents to listen to me and listen to what I have to say instead of just complaining all the time."

\section{Assessment Findings}

\subsection{Family Dynamics}

Michelle lives with her mother, stepfather, and 13-year-old brother. She described home as "stressful because I'm always having fights with my parents." Her biological father is in rehabilitation for alcohol and drug use with no current contact. Michelle reported her parents divorced when she was eight due to her father's substance use and associated financial problems. She recalled "whenever I slept, I was asleep at that time like at 11:00 p.m. So I usually woke up with them fighting."

Regarding her stepfather, Michelle stated "whatever he says, that's what goes." When seeking maternal support, her mother responds with "talk to your dad" or "I'll talk to your dad about it." Michelle expressed "I don't really mind not having a father" while describing avoiding her stepfather except when needing money for school.

Family conflicts center on clothing choices, academic performance, and social restrictions. Michelle reported her parents call her clothes "revealing" and worry about "perverts," while she considers them "quite normal because all of my friends wear them." Parents have stopped her from leaving the house due to poor grades, requiring her to "keep studying after I come back from school."

\subsection{School Functioning}

Michelle reported difficulty understanding material since starting high school, stating "the subjects have gotten a lot harder and teachers are also not very accommodating." When seeking help, teachers "just circle it back to me that I need to answer my own question." She experiences bullying from "popular girls" who mock her introversion, saying "I am introverted, but I'm still not intelligent." Following conflicts with these students, Michelle reported being "boycotted" by classmates, maintaining only "a small group of friends."

Michelle described fighting back against bullying but stated "the teacher just calls me disruptive" and contacts her parents, who "never believe me" and "always start the conversation with like, again, you have done something."

\subsection{Work Functioning}

Michelle has worked as a cook at a fast-food restaurant for two years. She reported "I'm pretty good at it" and "I enjoy it quite a lot." Her boss praises her work, and she stated "it feels really great compared to school." This job provides financial independence from her stepfather and social connection with coworkers.

\subsection{Substance Use}

Michelle disclosed drinking "two cans of beer" daily "on school days" after school, beginning at age 14. She initially drank due to peer pressure, hoping "maybe if I started drinking, I will fit in with my peers." Currently, alcohol helps her anxiety "vanish" and enables her to "say whatever I wanted." She experiences hangovers causing "really bad headache" making it "hard to focus" during morning classes. Michelle conceals her drinking using "a mint or lemon" to mask the smell.

\subsection{Relationships}

Michelle has a boyfriend, Adrian (age 16), whom she met at work. They dated for six months, broke up, then reunited a year ago. Adrian recently initiated a "break" due to academic pressure. Michelle stated "I wanted to talk it out with him, but he was like, I don't really have time for this." She worries "this break might never end." The relationship remains hidden from her parents. Michelle denied being sexually active, stating "it was never in my family to have sex before marriage."

Her peer support consists of friends whose parents are "way more progressive than mine." These friends help by telling her parents they're having "group study sessions" when socializing. Activities with friends include chess, Monopoly, and basketball, which "relieves the stress a lot."

\subsection{Mental Health Symptoms}

\textbf{Anxiety:} Michelle reported feeling "worried a lot of the time" about schoolwork, appearance, and peer opinions. She stated "I feel like I'm letting everyone down."

\textbf{Sleep:} Michelle sleeps 3-5 hours nightly, going to bed between 2:00-4:00 AM. She reported "I just keep looking at my phone" and after parental arguments, "I keep thinking about the fight... I can't sleep just thinking about it."

\textbf{Appetite} Michelle controls her diet "because I don't want to get fat" and fears being "bullied for being fat" in addition to existing bullying about her acne.

\textbf{Activity:} Previously enjoyed reading but stated "nowadays, it's all I do is read books, like school books. So I don't want any more reading in my life."

\textbf{Safety:} Michelle denied current suicidal ideation, self-harm behaviors, or homicidal ideation.