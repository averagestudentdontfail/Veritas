\chapter{Case Formulation Schematic}
\label{appendix:formulation}

\section*{Biopsychosocial Formulation: Bill Wynsky}

\begin{xltabular}{\textwidth}{|l|X|}
\hline
\textbf{Factor Category} & \textbf{Specific Components} \\
\hline
\endfirsthead

\hline
\textbf{Factor Category} & \textbf{Specific Components} \\
\hline
\endhead

\hline
\endfoot

\hline
\endlastfoot

\textbf{PREDISPOSING FACTORS} & 
\textbf{Biological:}
\begin{itemize}[nosep,leftmargin=*]
\item Genetic vulnerability (paternal alcohol use disorder)
\item Potential epigenetic trauma transmission
\item Neurodevelopmental impacts of childhood adversity
\end{itemize}

\textbf{Psychological:}
\begin{itemize}[nosep,leftmargin=*]
\item Insecure attachment (maternal hospitalisation age 8)
\item Childhood trauma exposure (domestic violence)
\item Negative self-schemas ("weak," "inadequate," "can't protect")
\item Emotion dysregulation vulnerabilities
\end{itemize}

\textbf{Social:}
\begin{itemize}[nosep,leftmargin=*]
\item Immigration stress and acculturation challenges (age 6)
\item Family violence and dysfunction
\item School bullying and educational disruption
\item Limited socioeconomic resources
\end{itemize} \\
\hline

\textbf{PRECIPITATING FACTORS} & 
\begin{itemize}[nosep,leftmargin=*]
\item Combat exposure Afghanistan (12 months service)
\item Index trauma: Friend's death (shot in head, unable to retrieve body)
\item Moral injury (witnessed civilian abuse without intervening)
\item Multiple combat-related traumatic exposures
\item Medical discharge following knee injury (loss of military identity)
\end{itemize} \\
\hline

\textbf{PERPETUATING FACTORS} & 
\textbf{Cognitive:}
\begin{itemize}[nosep,leftmargin=*]
\item Trauma-related stuck points ("I should have saved him")
\item Negative self-concept ("weak," "bad person," "worthless")
\item Overgeneralised threat perception
\item Moral injury cognitions (violated values)
\end{itemize}

\textbf{Behavioural:}
\begin{itemize}[nosep,leftmargin=*]
\item Experiential and situational avoidance
\item Alcohol use for emotional numbing (20--40g weeknights, 100g weekends)
\item Social withdrawal and interpersonal isolation
\item Aggressive behaviour (property destruction)
\end{itemize}

\textbf{Physiological:}
\begin{itemize}[nosep,leftmargin=*]
\item Hyperarousal and hypervigilance
\item Sleep disturbance and parasomnias (sleepwalking)
\item Dysregulated HPA axis and stress response
\item Altered fear circuitry (amygdala hyperactivation)
\end{itemize}

\textbf{Environmental:}
\begin{itemize}[nosep,leftmargin=*]
\item Temporary military accommodation (housing instability)
\item Career uncertainty (fitness for duty unknown)
\item Geographical separation from family support
\item Military culture stigma regarding mental health
\end{itemize} \\
\hline

\textbf{PROTECTIVE FACTORS} & 
\begin{itemize}[nosep,leftmargin=*]
\item Help-seeking behaviour (despite initial reluctance)
\item Family connections maintained (mother, sister Hannah)
\item Military peer support and identity
\item Future-oriented goals (family, meaningful work)
\item Previous adaptive functioning capacity
\item No active suicide plan or intent
\item Physical health intact (successful knee recovery)
\end{itemize} \\
\hline

\end{xltabular}
