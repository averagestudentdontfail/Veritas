\part{REFLECTION}
\label{part:discussion}

\chapter{REFLECTION}

\section{Strengths and Limitations}

I believe this report comprehensively integrates complex trauma presentations within evidence-based frameworks, acknowledging developmental vulnerabilities without deterministic conclusions. I was selective in the intervention selection, and was careful to balance empirical support with practical engagement considerations addressing military culture. However, I do admit that neurobiological factors including potential traumatic brain injury received insufficient consideration. Social determinants (housing instability, employment uncertainty) warranted greater therapeutic planning emphasis. There were also systemic military institutional factors perpetuating tigma that deserved deeper critical analysis.

\section{Personal Clinical Challenges}

Bill's moral injury narratives would certaintly activate my personal distress regarding institutional failures permitting atrocities. I think maintaining therapeutic neutrality whilst validating legitimate moral concerns requires careful self-monitoring preventing either dismissing ethical violations or reinforcing paralysing guilt which i would find difficult. My cognitive intervention preference might overshadow necessary emotional processing given Bill's defensive numbing. His suicidal ideation would trigger personal anxiety potentially leading to overly cautious risk management compromising therapeutic alliance. I would certaintly need extensive supervision to explore countertransference, and would need to rely on my peers to help me manage the disclosure of such traumas, and of course, personal therapy to process any vicarious traumatisation.
