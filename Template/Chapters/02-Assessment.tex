% Chapters/02-Assessment.tex

\part{METHODOLOGY}
\label{part:methodology}

\chapter{ASSESSMENT}
\label{cp:assessment}

\section{Assessment Framework and Rationale}

Comprehensive evaluation for possible ASD and IDD requires systematic integration of multiple information sources and professional perspectives. The gold standard assessment involves multidisciplinary collaboration including paediatrician, psychologist, and speech-language pathologist \parencite{Ozonoff2005}, serving several functions: specialized expertise enabling comprehensive evaluation, cross-validation reducing diagnostic error, and integrated assessment facilitating coordinated treatment planning \parencite{Charman2013}.

For Max, multidisciplinary collaboration proves particularly valuable given that approximately 45\% of individuals with autism also have intellectual developmental disorder \parencite{Mefford2012}, necessitating careful evaluation to determine whether social communication difficulties exceed what would be expected based on nonverbal cognitive abilities alone.

\section{Clinical Interview and Information Sources}

Comprehensive assessment requires systematic information gathering from Stephanie and Charles (detailed developmental history and current functioning observations), ABC preschool teacher and teacher aide (functioning in structured settings with peers through systematic behavior rating scales), Dr. Smith (medical history and longitudinal perspectives), speech pathologist (language development trajectory and intervention response), and direct observation of Max through structured testing, unstructured play, and ideally preschool observation.

A comprehensive developmental interview following established frameworks such as the Autism Diagnostic Interview-Revised (ADI-R; \cite{Lord1994}) provides systematic coverage of areas essential for ASD diagnosis. The 93-item interview (1.5-3 hours) addresses pregnancy and birth history, early feeding and sleeping patterns, motor milestone attainment, language acquisition including pre-linguistic communication and regression, social development (early social responsiveness, attachment, imitation, joint attention, pretend play, peer relationships), play development across types, restricted interests and repetitive behaviours, clinically relevant behaviours, and current functioning across daily living skills and family participation.

Given Max's Aboriginal heritage, the interview should explicitly address family's connection to Aboriginal community and culture, cultural practices relevant to child-rearing, family's understanding of Max's difficulties within cultural framework, and preferences for involvement of Aboriginal Health Workers \parencite{Daniels2014}. This exploration should be conducted with cultural humility, recognizing that families' cultural identification exists on a continuum.

\section{Measures and Psychometric Assessment}

The Autism Diagnostic Observation Schedule, Second Edition (ADOS-2; \cite{Lord2012}) represents the gold standard observational assessment for ASD. For Max, Module 2 (designed for children with phrase speech who are not yet verbally fluent) provides standardized contexts for eliciting social communication behaviours through developmentally appropriate activities. However, the ADOS-2 is not independently diagnostic \parencite{Charman2013}. Diagnosis requires integration of ADOS-2 findings with developmental history, parent reports, and clinical judgment. The instrument demonstrates strong psychometric properties \parencite{Gotham2007}, though cultural considerations require acknowledgment that it was developed and normed primarily on Western populations.

Screening measures provide useful supplementary data. The Social Communication Questionnaire (SCQ) is a 40-item parent questionnaire (10-15 minutes) providing screening across autism symptom domains. The Social Responsiveness Scale, Second Edition (SRS-2) offers 65-item questionnaires completed by parents and teachers (15-20 minutes each), measuring social communication and restricted/repetitive behaviours dimensionally. Obtaining both ratings enables comparison across contexts, potentially revealing environmental variability in symptom expression.

The Wechsler Preschool and Primary Scale of Intelligence, Fourth Edition (WPPSI-IV; \cite{Wechsler2012}) assesses intellectual functioning across Verbal Comprehension, Visual Spatial, and Fluid Reasoning domains. Given Max's speech delays, careful interpretation of Verbal Comprehension scores is essential, as language-based subtests may underestimate nonverbal reasoning abilities. The WPPSI-IV's structure enables examination of discrepancy between verbal and nonverbal abilities, potentially revealing uneven cognitive profiles common in ASD \parencite{Charman2011}. If verbal abilities significantly limit valid administration, the Leiter International Performance Scale, Third Edition \parencite{Roid2013} could provide alternative assessment.

IQ scores in ASD may demonstrate instability in early childhood \parencite{Munson2008}, making reassessment across developmental periods essential. Standard cognitive tests may underestimate abilities due to social communication demands, reduced motivation, anxiety, and difficulty comprehending instructions. Profile analysis examining scatter across subtests provides more useful clinical information than global IQ scores alone \parencite{Flanagan1997}. Careful observation during testing illuminates problem-solving approaches, frustration response, sustained attention capacity, and social referencing behaviours.

The Vineland Adaptive Behaviour Scales, Third Edition (Vineland-3; \cite{Sparrow2016}) evaluates functioning across Communication, Daily Living Skills, Socialization, and Motor Skills domains through semi-structured interview with parents. For differential diagnosis between ASD and IDD, examining the pattern of adaptive scores can illuminate whether deficits are global (consistent with IDD) or whether social communication deficits are disproportionate to other adaptive domains (consistent with ASD; \cite{Klin2007}). Additional assessment through preschool teacher report enables comparison across contexts.

Given Max's reported sensory sensitivities, the Sensory Profile-2 \parencite{Dunn2014} provides parent and teacher questionnaires (15-20 minutes each) assessing sensory processing patterns across multiple modalities. Research indicates that 69-95\% of children with ASD demonstrate atypical sensory processing \parencite{Tomchek2007}. The Child Behaviour Checklist for Ages 1.5-5 (CBCL/1.5-5) completed by parents provides broad-band screening for emotional and behavioural problems, as anxiety disorders and attention-deficit/hyperactivity disorder frequently co-occur with ASD.

\section{Clinical Observation and Timeline}

Systematic clinical observation throughout assessment sessions provides invaluable data complementing standardized testing. Observations should occur across structured testing situations (separation from parents, response to novel adult and environment, attention during tasks, frustration tolerance, social referencing, presence of repetitive behaviours), unstructured free-play scenarios (toy selection and themes, functional versus symbolic play, communication patterns, play flexibility), parent-child interaction (quality of social engagement, communication patterns, affective sharing), and ideally preschool setting (peer interaction quality, response to teacher instructions, transitions, behavioural regulation).

Comprehensive assessment typically requires multiple sessions distributed over several weeks \parencite{Zwaigenbaum2009}. A suggested sequence includes Session 1 (2 hours): developmental interview with parents, completion of questionnaires, introduction to Max with brief play observation; Session 2 (1.5-2 hours): ADOS-2 administration, parent-child interaction observation; Session 3 (1.5-2 hours): cognitive assessment, adaptive behaviour interview; Session 4 (1-1.5 hours): supplementary assessment, preschool observation if possible; Session 5 (1-1.5 hours): feedback session presenting findings, diagnostic formulation, and recommendations.