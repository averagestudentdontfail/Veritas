% Chapters/02-Method.tex

\part{ASSESSMENT METHODOLOGY}
\label{part:methodology}

\chapter{COMPREHENSIVE EVALUATION FRAMEWORK}
\label{cp:evaluation}

\section{Multidisciplinary Assessment Rationale}

Comprehensive evaluation for possible ASD and IDD requires systematic integration of multiple information sources and professional perspectives, an approach reflecting both methodological necessity and ethical responsibility. The gold standard assessment for ASD involves a multidisciplinary team approach, ideally including a paediatrician, psychologist, and speech-language pathologist \parencite{Ozonoff2005}, though resource constraints and geographical factors may necessitate modified approaches while maintaining assessment quality.

This collaborative model serves several functions. First, different professionals contribute specialized expertise enabling comprehensive evaluation across developmental domains that no single discipline can address adequately. Second, cross-validation of findings reduces risk of diagnostic error by enabling multiple perspectives on the same behavioural observations. Third, integrated assessment facilitates coordinated treatment planning by ensuring that all team members understand Max's profile comprehensively \parencite{Charman2013}.

For Max, multidisciplinary collaboration proves particularly valuable given the complexity of differentiating ASD from IDD. Research by \textcite{Mefford2012} indicates that approximately 45\% of individuals with autism also have intellectual developmental disorder, necessitating careful evaluation to determine whether social communication difficulties exceed what would be expected based on nonverbal cognitive abilities alone. This differential diagnosis requires integration of cognitive assessment, adaptive behaviour evaluation, and autism-specific assessment in ways that single-discipline evaluation cannot achieve adequately.

\section{Developmental History and Clinical Interview}

\subsection{Structured Interview Framework}

A comprehensive developmental interview following established frameworks such as the Autism Diagnostic Interview-Revised (ADI-R; \cite{Lord1994}) provides systematic coverage of areas essential for ASD diagnosis while gathering broader developmental information relevant to IDD consideration. The ADI-R is a 93-item interview taking 1.5-3 hours, requiring intensive training for reliable administration and scoring.

The interview should explore developmental milestones across multiple domains including motor development, language acquisition, social development, play skills, and adaptive functioning. Understanding the trajectory of development—whether skills were acquired and subsequently lost (regression), never fully acquired (developmental delay), or developed atypically from the outset—provides essential information for differential diagnosis. The timing, sequence, and pattern of skill acquisition offer insights that cross-sectional assessment cannot capture.

\subsection{Cultural Responsiveness in Assessment}

Given Max's Aboriginal heritage, the interview should explicitly address cultural considerations while avoiding assumptions about the family's cultural identification, practices, or preferences. The interview might explore the family's connection to Aboriginal community and culture, cultural practices relevant to child-rearing and development, preferences for involvement of Aboriginal Health Workers or other cultural supports, and how the family conceptualizes Max's development within cultural frameworks \parencite{Daniels2014}.

This exploration should be conducted with cultural humility, recognizing that families' cultural identification exists on a continuum and that individuals may identify strongly, moderately, or minimally with Aboriginal cultural heritage depending on personal history, community connection, and other factors. The goal is understanding how culture informs the family's understanding of Max's development and what cultural resources might support assessment and intervention, not imposing expectations about cultural practice or identity.

\section{Autism-Specific Diagnostic Assessment}

\subsection{Structured Observational Assessment}

The Autism Diagnostic Observation Schedule, Second Edition (ADOS-2; \cite{Lord2012}) represents the gold standard observational assessment for ASD, providing standardized contexts for eliciting social communication behaviours through developmentally appropriate activities. For Max, Module 2 would likely be most appropriate, designed for children with phrase speech who are not yet verbally fluent.

The ADOS-2 includes activities designed to create opportunities for social communication, including free play, response to joint attention, demonstration task, description of a picture, storytelling, and conversation. The semi-structured format allows flexibility in following the child's interests while ensuring that specific social communication behaviors receive systematic evaluation. Trained examiners code behaviors across social affect and restricted/repetitive behavior domains, generating algorithm scores that inform diagnostic decision-making.

However, the ADOS-2 is not independently diagnostic \parencite{Charman2013}, a limitation sometimes misunderstood by professionals seeking definitive assessment tools. Diagnosis requires integration of ADOS-2 findings with developmental history, parent/caregiver reports, and clinical judgment informed by understanding of developmental context, cultural factors, and individual circumstances. Research by \textcite{Gotham2007} indicates that the ADOS-2 demonstrates strong psychometric properties including sensitivity and specificity, though cultural considerations require acknowledgment that the instrument was developed and normed primarily on Western populations.

\subsection{Cultural Validity Considerations}

The cultural validity of autism assessment tools for Aboriginal Australian children requires careful consideration. While the behaviors assessed by the ADOS-2 (eye contact, social reciprocity, communication patterns) may have some cultural universality, their expression and interpretation can vary across cultural contexts. Aboriginal Australian cultural practices regarding eye contact, social interaction styles, and communication patterns may differ from Western norms embedded in assessment tools.

Assessment should therefore interpret ADOS-2 findings within cultural context rather than applying diagnostic algorithms mechanically. Consultation with Aboriginal health professionals, cultural advisors, or community members can provide essential perspective on whether observed behaviors reflect autism-specific difficulties or culturally normative patterns. This consultation represents not mere procedural compliance but essential methodological rigor ensuring that assessment validity extends across cultural contexts.

\section{Cognitive and Intellectual Assessment}

\subsection{Comprehensive Cognitive Evaluation}

Evaluation for IDD requires comprehensive cognitive assessment using individually administered measures providing information about intellectual functioning across multiple domains. The Wechsler Preschool and Primary Scale of Intelligence, Fourth Edition (WPPSI-IV; \cite{Wechsler2012}) would be appropriate given Max's age, assessing intellectual functioning across Verbal Comprehension, Visual Spatial, Fluid Reasoning, Working Memory, and Processing Speed indices.

Given Max's speech delays, careful interpretation of Verbal Comprehension scores becomes essential, as language-based assessment may underestimate cognitive capabilities when verbal expression difficulties limit performance on verbally mediated tasks. The WPPSI-IV's structure enables examination of discrepancy between verbal and nonverbal abilities, potentially revealing uneven cognitive profiles common in ASD \parencite{Charman2011}. If verbal abilities significantly limit valid administration of standard cognitive assessment, nonverbal intelligence measures such as the Leiter International Performance Scale, Third Edition \parencite{Roid2013} could provide alternative assessment of cognitive capabilities minimally dependent on language comprehension or expression.

\subsection{Cognitive Profile Analysis}

Research by \textcite{Munson2008} indicates that IQ scores in ASD may demonstrate instability, particularly in early childhood, making reassessment across developmental periods essential for understanding cognitive trajectory. Some children show improved scores following intervention, while others show stable or declining trajectories, with these different patterns potentially reflecting different underlying profiles or intervention responsiveness.

Profile analysis examining scatter across subtests provides more clinically useful information than global IQ scores alone \parencite{Flanagan1997}. Understanding relative strengths and weaknesses across different cognitive domains informs both diagnostic formulation and intervention planning. For example, relatively stronger visual-spatial reasoning compared to verbal comprehension might suggest specific teaching approaches leveraging visual learning strengths, while significant weakness in working memory might indicate need for instructional modifications reducing cognitive load.

\section{Adaptive Behaviour Assessment}

The Vineland Adaptive Behaviour Scales, Third Edition (Vineland-3; \cite{Sparrow2016}) represents the gold standard adaptive assessment, evaluating functioning across Communication, Daily Living Skills, Socialization, and Motor Skills domains. For differential diagnosis between ASD and IDD, examining the pattern of adaptive scores can illuminate whether deficits are global (consistent with IDD) or whether social communication deficits are disproportionate to other adaptive domains (consistent with ASD; \cite{Klin2007}).

The Vineland-3 includes multiple forms (Interview, Parent/Caregiver, and Teacher Rating Forms) enabling assessment across contexts and informants. Systematic comparison of parent and teacher reports can reveal whether adaptive functioning varies across settings, potentially indicating that contextual factors (structure, support, demands) influence functioning in ways relevant to intervention planning. Discrepancies between informants may also reflect different expectations or observational opportunities rather than inconsistent child functioning.

Interpretation requires understanding that adaptive behavior scores reflect not only the child's capabilities but also opportunities for skill demonstration and cultural expectations about developmental milestones. For Aboriginal Australian families, expectations about toilet training timing, independent self-care, or peer interaction may differ from mainstream norms embedded in adaptive behavior measures. Assessment should therefore interpret scores within cultural context while recognizing that adaptive functioning reflects the match between individual capabilities and environmental demands rather than decontextualized skill possession.

\section{Sensory Processing Evaluation}

Given Max's reported sensory sensitivities, the Sensory Profile-2 \parencite{Dunn2014} provides parent and teacher questionnaires assessing sensory processing patterns across multiple modalities including auditory, visual, tactile, taste/smell, movement, and body position. Research by \textcite{Tomchek2007} indicates that 69-95\% of children with ASD demonstrate atypical sensory processing, though the wide range reflects methodological variations and population heterogeneity rather than inconsistent findings.

Sensory assessment serves multiple purposes beyond diagnostic clarification. Understanding Max's specific sensory sensitivities and preferences informs intervention planning by identifying environmental modifications that might reduce distress, instructional approaches that accommodate sensory needs, and therapeutic activities that might improve sensory modulation. The assessment also provides baseline data enabling monitoring of whether sensory difficulties improve with intervention or development.

\section{Assessment Timeline and Integration}

Comprehensive assessment typically requires multiple sessions distributed over several weeks, enabling observation across occasions and reducing fatigue effects that might compromise performance validity. Research by \textcite{Zwaigenbaum2009} emphasizes that assessment quality improves when children are evaluated across multiple contexts and occasions, allowing for more valid conclusions about typical functioning patterns rather than performance during single encounters that may be influenced by anxiety, fatigue, or other transient factors.

The assessment sequence might proceed as follows. Initial session includes clinical interview with parents covering developmental history, current concerns, family context, and cultural factors. Second session includes cognitive assessment using WPPSI-IV or alternative measures as appropriate. Third session includes ADOS-2 administration with Max. Fourth session includes additional assessment as needed based on initial findings, potentially including play-based assessment, language sampling, or additional parent interview. Between sessions, parents complete adaptive behavior rating scales (Vineland-3) and sensory profile questionnaires, while teachers or childcare providers complete parallel forms if Max attends group care settings.

Following assessment completion, the multidisciplinary team reviews all findings systematically, considering convergent evidence across measures and informants, contextual factors influencing interpretation, cultural considerations affecting assessment validity, and the relative support for different diagnostic hypotheses. This integration process recognizes that diagnosis emerges from systematic consideration of multiple data sources rather than from any single test score or behavioral observation, with clinical judgment informed by research evidence playing essential roles in reaching conclusions that serve Max and his family well.