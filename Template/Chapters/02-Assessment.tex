% Chapters/02-Method.tex

\part{METHODOLOGY}
\label{part:methodology}

\chapter{ASSESSMENT}
\label{cp:assessment}

\section{Comprehensive Assessment Process}

\subsection{Assessment Framework and Rationale}

Comprehensive evaluation for possible ASD and IDD requires systematic integration of multiple information sources and professional perspectives. The gold standard assessment for ASD involves a multidisciplinary team approach, ideally including a paediatrician, psychologist, and speech-language pathologist \parencite{Ozonoff2005}. This collaborative model serves several functions: different professionals contribute specialized expertise enabling comprehensive evaluation; cross-validation of findings reduces risk of diagnostic error; and integrated assessment facilitates coordinated treatment planning \parencite{Charman2013}.

For Max, multidisciplinary collaboration proves particularly valuable given the complexity of differentiating ASD from IDD. Research by \textcite{Mefford2012} indicates that approximately 45\% of individuals with autism also have intellectual developmental disorder, necessitating careful evaluation to determine whether social communication difficulties exceed what would be expected based on nonverbal cognitive abilities alone.

\subsection{Clinical Interview and Developmental History}

A comprehensive developmental interview following established frameworks such as the Autism Diagnostic Interview-Revised (ADI-R; \cite{Lord1994}) provides systematic coverage of areas essential for ASD diagnosis while gathering broader developmental information relevant to IDD consideration. The ADI-R is a 93-item interview taking 1.5-3 hours, requiring intensive training for reliable administration.

Given Max's Aboriginal heritage, the interview should explicitly address cultural considerations including family's connection to Aboriginal community and culture, cultural practices relevant to child-rearing, and preferences for involvement of Aboriginal Health Workers \parencite{Daniels2014}. This exploration should be conducted with cultural humility, recognizing that families' cultural identification exists on a continuum.

\section{Measures and Psychometric Assessment}

\subsection{Autism-Specific Diagnostic Instruments}

The Autism Diagnostic Observation Schedule, Second Edition (ADOS-2; \cite{Lord2012}) represents the gold standard observational assessment for ASD. For Max, Module 2 would likely be most appropriate, designed for children with phrase speech who are not yet verbally fluent. The ADOS-2 provides standardized contexts for eliciting social communication behaviours through developmentally appropriate activities.

However, the ADOS-2 is not independently diagnostic \parencite{Charman2013}. Diagnosis requires integration of ADOS-2 findings with developmental history, parent/caregiver reports, and clinical judgment. Research by \textcite{Gotham2007} indicates that the ADOS-2 demonstrates strong psychometric properties, though cultural considerations require acknowledgment that the instrument was developed and normed primarily on Western populations.

\subsection{Cognitive Assessment}

Evaluation for IDD requires comprehensive cognitive assessment using individually administered measures. The Wechsler Preschool and Primary Scale of Intelligence, Fourth Edition (WPPSI-IV; \cite{Wechsler2012}) would be appropriate given Max's age, assessing intellectual functioning across multiple domains including Verbal Comprehension, Visual Spatial, and Fluid Reasoning.

Given Max's speech delays, careful interpretation of Verbal Comprehension scores is essential. The WPPSI-IV's structure enables examination of discrepancy between verbal and nonverbal abilities, potentially revealing uneven cognitive profiles common in ASD \parencite{Charman2011}. If verbal abilities significantly limit valid administration, nonverbal intelligence measures such as the Leiter International Performance Scale, Third Edition \parencite{Roid2013} could provide alternative assessment.

Research by \textcite{Munson2008} indicates that IQ scores in ASD may demonstrate instability, particularly in early childhood, making reassessment across developmental periods essential. Profile analysis examining scatter across subtests provides more useful clinical information than global IQ scores alone \parencite{Flanagan1997}.

\subsection{Adaptive Behaviour Assessment}

The Vineland Adaptive Behaviour Scales, Third Edition (Vineland-3; \cite{Sparrow2016}) represents the gold standard adaptive assessment, evaluating functioning across Communication, Daily Living Skills, Socialization, and Motor Skills domains. For differential diagnosis between ASD and IDD, examining the pattern of adaptive scores can illuminate whether deficits are global (consistent with IDD) or whether social communication deficits are disproportionate to other adaptive domains (consistent with ASD; \cite{Klin2007}).

\subsection{Sensory Processing Assessment}

Given Max's reported sensory sensitivities, the Sensory Profile-2 \parencite{Dunn2014} provides parent and teacher questionnaires assessing sensory processing patterns across multiple modalities. Research by \textcite{Tomchek2007} indicates that 69-95\% of children with ASD demonstrate atypical sensory processing, making systematic assessment valuable for intervention planning.

\subsection{Timeline and Sequencing}

Comprehensive assessment typically requires multiple sessions distributed over several weeks, enabling observation across occasions and reducing fatigue effects. Research by \textcite{Zwaigenbaum2009} emphasizes that assessment quality improves when children are evaluated across multiple contexts and occasions, allowing for more valid conclusions about typical functioning patterns.