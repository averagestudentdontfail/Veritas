\part{EVALUATION}
\label{part:evaluation}

\chapter{DISCUSSION}

This analysis revealed that one individual's understanding of how nature experiences influenced psychological wellbeing and environmental attitudes involved multiple dimensions operating both independently and in interaction. The participant's narrative suggested that nature contact provided immediate cognitive-emotional benefits and contributed to longer-term shifts in environmental perspective and motivation, while also highlighting tensions between environmental values and behavioral realities that nature experiences alone appeared unable to resolve.

These findings resonate with existing research documenting relationships between nature exposure and mental health outcomes \citep{bratman2019}, attention restoration \citep{kaplan1995}, and pro-environmental behavior \citep{soga2023}. However, the qualitative analysis revealed subjective dimensions that quantitative studies may not fully capture, particularly regarding the participant's experience of psychological tension between environmental values and actions, and the complex, sometimes ambivalent ways that nature experiences appeared to influence environmental motivation. The theme of disconnection between values and behaviors adds nuance to research examining relationships between nature contact and pro-environmental action, suggesting that these relationships may be mediated by structural and psychological factors that deserve greater attention.

The cognitive restoration theme aligns with attention restoration theory \citep{kaplan1995}, though the participant's experiential account provides texture regarding how this restoration feels subjectively and what specific features of nature might contribute to restorative effects. The perspective transformation theme extends research on nature experiences and psychological wellbeing by highlighting how certain encounters might produce lasting cognitive-emotional resources for managing stress, though the participant's observation that these effects attenuate without reinforcement suggests the importance of regular rather than occasional nature contact.

Several limitations warrant acknowledgment. As single-case analysis, findings reflect one individual's particular experiences and sense-making processes, limiting generalizability to broader populations. The participant's demographic characteristics as urban professional with discretionary time and resources for nature recreation may shape their experiences in ways differing from individuals with different social locations or access to natural environments. The interview method relied on retrospective reflection and verbal articulation, which may not fully capture pre-reflective dimensions of experience or non-verbal forms of knowing. My own relationship with nature and assumptions about its psychological significance likely influenced both the interview process and analytical interpretations, though I maintained reflexive awareness of these influences throughout analysis.

The findings may hold implications for environmental psychology and public health approaches to promoting both wellbeing and environmental stewardship. The participant's account suggested that nature experiences might contribute to multiple outcomes simultaneously, including cognitive restoration, emotional wellbeing, perspective shifts, and environmental motivation, implying that policies supporting equitable nature access could serve diverse individual and societal goals. However, the disconnect between environmental values and behaviors suggests that nature experiences alone may be insufficient for generating substantial behavioral change without addressing structural barriers, including transportation infrastructure, urban design, economic systems, and social norms that constrain pro-environmental action even among concerned individuals.

This analysis demonstrates the value of qualitative approaches for understanding subjective dimensions of human-nature relationships that complement quantitative research on nature exposure and wellbeing. While themes identified here emerged from one individual's account, they suggest patterns that might be explored with larger, more diverse samples through further qualitative or mixed-methods research. The psychological impacts of nature experiences appear to be neither simple nor uniform, instead involving multiple pathways and outcomes that warrant continued investigation through methods honouring both individual particularity and broader patterns across populations and contexts.