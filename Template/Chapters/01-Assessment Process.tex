\chapter{Assessment Process}
\label{cp:assessment process}

\textit{Client: Michelle Nguyễn}

\textit{Provisional Psychologist: Kiran Nath}

\textit{Supervising Psychologist: Dr. John Smith}

\textit{Date of Assessment: Friday, 5th of September, 2025}

\textit{Time of Assessment: 4:40 PM}

\textit{Duration of Assessment: 53 minutes, 40 seconds}

\textit{Format of Assessment: Telehealth via Zoom}

\textit{Referer of Assessment: Rebecca Wise, School Counsellor, Kingswood Park High School}

\section{Presenting Problems and Reason for Referral}

Michelle was referred for psychological assessment due to persistent anxiety symptoms, declining academic performance, and increasing family conflict. The school counsellor reported verbal altercations with peers, classroom disruptions, and incomplete homework. Michelle attended the assessment stating her parents "forced" her to come, though she acknowledged wanting her voice to be heard within her family system. The referral letter indicated that Michelle feels she is "letting everyone down," suggesting internalized pressure and shame regarding multiple role expectations.

\clearpage

\section{Home and Family Environment}

Michelle lives with her mother, stepfather, and 13-year-old brother in what she describes as a highly conflictual environment. The family system reflects complex dynamics shaped by multiple transitions including parental divorce when Michelle was eight years old due to her biological father's substance use, her mother's subsequent remarriage, and ongoing acculturation tensions between Vietnamese heritage values and Australian contemporary norms. Michelle's biological father is currently in rehabilitation for alcohol and drug use, with no current contact maintained.

Michelle experiences her stepfather as controlling and authoritarian, reporting that "whatever he says, that's what goes" in family decisions. Her mother's role has become peripheral, consistently redirecting Michelle's concerns to the stepfather rather than providing direct maternal support. This family structure creates a double bind where Michelle cannot access maternal support while feeling persistently invalidated by paternal authority. The family's Vietnamese cultural background emphasizes collective harmony, academic achievement, and hierarchical respect, values that create tension with Michelle's developmental need for autonomy and identity formation within Australian adolescent culture.

The clinical significance of these family dynamics extends beyond typical adolescent-parent conflict. Michelle's statement "I feel like I'm letting everyone down" reflects internalized shame stemming from her inability to meet competing cultural expectations while maintaining authentic self-expression. The family system lacks secure attachment relationships, with Michelle reporting emotional distance from both biological and stepfather figures, stating "I don't really mind not having a father" while simultaneously expressing distress about the lack of parental understanding and support.

\section{Educational and Occupational Functioning}

Michelle demonstrates significant functional splitting between academic struggle and occupational success. At school, she reports inability to understand increasingly complex material, particularly since beginning high school. She experiences teachers as dismissive, stating they "just circle it back to me" when she seeks help. The school environment has become increasingly hostile due to persistent bullying from what Michelle describes as "popular girls" who ostracize her while mocking her introversion and academic difficulties. She reports being boycotted by classmates following conflicts with these students, leaving her with only a small group of supportive friends.

Conversely, Michelle thrives in her part-time position as a cook at a fast-food restaurant, a role she has maintained for two years. She receives consistent praise from her supervisor and maintains positive peer relationships in this environment. She explicitly stated that work "feels really great compared to school" and provides crucial self-esteem and financial independence from her stepfather. This functional discrepancy suggests that Michelle's difficulties are context-specific rather than pervasive, indicating environmental factors significantly impact her functioning rather than global capability deficits.

From a cultural perspective, Vietnamese families typically prioritize academic achievement as the primary pathway to success and family honor as per \textcite{Nguyen2012}. Michelle's academic struggles represent not just personal failure but perceived family shame, intensifying pressure and anxiety. Her parents' inability to recognize her occupational competence reflects rigid cultural values that may not accommodate alternative definitions of success within the Australian context.

\section{Substance Usage}

Michelle disclosed daily alcohol consumption consisting of two beers after school, beginning at age 14. Initial use was motivated by peer acceptance attempts following bullying experiences, but has evolved to serve anxiety management and emotional regulation functions. She reports alcohol makes her anxiety "vanish" and enables her to "say whatever I wanted," indicating both anxiolytic and disinhibiting effects. This pattern suggests developing psychological dependence, though physical dependence markers were not formally assessed during this initial interview.

The epigenetic vulnerability through paternal substance use history significantly elevates Michelle's risk profile for substance use disorder development as per \textcite{Merikangas1998}. Her alcohol use occurs within a complex trauma context including witnessed parental conflict during early childhood, effective paternal abandonment, and current family dysfunction. Michelle's concealment strategies, including using mints and careful timing to avoid detection, demonstrate both awareness of parental disapproval and entrenchment of use patterns. She reports experiencing hangovers that affect morning academic performance, creating a cyclical pattern where substance use interferes with the very domain causing distress.

\section{Relational and Social Functioning}

Michelle's romantic relationship with Adrian, also 16 years old, reflects broader attachment patterns characterized by approach-avoidance conflict. Their relationship history of initial connection, six-month breakup, reconciliation, and current "break" initiated by Adrian parallels her inconsistent early attachment experiences. Adrian's unilateral decision to pause the relationship reportedly due to academic pressures triggers abandonment concerns while Michelle's statement about lacking "sense of connection" suggests defensive deactivation of attachment needs.

The hidden nature of this relationship from parents reflects both normative adolescent privacy needs and specific cultural tensions around Vietnamese parental expectations regarding adolescent romantic relationships. Michelle's values around physical intimacy, stating it was "never in my family to have sex before marriage," indicate partial internalization of traditional values despite rejecting other parental expectations. This selective adoption of cultural values suggests active identity negotiation rather than wholesale acceptance or rejection of either cultural framework.

Peer relationships remain limited to a small group of friends whose parents Michelle describes as "way more progressive" than her own. These friends provide crucial support, including covering for Michelle when she wants to socialize by telling her parents they are having "group study sessions." Michelle also enjoys activities with these friends including chess, Monopoly, and basketball, which provide stress relief and normalizing adolescent experiences.

\section{Symptomatology of Psychopathology}

Michelle presents with pervasive anxiety across multiple domains including academic performance, peer relationships, family interactions, and identity formation. Her anxiety manifests somatically through sleep disturbance and concentration difficulties, and behaviorally through alcohol use and social withdrawal. The anxiety appears both reactive to environmental stressors and internalized as persistent worry about meeting others' expectations.

Sleep patterns reveal severe disruption with Michelle reporting 3 to 5 hours of sleep nightly, with bedtimes between 2:00 and 4:00 AM. She engages in prolonged phone use and ruminates about interpersonal conflicts, particularly after parental arguments. While Michelle denied current depression, her presentation suggests subsyndromal depressive features including anhedonia evidenced by lost interest in previously enjoyed reading, social withdrawal beyond her small friend group, and persistent negative self-concept.

Eating behaviors indicate body image concerns with controlled diet to avoid weight gain. Michelle reported being bullied for acne and fears additional bullying if she gains weight. While not meeting criteria for an eating disorder, these concerns warrant monitoring given their interaction with anxiety and self-esteem issues.

\section{Risk Assessment}

Michelle denied current suicidal ideation, self-harm behaviors, or homicidal ideation during the assessment. Protective factors include future orientation, work satisfaction, supportive friend network, and demonstrated help-seeking capacity despite ambivalence. Risk factors include daily alcohol use, significant family conflict, social isolation at school, academic failure, and paternal substance use history. Current risk appears low to moderate, requiring ongoing monitoring particularly given the accumulation of stressors and limited coping strategies beyond alcohol use.

\clearpage