\part{EPISTEOMOLOGY}
\label{part:epistemology}

\chapter{METHOD}

\section{Theoretical Positioning}

This study employed reflexive thematic analysis \citep{braun2019}, positioned within a contextualist epistemology that acknowledges both individual meaning-making and broader social-cultural contexts shaping experience. This positioning recognizes that while individuals construct personal meanings from their experiences, these constructions occur within social and cultural frameworks that provide interpretive resources and constraints. The analysis sought patterns of meaning across the dataset rather than quantifying explicit content, privileging interpretative depth over descriptive coverage. This approach recognizes the researcher's active role in theme development, with themes understood as analytical constructs created through the interpretive process rather than emerging passively from data.

As researcher, my own positive experiences with nature and interest in environmental psychology inevitably shaped both the interview and analytical process. I approach long periods in natural settings as psychologically restorative and hold concerns about environmental degradation that influenced what I noticed and how I interpreted the participant's account. Rather than positioning this as bias requiring elimination, I approached it as productive context enabling empathetic engagement while maintaining analytical awareness. Throughout analysis, I attended to how my interpretations reflected particular theoretical lenses and personal experiences, considering alternative readings while making transparent the rationale for interpretations presented here. This reflexive stance acknowledges that the analysis represents one possible reading shaped by my subjectivity, while arguing that this reading is defensible and subject to falsibility when examined against the data.

\section{Participant Protocol}

The participant was a 34-year-old urban professional with self-reported regular nature engagement throughout childhood and continuing into adulthood. Recruitment occurred through purposive sampling, with the participant selected based on their history of diverse nature experiences and willingness to reflect on potential psychological impacts. Following informed consent that assured confidentiality, a semi-structured interview explored the participant's nature experiences and perceived psychological impacts. The 45-minute interview prioritized phenomenological attention to lived experience, using prompts such as ``What happened during that experience?'' and ``How did you feel then?'' rather than ``Why?'' questions that might elicit intellectualized explanations divorced from experiential immediacy. This approach aimed to access concrete descriptions and felt experiences rather than post-hoc theorizing about causal mechanisms.

The interview concluded with invitation for additional reflections and brief summary checking whether key points resonated with the participant's intended meanings, providing opportunity to clarify misunderstandings or add dimensions that had not emerged through the preceding questions. The audio-recorded interview was transcribed verbatim, producing approximately 6,800 words of text.

\section{Analytical Protocol}

Analysis followed \citet{braun2006,braun2019} reflexive approach through six phases, though these phases were not strictly linear but involved recursive movement between different analytical activities. Phase 1 involved repeated reading of the transcript while listening to audio, developing familiarity with both semantic content and emotional tone conveyed through vocal expression. This immersion generated preliminary impressions about potential areas of interest, though formal coding had not yet commenced.

Phase 2 generated 78 initial codes through systematic line-by-line examination, capturing both semantic content, what the participant explicitly stated, and latent meanings representing interpretive reading of underlying concepts or assumptions. Coding was inclusive at this stage, with segments receiving multiple codes where applicable and maintaining sufficient surrounding context to preserve meaning. Examples of codes included ``childhood nature as freedom,'' ``nature reducing mental clutter,'' ``feeling small in natural settings,'' and ``guilt about environmental impact.''

Phase 3 involved sorting codes into candidate themes through visual mapping techniques, exploring relationships and patterns across codes by arranging them spatially to identify potential clustering. This exploratory phase generated approximately eight candidate themes, though their boundaries remained tentative and relationships among them unclear. Phase 4 reviewed candidate themes for internal homogeneity, checking that coded extracts within each theme formed coherent patterns, and external heterogeneity, ensuring themes were sufficiently distinct from one another. This review led to collapsing two candidate themes that proved difficult to distinguish consistently, subdividing one theme encompassing qualitatively different content, and eliminating one candidate theme appearing weakly supported by the data.

Phase 5 defined and named each theme, developing preliminary analytical narratives specifying scope and boundaries while clarifying relationships among themes. Theme names were developed to be concise yet sufficiently descriptive to convey the essence of each pattern. Phase 6 involved selecting illustrative extracts and constructing the analytical narrative presented below, with extract selection aimed at providing vivid examples capturing theme essence while representing the range of content within themes. Throughout this process, theme identification represented one possible reading shaped by my theoretical understanding and experiences, considering alternative interpretations while presenting the structure judged most strongly supported by the data.