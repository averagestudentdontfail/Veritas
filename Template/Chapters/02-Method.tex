\chapter{Method}
\label{cp:method}

\section{Research Design}

\subsection{Study Type and Participants}
This randomized controlled trial will recruit 48 couples currently undergoing fertility treatment through reproductive medicine clinics in metropolitan areas. Inclusion criteria encompass couples who have attempted conception for at least twelve months, score above clinical thresholds on the Depression Anxiety Stress Scales, and demonstrate relationship distress on the Dyadic Adjustment Scale. Exclusion criteria include active substance use disorders, current psychotic symptoms, recent domestic violence, or concurrent couples therapy. Random allocation will assign 24 couples to immediate EFT intervention and 24 to waitlist control, with the control group receiving treatment after post-intervention assessment.

\subsection{Study Protocol}

The EFT intervention follows Johnson's validated three-stage model across twelve weekly sessions of 90 minutes duration. Stage One focuses on deescalation through identifying negative interaction cycles and accessing underlying attachment emotions. Therapists guide couples to recognize how infertility-related fears trigger pursuit-withdrawal patterns that amplify distress. Stage Two involves restructuring attachment bonds through expressing vulnerability and responding with emotional accessibility. Partners learn to share primary emotions underlying secondary reactions, particularly fears of abandonment and inadequacy triggered by reproductive challenges. Stage Three consolidates new interaction patterns and develops resilience narratives that integrate infertility experiences within secure attachment frameworks.

Treatment fidelity protocols include therapist training through certified EFT trainers, weekly supervision with fidelity monitoring, and random session coding using the EFT Therapist Fidelity Scale. All therapists must achieve certification standards and maintain fidelity ratings above established thresholds throughout intervention delivery.

\subsection{Sample Size and Power-Analysis Calculation}
Power analysis calculations indicate that detecting a large effect size (d = 0.80), based on conservative estimates from previous EFT research with distressed populations, requires 21 couples per group to achieve 80 percent power at alpha level 0.05. The recruitment target of 24 couples per group provides buffer for anticipated 15 percent attrition, ensuring adequate power for primary analyses. Secondary analyses examining moderator effects of infertility duration and gender will employ the full sample, though power for interaction effects remains exploratory given sample constraints.

\subsection{Data Collection and Outcome Measures}

Primary outcomes encompass psychological distress measured through the Depression Anxiety Stress Scales and relationship satisfaction assessed via the Dyadic Adjustment Scale, administered at baseline, post-intervention, and three-month follow-up. Secondary measures include the Fertility Quality of Life questionnaire, Adult Attachment Scale, and treatment persistence indicators from medical records. Process measures collected at mid-treatment examine therapeutic alliance and emotional processing depth to investigate mechanisms of change.

\subsection{Analytical Protocol}

Multilevel modeling will accommodate the dyadic data structure, with individuals nested within couples. Primary analyses will employ intent-to-treat principles using maximum likelihood estimation for missing data. Effect sizes will be calculated using Hedge's g to enable comparison with existing literature. Moderation analyses will explore whether treatment effects vary by infertility cause, duration, or baseline symptom severity.

\section{Fiscal Budget and Justification}
The following fiscal budget reflects realistic costs for conducting rigorous intervention research while maintaining fiscal responsibility. The staffing allocation ensures adequate expertise for complex dyadic analyses while therapist costs reflect market rates for specialized clinical services. Participant reimbursement supports retention in this vulnerable population experiencing multiple stressors. The total investment yields potential benefits including reduced mental health service utilization, improved fertility treatment outcomes, and improved relationship stability for affected couples.