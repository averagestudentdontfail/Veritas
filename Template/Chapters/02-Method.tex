% Chapters/02-Method.tex

\chapter{Method}
\label{cp:method}

\section{Participants and Recruitment}

This study will recruit 48 heterosexual Australian couples currently experiencing infertility from fertility clinics in metropolitan Sydney, Melbourne, and Brisbane. Infertility is operationally defined according to World Health Organization criteria as failure to achieve clinical pregnancy after twelve months of regular unprotected intercourse for women under 35 years, or six months for women 35 years or older, confirmed through medical documentation from treating fertility specialists.

Inclusion criteria encompass couples where both partners score above mild symptom thresholds on at least one DASS-42 subscale (Depression $>$ 14, Anxiety $>$ 10, or Stress $>$ 19), indicating clinically relevant distress requiring intervention. Couples must demonstrate relationship commitment through cohabitation for minimum twelve months and both partners must consent to participation. Exclusion criteria include current substance use disorders assessed through AUDIT and DUDIT screening instruments, active psychotic symptoms evaluated via clinical interview, domestic violence history screened through Conflict Tactics Scale, or concurrent couples therapy that would confound treatment effects.

\section{Sample Size Calculation}

Sample size determination employed G*Power 3.1 software using the following formula for independent samples t-test:

\begin{equation}
n = \frac{2\sigma^2(Z_{\alpha/2} + Z_{\beta})^2}{\delta^2}
\end{equation}

Where $\sigma$ represents population standard deviation, $Z_{\alpha/2}$ represents critical value for Type I error (1.96 for $\alpha$ = 0.05), $Z_{\beta}$ represents critical value for Type II error (0.84 for $\beta$ = 0.20), and $\delta$ represents minimum detectable difference.

Based on previous EFT research demonstrating effect sizes ranging from 0.80 to 2.09, conservative estimation using d = 0.80 yields:

\begin{equation}
n = 2 \times \left[\frac{(1.96 + 0.84)}{0.80}\right]^2 = 24.5 \text{ couples per group}
\end{equation}

Accounting for 15\% attrition based on previous couples intervention research, final recruitment target equals 28 couples per group, totaling 56 couples. However, resource constraints limit recruitment to 48 couples (24 per group), providing 80\% power to detect large effects (d = 0.88) while accepting reduced power for medium effect detection.

\section{Intervention Protocol}

Emotionally Focused Couples Therapy follows Johnson's (2019) validated treatment manual across twelve weekly 90-minute sessions delivered by certified EFT therapists. The intervention progresses through three systematically defined stages with specific therapeutic tasks and process markers.

\textbf{Stage One} encompasses sessions one through four, focusing on assessment and deescalation. Therapists identify negative interaction cycles specific to infertility stressors, mapping pursuit-withdrawal patterns triggered by reproductive challenges. Couples learn to recognize how secondary emotions like anger mask primary attachment fears of abandonment or inadequacy. Process markers include couples articulating their cycle using "when-then" statements and expressing understanding of partner's position within the cycle.

\textbf{Stage Two} spans sessions five through nine, restructuring attachment bonds through choreographed enactments. Partners express attachment needs and fears directly while therapists facilitate responsive engagement from the other partner. Withdrawal partners articulate fears underlying distancing behaviors while pursuing partners soften demands to create emotional safety. Successful stage completion requires both partners accessing and expressing vulnerable emotions while experiencing acceptance from their partner.

\textbf{Stage Three} comprises sessions ten through twelve, consolidating new patterns and developing resilience narratives. Couples practice applying secure interaction patterns to infertility-specific challenges including treatment decisions and pregnancy loss. Therapists guide creation of relationship stories integrating infertility experiences within broader relationship meaning, establishing templates for future challenge navigation.

\section{Measurement Strategy}

Primary outcome assessment employs the Depression Anxiety Stress Scales-42 (DASS-42; \textcite{Lovibond1995}), a psychometrically validated instrument demonstrating internal consistency coefficients of 0.91 for depression, 0.84 for anxiety, and 0.90 for stress subscales. Australian normative data enables clinical threshold determination with established cut-points distinguishing mild, moderate, severe, and extremely severe symptom levels.

Relationship satisfaction assessment utilizes the Dyadic Adjustment Scale (DAS; \textcite{Spanier1976}), comprising 32 items measuring dyadic consensus, satisfaction, cohesion, and affectional expression. Clinical distress threshold of 97 distinguishes distressed from non-distressed couples with demonstrated sensitivity to therapeutic change. Australian validation studies confirm factor structure and predictive validity for relationship outcomes \parencite{Sharpley1982}.

Secondary measures include the Fertility Quality of Life questionnaire (FertiQoL; \textcite{Boivin2011}) assessing fertility-specific emotional, relational, and social functioning across 36 items with established international norms. The Experiences in Close Relationships-Revised (ECR-R; \textcite{Fraley2000}) measures attachment anxiety and avoidance dimensions relevant to EFT mechanisms. Process assessment employs the Working Alliance Inventory \parencite{Horvath1989} at sessions 3, 6, and 9 to monitor therapeutic engagement.

\section{Data Analysis Plan}

Primary analyses employ multilevel modeling accommodating dyadic data structure with individuals (Level 1) nested within couples (Level 2) using the following equation:

\begin{equation}
Y_{ij} = \beta_0 + \beta_1(\text{Time}) + \beta_2(\text{Treatment}) + \beta_3(\text{Time} \times \text{Treatment}) + u_j + e_{ij}
\end{equation}

Where $Y_{ij}$ represents outcome for individual $i$ in couple $j$, $\beta$ coefficients represent fixed effects, $u_j$ represents random couple intercept, and $e_{ij}$ represents residual error.

Effect sizes will be calculated using Hedges' g formula accounting for small sample bias:

\begin{equation}
g = \frac{M_1 - M_2}{S_{\text{pooled}}} \times \left(1 - \frac{3}{4N - 9}\right)
\end{equation}

Intent-to-treat principles guide primary analyses with maximum likelihood estimation handling missing data under missing-at-random assumptions. Sensitivity analyses compare complete-case results to evaluate missing data impact. Moderation analyses explore treatment effect variation by infertility duration, cause, and baseline severity using interaction terms within multilevel models.

\section{Indicative Budget and Budget Justification}

The project budget employs activity-based costing methodology normalizing expenses across the 18-month study period using the formula:

\begin{equation}
\text{Total Cost} = \sum(\text{Resource Units} \times \text{Unit Cost} \times \text{Time Allocation})
\end{equation}

Personnel costs constitute the primary expenditure, calculated using Australian university salary scales plus 30\% on-costs for superannuation, leave provisions, and payroll tax. Research positions align with Higher Education Worker classifications based on qualification requirements and responsibility levels. Therapeutic services reflect Australian Psychological Society recommended rates for specialized couples therapy. Participant reimbursement follows National Health and Medical Research Council guidelines for research participation compensation.

\begin{landscapelongtable}
\begin{longtable}{@{}p{5cm}p{6cm}p{3cm}@{}}
    \caption{Detailed Project Budget}
    \label{tab:project-budget}\\
    
    % First header
    \toprule
    \textbf{Budget Category} & \textbf{Calculation} & \textbf{Total Cost} \\
    \midrule
    \endfirsthead
    
    % Continued header for subsequent pages
    \multicolumn{3}{c}{\tablename\ \thetable{} -- \textit{Continued from previous page}} \\
    \toprule
    \textbf{Budget Category} & \textbf{Calculation} & \textbf{Total Cost} \\
    \midrule
    \endhead
    
    % Footer for pages except last
    \midrule
    \multicolumn{3}{r}{\textit{Continued on next page}} \\
    \endfoot
    
    % Footer for last page
    \bottomrule
    \endlastfoot
    
    % Table content
    \multicolumn{3}{l}{\textbf{Personnel}} \\
    Research Officer (HEW 6.4) & \$89,426 $\times$ 0.4 FTE $\times$ 1.5 years $\times$ 1.3 on-costs & \$69,751 \\
    Senior Research Officer (HEW 7.2) & \$93,841 $\times$ 0.2 FTE $\times$ 1.5 years $\times$ 1.3 on-costs & \$36,598 \\
    \\
    \multicolumn{3}{l}{\textbf{Clinical Services}} \\
    EFT Therapists & 3 therapists $\times$ 288 sessions $\times$ \$200/session & \$57,600 \\
    Clinical Supervision & 36 hours $\times$ \$250/hour & \$9,000 \\
    \\
    \multicolumn{3}{l}{\textbf{Participant Costs}} \\
    Reimbursement & 48 couples $\times$ 3 assessments $\times$ \$50/assessment & \$7,200 \\
    \\
    \multicolumn{3}{l}{\textbf{Materials and Resources}} \\
    Assessment Licenses & DASS, DAS, FertiQoL licenses $\times$ 150 administrations & \$3,500 \\
    Data Management & REDCap database + SPSS license & \$2,000 \\
    \\
    \multicolumn{3}{l}{\textbf{Dissemination}} \\
    Conference and Publication & Registration + travel + open access fees & \$4,000 \\
    \\
    \multicolumn{3}{l}{\textbf{Indirect Costs}} \\
    University Overhead & 10\% of direct costs & \$18,965 \\
    \\
    \midrule
    \textbf{Total Project Budget} & & \textbf{\$208,614} \\
    
\end{longtable}
\end{landscapelongtable}

\clearpage
\restoregeometry
\newpage
\thispagestyle{plain}