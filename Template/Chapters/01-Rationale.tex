\part{FOUNDATION}
\label{part:foundation}

\chapter{RATIONALE}

Modern urbanisation patterns have reduced opportunities for regular nature contact among many populations \citep{cox2017}, raising questions about consequences for both individual wellbeing and environmental stewardship. While quantitative research has documented associations between nature exposure and mental health outcomes \citep{bratman2019,gascon2015}, these approaches may not fully capture the subjective meanings individuals attach to their nature experiences or the personal narratives through which people understand connections between nature contact and their psychological states. Qualitative investigation could reveal dimensions of these relationships that remain less visible in survey-based research, particularly regarding how individuals make sense of the mechanisms through which nature experiences might influence both immediate affective states and longer-term patterns of environmental concern.

This study explored how one person articulates the psychological impacts of their nature experiences, with particular attention to perceived influences on wellbeing and environmental attitudes. The research question guiding this investigation was: How do personal experiences with nature appear to shape an individual's psychological wellbeing and environmental attitudes? Through detailed examination of one person's account, this analysis aimed to identify themes that might illuminate broader patterns while respecting the particular context and meaning-making processes of the individual participant.