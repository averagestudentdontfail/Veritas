\chapter{Introduction}
\label{cp:introduction}

\section{Scope, Impact and Significance}

Infertility affects approximately one third of Australian couples, creating profound psychological distress that extends beyond individual suffering to threaten relationship stability and mental health. The convergence of failed conception attempts, financial strain from fertility treatments, and social isolation creates a cascade of depression, anxiety, and relationship dysfunction that current medical interventions inadequately address. This project proposes implementing Emotionally Focused Couples Therapy to address the psychological and relational consequences of infertility, providing evidence-based support that complements medical treatment. The intervention targets both individual psychological symptoms and dyadic communication patterns, addressing the bidirectional relationship between emotional distress and relationship quality. Given that relationship discord during infertility predicts both treatment discontinuation and poorer mental health outcomes, this intervention addresses a critical gap in current fertility care provision, potentially improving both psychological wellbeing and treatment persistence for affected couples.

\section{Literature Review}

\subsection{Evidence Basis}
The psychological burden of infertility creates distress comparable to cancer diagnosis, with prevalence rates of depression reaching 57 percent and anxiety affecting 67 percent of women undergoing fertility treatment (Gozuyesil et al., 2019). Recent systematic reviews demonstrate that Emotionally Focused Couples Therapy produces substantial improvements in relationship satisfaction for couples experiencing various stressors, with effect sizes exceeding those of alternative interventions (Beasley & Ager, 2019). The meta-analytic evidence reveals a weighted effect size of 2.09 for EFT interventions with couples, indicating remarkably strong therapeutic benefits that persist at follow-up assessment.

The theoretical foundation of EFT aligns particularly well with infertility-related distress through its focus on attachment disruption and emotional processing. Infertility fundamentally threatens the attachment bond between partners, triggering cascading patterns of pursuit and withdrawal that amplify individual distress while eroding relationship quality (Johnson, 2015). Research demonstrates that couples who develop secure attachment patterns and effective emotional communication show improved coping with infertility stress, reduced psychological symptoms, and enhanced treatment persistence (Peloquin et al., 2018). Furthermore, meaning-based coping strategies, which EFT facilitates through emotional processing and attachment restructuring, predict superior psychological outcomes compared to avoidance or problem-focused approaches alone.

\subsection{Research Question}
The proposed research addresses this critical question through systematic investigation: In couples experiencing infertility (Population), does Emotionally Focused Couples Therapy (Intervention) compared to waitlist control (Comparison) reduce psychological distress and improve relationship satisfaction (Outcomes)? This PICO-structured inquiry enables precise hypothesis testing while maintaining clinical relevance for implementation in fertility treatment settings.



