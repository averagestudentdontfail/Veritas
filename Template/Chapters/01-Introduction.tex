% Chapters/01-Introduction.tex

\chapter{Introduction}
\label{cp:introduction}

\section{Scope, Impact and Significance}

Infertility, operationally defined as the inability to achieve clinical pregnancy after twelve months of regular unprotected sexual intercourse or six months for women aged 35 years or older \parencite{Zegers2017}, affects one in six Australian couples according to Family Planning Australia (2021). This reproductive challenge creates complex psychological sequelae characterized by clinically significant symptoms of depression, anxiety, and stress that meet diagnostic thresholds on standardized assessment instruments. The psychological burden extends beyond individual symptomatology to encompass relationship distress, defined as scores below 97 on the Dyadic Adjustment Scale indicating clinically significant relationship dysfunction \parencite{Spanier1976}.

The Australian healthcare system currently addresses infertility through medical interventions including assisted reproductive technologies, yet psychological support remains fragmented and inadequately integrated within fertility care pathways. This project proposes implementing Emotionally Focused Couples Therapy, a manualized intervention based on attachment theory that restructures emotional responses and interaction patterns through systematic therapeutic processes \parencite{Johnson2019}. The intervention addresses both individual psychological distress and dyadic functioning through three operationalized stages: cycle deescalation, attachment restructuring, and consolidation of secure bonding patterns.

The significance of this intervention extends beyond symptom reduction to address fundamental attachment disruptions that infertility creates within couple relationships. When reproductive expectations remain unfulfilled, partners experience attachment injury characterized by accessibility failures and responsiveness breakdowns that amplify individual distress while eroding relationship security. This project provides systematic intervention addressing these attachment disruptions through evidence-based therapeutic processes, potentially improving both psychological outcomes and fertility treatment persistence for Australian couples navigating reproductive challenges.

\section{Literature Review}

\subsection{Empirical Foundation for Intervention}

The psychological impact of infertility demonstrates consistency across international samples, with Australian data revealing comparable distress patterns to global populations. Depression, operationally defined as scores exceeding 21 on the Depression subscale of the Depression Anxiety Stress Scales indicating severe symptomatology \parencite{Lovibond1995}, affects between 23\% and 57\% of individuals undergoing fertility treatment. Anxiety disorders, characterized by excessive worry and physiological arousal scoring above clinical thresholds, manifest in 67\% of women experiencing fertility challenges \parencite{Gozuyesil2019}.

Recent meta-analytic evidence examining Emotionally Focused Couples Therapy demonstrates robust effectiveness for relationship distress across diverse populations. \textcite{Beasley2019} analyzed nine randomized controlled trials revealing a weighted effect size using Hedges' g = 2.09 (95\% CI: 0.04, 4.14), indicating substantial therapeutic benefit exceeding conventional intervention thresholds. Follow-up analyses demonstrated maintenance of gains with Friedman's test revealing sustained improvement ($\chi^2$ = 6.500, p = 0.039), suggesting durability of therapeutic changes beyond active intervention periods.

The theoretical mechanisms underlying EFT align particularly with infertility-related distress patterns. Attachment theory posits that threats to reproductive goals activate attachment systems, triggering hyperactivation strategies characterized by anxious pursuit or deactivation strategies manifesting as emotional withdrawal \parencite{Mikulincer2016}. These patterns create negative interaction cycles wherein one partner's pursuit for emotional connection triggers the other's withdrawal, establishing self-perpetuating distress patterns that EFT specifically targets through systematic intervention processes.

Australian couples face unique contextual factors influencing infertility experiences, including Medicare funding limitations for assisted reproductive technologies and geographical barriers to specialized fertility services. These structural constraints compound psychological distress, making accessible psychological interventions particularly critical for this population. Evidence from Iranian samples demonstrates EFT effectiveness in reducing depression, anxiety, and stress among infertile couples \parencite{Soltani2014}, though Australian-specific outcome data remains absent, highlighting the need for culturally contextualized intervention research.

\subsection{Research Question and Hypotheses}

This study addresses the following primary research question using PICO framework: Among Australian couples experiencing infertility (Population), does Emotionally Focused Couples Therapy (Intervention) compared to waitlist control (Comparison) reduce psychological distress and improve relationship satisfaction (Outcomes)?

Primary hypotheses specify that couples receiving EFT will demonstrate significantly greater reductions in depression, anxiety, and stress scores on the DASS-42 compared to waitlist controls, with effect sizes exceeding d = 0.80. Secondary hypotheses predict improved relationship satisfaction on the Dyadic Adjustment Scale and improved fertility-specific quality of life, with treatment gains maintained at three-month follow-up assessment.