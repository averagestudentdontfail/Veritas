\part{DIAGNOSES}
\label{part:diagnoses}

\chapter{DIAGNOSES}

\section{Provisional Diagnoses}

Bill's clinical presentation meets criteria for multiple DSM-5-TR diagnoses requiring careful differential assessment \parencite{apa2022}. Post-Traumatic Stress Disorder [309.81] emerges as the primary diagnosis, with Criterion A unequivocally met through direct combat exposure and witnessing his closest friend's death via gunshot to the head in Afghanistan. Bill demonstrates all required symptom clusters with clinically significant severity: intrusive symptoms manifest through vivid combat-themed nightmares occurring biweekly, unwanted trauma memories intruding during waking hours, and physiological reactivity to trauma cues (touching friend's dog tag); persistent avoidance evidenced by steadfast refusal discussing military experiences, emotional numbing during trauma recounting, and avoidance of reminders including military media; negative cognition and mood alterations encompassing persistent negative beliefs ("I am weak," "I'm a bad person"), persistent shame and guilt, markedly diminished interest in previously enjoyed activities (soccer, socialising), emotional detachment from family members, and constricted positive affect; marked arousal and reactivity alterations including irritability culminating in property destruction (punching walls), hypervigilance, exaggerated startle responses, concentration impairment affecting daily functioning, and severe sleep disturbance with new-onset parasomnias. Symptoms persist beyond one month with profound functional impairment across interpersonal, occupational, and social domains \parencite{friedman2011,phoenix2019}.

Major Depressive Disorder, Moderate [296.22] diagnosis appears warranted given persistent depressed mood described as feeling "low" and "empty," anhedonia affecting previously pleasurable activities, excessive worthlessness and inappropriate guilt unrelated to trauma, diminished concentration capacity, and recurrent death thoughts with passive suicidal ideation (visualising throwing himself under buses), persisting several months representing marked change from premorbid functioning \parencite{kroenke2001}. Alcohol Use Disorder, Mild [305.00] reflects problematic consumption patterns demonstrating tolerance development (increased quantities: 20--40g weeknights, 100g weekends versus previous social drinking), unsuccessful control efforts despite recognition of problems, continued use despite persistent interpersonal consequences (relationship breakdown), and psychological dependence using alcohol for stress management, meeting three DSM-5-TR criteria \parencite{saunders1993}.

\section{Differential Diagnoses}

Complex PTSD (ICD-11) warrants primary differential consideration given comprehensive symptom presentation meeting all diagnostic requirements: core PTSD symptoms plus self-organisation disturbances encompassing affect dysregulation (anger outbursts, emotional numbing), negative self-concept ("bad person," persistent shame), and interpersonal difficulties (detachment, relationship failures), though this diagnosis exists outside DSM-5-TR nosology \parencite{cloitre2013,who2018}. Persistent Complex Bereavement Disorder remains differential given traumatic loss circumstances, though broader symptomatology exceeds bereavement-specific criteria \parencite{prigerson2009}. Adjustment Disorder with Mixed Anxiety and Depressed Mood [309.28] appears insufficient given symptom severity, duration exceeding six months, and specific PTSD criteria fulfilment. REM Sleep Behavior Disorder necessitates polysomnography evaluation given new-onset sleepwalking potentially representing dream enactment behaviour associated with combat nightmares \parencite{mysliwiec2014}.

\section{DSM-5 Diagnostic Challenges}

The DSM-5-TR categorical framework presents substantial limitations capturing Bill's complex phenomenology. Artificial diagnostic boundaries separate interconnected symptoms; alcohol use, depression, and PTSD manifestations likely represent integrated affect regulation attempts rather than discrete, independent disorders, risking fragmented treatment approaches \parencite{vanderkolk2005}. The framework inadequately conceptualises developmental trauma's profound personality organisation impact; Bill's childhood domestic violence witnessing, perceived maternal protection failure during psychiatric hospitalisation, and immigration-related cultural dislocation created foundational vulnerabilities that military trauma subsequently reactivated, yet this developmental trajectory receives minimal diagnostic weight within current nosology \parencite{herman1992}. Additionally, DSM-5-TR fails adequately addressing moral injury: Bill's profound survivor guilt and witnessing civilian abuse without intervening represents violated deeply-held moral beliefs requiring different therapeutic approaches than fear-based traumatic stress \parencite{litz2009}.