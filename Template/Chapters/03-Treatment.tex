% Chapters/03-Treatment.tex

\part{PLANNING}
\label{part:planning}

\chapter{INTERVENTION}
\label{cp:intervention}

\section{Evidence-Based Treatment Approaches}

Research on ASD intervention has expanded substantially, with systematic reviews guiding evidence-based practice \parencite{Reichow2012, Warren2011}. Interventions categorize into comprehensive approaches targeting broad developmental domains versus focused interventions addressing specific skills \parencite{NationalResearchCouncil2001}.

Early intensive behavioural intervention (EIBI) based on Applied Behaviour Analysis principles typically involves 20-40 hours per week of structured teaching. Research suggests EIBI can produce improvements in intelligence, language, and adaptive functioning, though effect sizes vary and response is heterogeneous \parencite{Reichow2012}. Developmentally-based approaches integrating behavioral principles with relationship-focused methods include the Early Start Denver Model (ESDM), which may produce changes in brain activity patterns alongside behavioral improvements \parencite{Dawson2012}.

Focused interventions target specific developmental domains. Joint attention interventions show particular promise, demonstrating improvements in both joint attention abilities and language outcomes \parencite{Kasari2010}. Social skills training programs teach specific social competencies through direct instruction, modeling, and role-play, though generalization to natural contexts remains challenging. Communication-focused interventions including the Picture Exchange Communication System (PECS) can increase communication initiations, though evidence for effects on spoken language development remains mixed \parencite{Maglione2012}.

Parent-mediated interventions teach parents to implement intervention strategies during daily routines, enabling intensive intervention within natural contexts. Research shows variable results, with some studies demonstrating improvements in parent-child interaction and child communication. Parental involvement represents a critical component regardless of specific model. Interventions for co-occurring concerns address the high rates of anxiety, attention difficulties, and behavioural challenges through modified cognitive-behavioural therapy, functional behaviour assessment, and parent training in behaviour management.

\section{Naturalistic Developmental Behavioral Intervention}

For Max, naturalistic developmental behavioral intervention (NDBI) approaches represent a particularly appropriate evidence-based strategy. NDBI integrates principles from applied behavior analysis with developmental science, implemented within natural play-based contexts \parencite{Schreibman2015}.

\subsection{Conceptual Foundation and Rationale}

NDBI approaches rest on several principles: young children learn most effectively through natural social interactions and play rather than highly structured teaching; embedding learning within child-initiated activities enhances motivation; using natural consequences produces more meaningful learning than arbitrary reinforcers; and targeting pivotal developmental skills (joint attention, imitation, affect sharing) produces cascading effects across multiple domains.

The Early Start Denver Model \parencite{Rogers2010} exemplifies NDBI approaches, combining ABA teaching principles with developmental relationship-based strategies. Key features include targeting developmental skills across all domains, teaching within playful social interactions following the child's interests, using naturalistic behavioural teaching strategies, implementing intervention intensively (20-25 hours weekly ideally) across contexts, involving parents as active intervention agents, and monitoring progress through frequent data collection \parencite{Dawson2010}.

Several factors suggest NDBI would be suitable for Max. His young age (5 years) falls within the developmental window where early intensive intervention demonstrates strongest effects \parencite{Rogers2012}. His emerging communication abilities (word combinations) suggest he is positioned to benefit from intervention targeting language expansion within social contexts. His strong vehicle interests provide natural motivators. His good sleep patterns suggest capacity to tolerate intensive intervention. The availability of both parents and his teacher aide as potential intervention agents enables intervention across contexts.

NDBI approaches can be culturally adapted more readily than highly structured ABA approaches \parencite{Schreibman2015}, potentially aligning better with Aboriginal cultural values emphasizing learning through observation and participation in meaningful activities. However, explicit exploration with Charles and Stephanie would be necessary to understand their preferences.

\subsection{Implementation and Expected Outcomes}

NDBI implementation begins with comprehensive assessment using developmental frameworks evaluating skills across receptive communication, expressive communication, social skills, imitation, cognition, play, motor skills, behaviour, and independence domains \parencite{Rogers2010}. Goal-setting occurs collaboratively with parents. For Max, potential priority goals might include increasing spontaneous communication initiations, expanding vocabulary and phrase length, improving joint attention skills, developing pretend play schemes, reducing distress with transitions, and expanding self-feeding independence.

Core intervention techniques include following the child's lead (interventionist joins Max's activity rather than redirecting), creating communication opportunities (environment arranged to prompt communication with desired objects placed in view but out of reach, containers difficult to open, activities interrupted), modeling and expanding language (when Max communicates, interventionist responds immediately with slightly more complex language), establishing joint activity routines (predictable, enjoyable routines providing multiple opportunities for social engagement and turn-taking), and using positive reinforcement (Max's attempts reinforced immediately with natural consequences and specific labeled praise) \parencite{Dawson2010}. Additional techniques include task analysis and shaping (complex skills broken into smaller steps taught sequentially), prompting and prompt fading (prompts provided at minimum necessary level, systematically faded), and generalization planning (skills taught across multiple contexts, materials, and people from the outset).

Parent coaching represents a critical component, enabling intensive intervention within daily routines. Research indicates that parents can implement strategies effectively with appropriate coaching, producing meaningful improvements \parencite{Rogers2012}. For Stephanie and Charles, parent coaching sessions might occur weekly initially, focusing on embedding strategies within mealtimes, playtime, bedtime routines, and community outings.

Implementing NDBI with Max's family requires cultural responsiveness. Explicit discussion with Charles and Stephanie about their values, goals, and preferences would guide intervention planning. Incorporating cultural activities, stories, and materials into intervention would support Max's cultural identity development. Charles's Aboriginal community involvement might provide opportunities for Max to participate in cultural activities with structure supporting his learning. The interventionist would need to demonstrate cultural humility, seeking to understand the family's perspectives rather than imposing mainstream assumptions.

Research suggests that 20-25 hours per week of intervention produces optimal outcomes \parencite{Dawson2010}. For Max, a combination of direct therapy sessions (5-10 hours weekly), parent-implemented intervention during daily routines (10-15 hours weekly), and preschool-implemented strategies by teacher aide (15-20 hours during preschool days) could approximate this intensity. Duration typically spans 2-3 years, with progress monitoring occurring at least quarterly.

Research on ESDM indicates that children receiving this intervention demonstrate improvements including increased language abilities, enhanced social communication, reduced autism symptom severity, and improved cognitive functioning \parencite{Dawson2012}. However, response is variable \parencite{Vivanti2014}. For Max, reasonable expectations might include expansion of expressive language to longer utterances, increased spontaneous communication, improved joint attention, more elaborate pretend play, reduced transition distress through visual supports, and increased self-care independence. Progress monitoring would enable evaluation of whether intervention produces expected gains and modification if progress is insufficient.

NDBI would be implemented as part of comprehensive intervention potentially including speech-language pathology services targeting specific communication skills, occupational therapy addressing sensory processing and fine motor skills if indicated, and preschool services including special education support and teacher aide assistance. Coordination across these services would ensure consistency in approaches and goals.