% Chapters/03-Treatment.tex

\part{PLANNING}
\label{part:planning}

\chapter{INTERVENTION}
\label{cp:intervention}

\section{Evidence-Based Treatment Approaches}

\subsection{Overview of Evidence-Based Interventions}

Research on ASD intervention has expanded substantially, with multiple systematic reviews now available to guide evidence-based practice \parencite{Reichow2012, Warren2011}. Interventions categorize into comprehensive approaches targeting broad developmental domains versus focused interventions addressing specific skills \parencite{NationalResearchCouncil2001}.

Early intensive behavioural intervention (EIBI) based on Applied Behaviour Analysis principles represents one comprehensive category. Research by \textcite{Reichow2012} in their Cochrane review suggests that EIBI can produce improvements in intelligence, language, and adaptive functioning, though effect sizes vary and response is heterogeneous across children. Developmentally-based approaches integrating behavioral principles with relationship-focused methods include the Early Start Denver Model (ESDM). Research by \textcite{Dawson2012} indicates that ESDM may produce changes in brain activity patterns alongside behavioral improvements, suggesting neuroplastic effects.

Focused interventions target specific developmental domains. Joint attention interventions show particular promise, with research by \textcite{Kasari2010} demonstrating that such interventions can improve both joint attention abilities and language outcomes. Communication-focused interventions including the Picture Exchange Communication System (PECS) provide alternative communication methods. Research by \textcite{Maglione2012} indicates PECS can increase communication initiations, though evidence for effects on spoken language development remains mixed.

\section{Detailed Strategy: Naturalistic Developmental Behavioral Intervention}

For Max, naturalistic developmental behavioral intervention (NDBI) approaches represent a particularly appropriate evidence-based strategy. NDBI integrates principles from applied behavior analysis with developmental science, implemented within natural play-based contexts \parencite{Schreibman2015}.

\subsection{Conceptual Foundation and Rationale}

The Early Start Denver Model \parencite{Rogers2010} exemplifies NDBI approaches, combining ABA teaching principles with developmental relationship-based strategies. Key features include targeting developmental skills across all domains, teaching within playful social interactions following the child's interests, and implementing intervention intensively across contexts \parencite{Dawson2010}.

Several factors suggest NDBI would be particularly suitable for Max. His young age falls within the developmental window where early intensive intervention demonstrates strongest effects \parencite{Rogers2012}. His emerging communication abilities suggest he is positioned to benefit from intervention targeting language expansion within social contexts. Research by \textcite{Schreibman2015} indicates that NDBI approaches can be culturally adapted more readily than highly structured ABA approaches, potentially aligning better with values emphasizing learning through observation and participation in meaningful activities.

\subsection{Implementation Process}

NDBI implementation begins with comprehensive assessment using developmental frameworks, evaluating skills across receptive communication, expressive communication, social skills, imitation, cognition, play, fine motor, gross motor, behaviour, and independence domains \parencite{Rogers2010}. Goal-setting occurs collaboratively with parents, ensuring alignment between intervention targets and family priorities.

Core intervention techniques include following the child's lead (interventionist joins Max's activities rather than redirecting), creating communication opportunities (arranging environment to prompt requests), modelling and expanding language (responding immediately to Max's communication with slightly more complex language), establishing joint activity routines (predictable, enjoyable routines providing multiple opportunities for social engagement), and using positive reinforcement (Max's attempts reinforced immediately with natural consequences; \cite{Dawson2010}).

Parent coaching represents a critical component, enabling intensive intervention within daily routines. Research by \textcite{Rogers2012} examining parent-delivered ESDM indicates that parents can implement strategies effectively with appropriate coaching, producing meaningful improvements in child outcomes. For Stephanie and Charles, parent coaching sessions might occur weekly initially, focusing on embedding intervention strategies within mealtimes, playtime, bedtime routines, and community outings.

\subsection{Dosage, Duration, and Expected Outcomes}

Research on NDBI approaches suggests that 20-25 hours per week of intervention produces optimal outcomes \parencite{Dawson2010}. For Max, a combination of direct therapy sessions, parent-implemented intervention during daily routines, and preschool-implemented strategies could approximate this intensity.

Research on ESDM indicates that children receiving this intervention demonstrate improvements including increased language abilities, enhanced social communication, reduced autism symptom severity, and improved cognitive functioning \parencite{Dawson2012}. However, response to intervention is variable, with some children making substantial gains while others show more modest improvements \parencite{Vivanti2014}. Progress monitoring would enable evaluation of whether intervention is producing expected gains and modification of approach if progress is insufficient.

\section{Conclusion}

Max's case illustrates the complexity inherent in neurodevelopmental assessment for young children presenting with concerns for autism spectrum disorder and intellectual developmental disorder. Comprehensive evaluation employing multiple assessment methods and professional perspectives will enable accurate diagnostic formulation while identifying Max's unique profile of strengths and needs \parencite{Ozonoff2005}. Evidence-based interventions such as naturalistic developmental behavioral approaches offer promise for supporting his development when implemented with appropriate intensity, cultural responsiveness, and family partnership \parencite{Rogers2010}.