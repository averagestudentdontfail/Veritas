\part{Intervention}
\label{part:intervention}

\chapter{Psychological Interventions}

\section{Risk and Safety Considerations}

Three critical safety domains require immediate stabilisation preceding trauma-focused intervention. Suicidal ideation management represents highest priority given passive ideation with specific method contemplation. Implementation requires collaborative safety planning using Stanley-Brown Safety Planning Intervention: identifying personal warning signs (increased isolation, hopelessness), internal coping strategies (distraction techniques, self-soothing), social distractions, family/friend crisis contacts, professional resources, and means restriction including medication security and avoiding high-risk locations \parencite{stanley2012}. Weekly Columbia Suicide Severity Rating Scale administration ensures systematic monitoring with clear escalation protocols \parencite{posner2011}.

Alcohol use stabilisation necessitates immediate intervention given escalating patterns potentially compromising treatment engagement and increasing impulsivity. Motivational interviewing explores ambivalence, developing discrepancy between current consumption and valued goals (military service, future family) whilst supporting self-efficacy for change \parencite{miller2013}. Psychoeducation addresses bidirectional trauma-alcohol relationships, introducing self-medication concepts whilst highlighting perpetuation of symptoms. AUDIT-C provides validated monitoring throughout treatment.

Anger and behavioural dysregulation manifesting through property destruction requires immediate skill development preventing interpersonal violence escalation. Dialectical Behaviour Therapy distress tolerance modules offer concrete strategies: TIPP (Temperature change, Intense exercise, Paced breathing, Paired muscle relaxation) for acute crises; ACCEPTS (Activities, Contributing, Comparisons, Emotions, Pushing away, Thoughts, Sensations) for sustained distress without destructive behaviour \parencite{linehan2015}.

\section{Social and Cultural Considerations}

Bill's treatment requires careful attention to intersecting sociocultural factors. South African immigration during childhood family dysfunction suggests acculturation stress affecting identity formation and belonging \parencite{bhugra2005}. Treatment should explore how cultural dislocation compounded trauma impacts, potentially incorporating narrative therapy examining cultural identity stories. Military culture's emphasis on strength, self-reliance, and stoicism conflicts with vulnerability required for trauma processing; reframing treatment using military-consistent language ("operational readiness," "psychological fitness," "mission planning") whilst acknowledging service meaning may enhance engagement \parencite{hoge2004}. Masculine socialisation creates additional emotional expression barriers; psychoeducation normalising neurobiological trauma responses rather than character weakness, using medical analogies comparing psychological to physical injuries, may reduce shame-based resistance \parencite{tolin2006}.

\section{Referral Recommendations}

Comprehensive treatment requires coordinated multidisciplinary intervention. Psychiatric evaluation for evidence-based pharmacotherapy: SSRIs (sertraline 50--200mg or paroxetine 20--60mg daily) demonstrating efficacy for military PTSD; prazosin (1--15mg nocte) specifically targeting trauma nightmares through noradrenergic blockade \parencite{raskind2013}. Sleep medicine consultation for comprehensive polysomnography evaluating parasomnia presentation differentiating PTSD-related disturbance from primary sleep disorders. Occupational therapy assessing functional capacity, vocational rehabilitation needs, and return-to-duty fitness \parencite{penk2002}.

\section{Therapeutic Interventions}

Three evidence-based interventions demonstrate strong empirical support for military PTSD.

Cognitive Processing Therapy (CPT) directly targets maladaptive cognitions maintaining PTSD through systematic examination of "stuck points" where traumatic experiences conflict with pre-existing beliefs \parencite{resick2017}. Bill's survivor guilt, moral injury, and negative self-concept represent cognitive maintenance factors CPT specifically addresses through Socratic dialogue and cognitive restructuring.

Prolonged Exposure (PE) facilitates emotional processing through systematic confrontation of avoided memories and situations, demonstrating robust military PTSD efficacy \parencite{foa2007}. Bill's marked avoidance and emotional numbing suggest habituation-based intervention could reduce symptoms through fear structure modification \parencite{rauch2006}.

Skills Training in Affective and Interpersonal Regulation/Modified Prolonged Exposure (STAIR-MPE) provides phased treatment prioritising emotion regulation before trauma processing, particularly suited for complex presentations with developmental trauma \parencite{cloitre2002}.

\section{Cognitive Processing Therapy: Implementation Protocol}

CPT follows manualised twelve-session protocol adapted for military populations \parencite{resick2017}. Sessions 1--2: Psychoeducation establishing cognitive model; impact statement exploring trauma's belief effects. Session 3: ABC worksheets teaching thought-emotion connections; Socratic questioning challenging initial stuck points. Sessions 4--5: Written trauma accounts facilitating emotional processing whilst identifying maintaining cognitions. Sessions 6--7: Challenging Questions Worksheets systematically examining evidence, alternatives, thinking patterns. Sessions 8--12: Five thematic modules comprising Safety (threat assessment), Trust (rebuilding capacity), Power/Control (accepting limitations, identifying genuine control), Esteem (balanced self-worth), Intimacy (connection capacity despite trauma).

Effectiveness monitoring: PCL-5 weekly (10-point reduction indicates clinically significant change); PHQ-9 tracking depression; Posttraumatic Cognitions Inventory assessing mechanism change \parencite{weathers2013}.