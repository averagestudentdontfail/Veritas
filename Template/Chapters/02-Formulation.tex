\part{FORMULATION}
\label{part:formulation}

\chapter{FORMULATION}

\section{Comprehensive Biopsychosocial Integration}

Bill's presentation reflects complex developmental vulnerability and military trauma interactions within an integrated biopsychosocial framework requiring multifaceted conceptualisation.

\subsection{Predisposing Vulnerabilities}

Multiple childhood experiences created intersecting vulnerability pathways. Immigration from South Africa aged six coincided with parental conflict escalation during critical attachment formation periods, disrupting secure base establishment essential for emotional regulation development \parencite{schore2003}. Witnessing father's alcohol-fuelled violence toward mother established trauma exposure templates, dysregulating neurobiological stress response systems and creating hypervigilance patterns, emotional dysregulation vulnerabilities, and maladaptive coping strategies modelling substance use for distress management \parencite{brewin2000}. Mother's psychiatric hospitalisation when Bill was eight represented critical attachment rupture, with expressed shame regarding protection failure ("couldn't protect her") establishing enduring schemas of personal inadequacy, excessive responsibility, and fundamental helplessness subsequently reactivated by military experiences. Parental separation at twelve created additional losses and divided loyalties ("guilty leaving his father"). School bullying targeting perceived intellectual deficits ("dumb") compounded negative self-concept development, with early educational discontinuation limiting vocational opportunities beyond military service. These accumulated adversities created what \textcite{briere2015} conceptualise as "complex developmental trauma": disrupted attachment patterns, impaired self-concept formation, emotion dysregulation vulnerabilities, and interpersonal difficulties increasing susceptibility to subsequent traumatic stress.

\subsection{Precipitating Factors}

Military service initially provided compensatory experiences addressing earlier developmental deficits: structure, belonging, identity, and meaningful peer connections ("real friends for the first time"). This environment temporarily scaffolded self-organisation capacities compromised by developmental trauma. However, Afghanistan combat exposure overwhelmed these compensatory mechanisms through multiple pathways. Witnessing his closest friend's death activated profound survivor guilt whilst shattering assumptions about predictability, control, and fairness. Forced retreat leaving friend's body triggered abandonment schemas established during childhood maternal protection failures. Equally significant, witnessing civilian abuse by service members without intervening created moral injury: deep psychological wounds resulting from perpetrating, witnessing, or failing preventing acts violating core moral beliefs, generating shame, self-condemnation, and existential crisis distinct from fear-based responses \parencite{litz2009}. Subsequent medical discharge following knee injury removed military identity scaffolding, precipitating psychological decompensation as underlying vulnerabilities re-emerged without environmental supports.

\subsection{Perpetuating Mechanisms}

Multiple interconnected factors maintain Bill's difficulties through self-reinforcing cycles. Cognitive factors include persistent negative trauma-related cognitions ("I should have saved him," "Good people die while bad people survive") creating information processing biases selectively attending to confirmatory evidence whilst dismissing disconfirmatory information \parencite{ehlers2000}. Behavioural patterns perpetuate dysfunction: alcohol provides temporary numbing but prevents processing whilst creating additional shame ("becoming like father"), social withdrawal maintains disconnection preventing corrective experiences, and avoidance prevents habituation. Neurobiological alterations sustain symptoms through chronically dysregulated stress systems evidenced by hypervigilance, exaggerated startle, and anger dyscontrol, suggesting altered amygdala-hippocampal-prefrontal circuitry characteristic of PTSD \parencite{shin2010}. Environmental factors including temporary accommodation, employment uncertainty, and family separation create ongoing instability maintaining threat perception.

\subsection{Protective Factors}

Despite severity, Bill demonstrates important strengths: help-seeking despite reluctance indicates motivation; family connections provide potential support; military identity offers belonging; future goals suggest hope; previous adaptive functioning indicates recovery capacity; absence of active suicide planning reduces immediate risk.

\textit{(See Appendix A for formulation schematic)}
