% Chapters/01-Introduction.tex

\part{FOUNDATION}
\label{part:foundation}

\chapter{FORMULATION}
\label{cp:formulation}

\section{Preface}

The referral for Max, a 5-year-old child of Aboriginal Australian and Irish heritage, warrants systematic evaluation for neurodevelopmental conditions. Two primary diagnostic considerations emerge from Dr. Smith's referral: Autism Spectrum Disorder (ASD; F84.0) as provisional diagnosis, and Intellectual Developmental Disorder (IDD; F70.0-F79.0) requiring further clarification.

\section{Autism Spectrum Disorder}

\subsection{Social Communication and Interaction Dynamics}

Max's presentation suggests potential ASD, pending comprehensive assessment. Regarding Criterion A (persistent deficits in social communication and interaction), several concerning features emerge \parencite{AmericanPsychiatricAssociation2022}:

Social-emotional reciprocity deficits include lack of shared enjoyment, failure to seek comfort when hurt, and apparent preference for solitary engagement (described as being ``independent in his own world''). Nonverbal communication shows inconsistent eye contact and joint attention (skills typically emerging by 12 months \parencite{Elsabbagh2010}, alongside reliance on leading his mother to desired objects rather than pointing or gestures. This protodeclarative communication delay represents a core early marker distinguishing autism from other developmental conditions \parencite{Mundy2009}. Relationship deficits manifest through parallel play preference, absence of friendships, sharing difficulties, and limited peer engagement despite regular preschool attendance.

\subsection{Restricted and Repetitive Behaviour Dynamics}

Criterion B features include:

Stereotyped motor behaviours show as repetitive lining up of cars and trucks, and fascination with spinning wheels. These behaviours occur in approximately 70\% of young children later diagnosed with ASD \parencite{Leekam2011}. Insistence on sameness shows as rigid morning routines (requiring rice bubbles and grapes specifically), significant distress when routines disrupted or preferred objects removed. This rigidity intensifies during preschool years \parencite{Rodgers2012}. Restricted interests shows as abnormal intensity focus on vehicles, possessiveness toward these toys, rejection of alternatives like balls. Sensory hyper-reactivity shows as pronounced auditory distress (hairdryer, vacuum cleaner, lawnmower) producing screaming and prolonged upset; tactile sensitivities during tooth brushing requiring physical restraint. These patterns align with findings that 69--95\% of children with ASD demonstrate atypical sensory processing across multiple modalities \parencite{Tomchek2007}.

\subsection{Diagnostic Criteria Considerations}

Criterion C (early developmental onset) appears met through concerns evident from infancy, including speech regression after initial word acquisition at 15 months and delayed motor milestones. Criterion D (clinically significant impairment) appears satisfied across peer relationships, self-care (regression in self-feeding, lack of toilet training indication), and adaptive functioning requiring teacher aide support. Criterion E (not better explained by IDD) requires careful evaluation, as Max's cognitive functioning may substantially contribute to presentation.

\section{Intellectual Developmental Disorder}

\subsection{Cognitive Functioning}

Criterion A (deficits in general mental abilities) warrants systematic evaluation. Max's GMDS-ER assessment at 29.9 months yielded General Quotient of 69 (below first percentile), suggesting mild global developmental delay. However, this two-year-old assessment requires contemporary evaluation using age-appropriate measures examining both verbal and nonverbal abilities \parencite{AmericanPsychiatricAssociation2022}. Cognitive profiles in young children with ASD demonstrate instability, with some showing improved scores following intervention \parencite{Munson2008}. The interaction between delayed speech development and assessment performance requires consideration, as language-based measures may underestimate nonverbal reasoning in children with limited verbal output \parencite{Charman2011}.

\subsection{Adaptive Functioning}

Criterion B (adaptive functioning deficits) shows documented difficulties across conceptual domains (limited vocabulary, word combinations rather than sentences at age 5), social domains (peer interaction difficulties, absence of reciprocal friendships), and practical domains (lack of toilet training indication, self-feeding regression, substantial preschool support required) \parencite{Tasse2012}. Systematic assessment using standardized instruments such as Vineland Adaptive Behaviour Scales, Third Edition \parencite{Sparrow2016} would provide essential quantitative data across contexts. Criterion C (developmental period onset) appears satisfied through concerns evident from infancy.

\section{Predisposing, Precipitating, Perpetuating, and Protective Factors}

\subsection{Predisposing Factors}

Biological: Prematurity (35 weeks, 3 days) and low birth weight (2645 grams) may have contributed to neurological vulnerability, though ``good condition'' at delivery suggests resilience. Late preterm infants show elevated neurodevelopmental difficulty rates, particularly with additional risk factors \parencite{Johnson2011}. Neonatal jaundice requiring phototherapy represents early biological stress potentially affecting neurodevelopment. Severe infant gastroesophageal reflux may have influenced feeding experiences and early parent-child interactions.

Genetic: Family history of ASD (maternal nephew) suggests genetic vulnerability, with heritability estimates ranging from 37\% to over 90\% \parencite{Tick2016, Bai2019}.

Developmental: Speech regression (acquiring ``cat'' at 15 months, then ceasing verbal output) suggests atypical language trajectory potentially reflecting neurological differences. Sensory sensitivities across auditory and tactile modalities may have contributed to affect regulation difficulties and stress responses during daily activities.

Environmental: Stephanie's postnatal depression following Max's birth potentially affected early attachment and maternal attunement during critical developmental windows. While maternal depression can influence parent-infant interaction quality and infant regulatory development \parencite{Feldman2009}, effects vary considerably based on severity, duration, and treatment access (details unspecified in referral). The transition from breastfeeding to bottle-feeding at six weeks occurred during this period.

Cultural: Max's Aboriginal heritage through his father requires respectful acknowledgment and integration into assessment and treatment planning. Aboriginal Australian children experience service access disparities, with IDD prevalence of 39 per 1,000 compared to 16 per 1,000 for non-Aboriginal children in Western Australia \parencite{Bourke2016}. These differences likely reflect systemic barriers and social determinants rather than genetic factors. Cultural sensitivity and knowledge of sociostructural conditions prove essential during assessment \parencite{AmericanPsychiatricAssociation2022}. The family's Parramatta residence on Darug land suggests active connection to Charles's cultural identity and community, representing both protective factor and consideration for culturally responsive service provision.

\subsection{Precipitating Factors}

No specific recent precipitating events emerge from referral. However, impending transition from preschool to primary school (High Street Primary School) represents significant environmental change potentially affecting functioning, as educational transitions often challenge children with neurodevelopmental differences, particularly those relying on predictable routines.

\subsection{Perpetuating Factors}

Ongoing sensory sensitivities create distress during routine activities (tooth brushing, household appliance noise), potentially maintaining heightened stress responses and limiting typical childhood participation. Rigid routine reliance and disruption-related distress may receive inadvertent reinforcement through parental accommodation. Limited communicative repertoire (word combinations rather than sentences) constrains expression of needs, preferences, and emotional states, potentially contributing to frustration and behavioral dysregulation manifesting as tantrums.

Current feeding patterns where Stephanie feeds Max, despite his demonstrated capacity for independence at 18 months, may reflect parental accommodation while limiting autonomy development opportunities. Teacher aide support at preschool, while appropriate, may inadvertently limit peer interaction if delivered primarily one-on-one rather than facilitating peer engagement.

\subsection{Protective Factors}

Max demonstrates developmental strengths: pretend play skill acquisition (feeding animals) through speech pathology intervention indicates symbolic representation capacity and responsiveness to targeted teaching. Good sleep pattern (8pm--7am) represents significant strength, as sleep difficulties frequently complicate neurodevelopmental presentations \parencite{Maski2011}. Language re-acquisition following regression demonstrates neuroplasticity and learning capacity. Strong vehicle interests, while potentially restricted, could serve as motivational engagement tools.

Family resources: Both parents' TAFE diploma completion indicates educational engagement and potential understanding of learning processes. Charles's active Aboriginal community involvement suggests access to cultural support networks and identity resources. Stephanie's part-time employment provides financial stability while enabling parental availability.

Service access: Current ABC preschool attendance with teacher aide support indicates access to educational services with appropriate accommodations. Upcoming High Street Primary School transition suggests educational provision continuity. Family connection to Darug land and Aboriginal community provides cultural anchoring and potential access to culturally responsive support services.