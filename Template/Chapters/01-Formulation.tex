% Chapters/01-Introduction.tex

\part{FOUNDATION}
\label{part:foundation}

\chapter{FORMULATION}
\label{cp:formulation}

\section{Preface}

The referral information for Max, a 5-year-old child of Aboriginal Australian and Irish heritage, presents developmental concerns warranting systematic evaluation for neurodevelopmental conditions. Based on Dr. Smith's referral, two primary diagnostic considerations emerge: Autism Spectrum Disorder (ASD; F84.0) as a provisional diagnosis, and Intellectual Developmental Disorder (IDD; F70.0-F79.0) as a differential diagnosis requiring further clarification.

\section{Autism Spectrum Disorder}

Max's presentation suggests he may meet criteria for ASD, though comprehensive assessment remains necessary. Regarding Criterion A (persistent deficits in social communication and social interaction), several concerning features appear documented. Max demonstrates deficits in social-emotional reciprocity, evidenced by his lack of shared enjoyment, failure to seek comfort when hurt, and apparent preference for solitary engagement \parencite{AmericanPsychiatricAssociation2022}. His nonverbal communicative behaviours appear atypical, with inconsistent eye contact and joint attention, skills typically emerging by 12 months in neurotypical development \parencite{Elsabbagh2010}. His reliance on leading his mother to desired objects rather than conventional pointing or gestures suggests delays in protodeclarative communication, which \textcite{Mundy2009} identify as a core early marker distinguishing autism from other developmental conditions.

\subsection{Social Communication and Interaction Dynamics}

Max demonstrates deficits in social-emotional reciprocity, evidenced by his lack of shared enjoyment, failure to seek comfort when hurt, and apparent preference for solitary engagement \parencite{AmericanPsychiatricAssociation2022}. His nonverbal communicative behaviours appear atypical, with inconsistent eye contact and joint attention, skills typically emerging by 12 months in neurotypical development \parencite{Elsabbagh2010}. His reliance on leading his mother to desired objects rather than conventional pointing or gestures suggests delays in protodeclarative communication, which \textcite{Mundy2009} identify as a core early marker distinguishing autism from other developmental conditions.

\subsection{Restricted and Repetitive Behaviour Dynamics}

Regarding Criterion B (restricted, repetitive patterns of behaviour, interests, or activities), Max exhibits characteristic features including stereotyped motor behaviours evident in his repetitive lining up of vehicles and fascination with spinning wheels, behaviours that \textcite{Leekam2011} suggest occur in approximately 70\% of young children later diagnosed with ASD. His insistence on sameness manifests through rigid morning routines and significant distress when routines are disrupted, consistent with research by \textcite{Rodgers2012} indicating that such rigidity often intensifies during preschool years. His hyper-reactivity to sensory input, particularly regarding auditory stimuli, aligns with findings by \textcite{Tomchek2007} that 69-95\% of children with ASD demonstrate atypical sensory processing across multiple modalities.

\section{Intellectual Developmental Disorder}

The differential diagnosis of IDD warrants systematic evaluation. Max's GMDS-ER assessment at 29.9 months yielded a General Quotient of 69, suggesting mild global developmental delay \parencite{AmericanPsychiatricAssociation2022}. However, this assessment occurred over two years ago, and research by \textcite{Munson2008} indicates that cognitive profiles in young children with ASD can demonstrate instability, with some children showing improved scores following intervention. Contemporary evaluation using age-appropriate measures becomes essential, particularly given that language-based measures may underestimate nonverbal reasoning capabilities in children with limited verbal output \parencite{Charman2011}.

\subsection{Cognitive Functioning Considerations}

Max's GMDS-ER assessment at 29.9 months yielded a General Quotient of 69, suggesting mild global developmental delay \parencite{AmericanPsychiatricAssociation2022}. However, this assessment occurred over two years ago, and research by \textcite{Munson2008} indicates that cognitive profiles in young children with ASD can demonstrate instability, with some children showing improved scores following intervention. Contemporary evaluation using age-appropriate measures becomes essential, particularly given that language-based measures may underestimate nonverbal reasoning capabilities in children with limited verbal output \parencite{Charman2011}.

\subsection{Adaptive Functioning Considerations}

Regarding Criterion B (deficits in adaptive functioning), Max demonstrates difficulties across conceptual, social, and practical domains. His limited vocabulary and use of word combinations rather than sentences at age 5 suggests delays in the conceptual domain, while his peer interaction difficulties and lack of toilet training indication suggest impairments in social and practical domains respectively \parencite{Tasse2012}. Systematic adaptive behaviour assessment using standardized instruments such as the Vineland Adaptive Behaviour Scales, Third Edition \parencite{Sparrow2016} would provide essential quantitative data regarding functioning relative to same-age peers.

\section{Predisposing, Precipitating, Perpetuating, and Protective Factors}

\subsection{Predisposing Factors}

Max's prematurity (born at 35 weeks, 3 days) and low birth weight (2645 grams) may have contributed to neurological vulnerability. Research by \textcite{Johnson2011} indicates that late preterm infants demonstrate elevated rates of neurodevelopmental difficulties, particularly when combined with other risk factors. The family history of ASD (maternal nephew) suggests genetic vulnerability, with heritability estimates for ASD ranging from 37\% to over 90\% depending on methodological approaches \parencite{Tick2016, Bai2019}.

Stephanie's postnatal depression following Max's birth potentially affected early attachment formation during critical developmental windows. Research by \textcite{Feldman2009} suggests that maternal depression can influence parent-infant interaction quality, though effects vary considerably based on severity, duration, and treatment access.

\subsection{Cultural Considerations}

Max's Aboriginal heritage requires respectful acknowledgment and integration into assessment and treatment planning. Aboriginal Australian children experience disparities in accessing developmental services, with research by \textcite{Bourke2016} reporting that intellectual developmental disorder prevalence among Aboriginal children in Western Australia was 39 per 1,000 compared to 16 per 1,000 for non-Aboriginal children, differences likely reflecting systemic barriers and social determinants of health rather than genetic factors. Cultural sensitivity and knowledge of sociostructurally conditions prove essential during assessment \parencite{AmericanPsychiatricAssociation2022}.

\subsection{Protective Factors}

Max demonstrates several strengths suggesting resources for development. His acquisition of some pretend play skills through speech pathology intervention indicates capacity for symbolic representation and responsiveness to targeted teaching. His good sleep pattern (8pm-7am) represents a significant strength, as sleep difficulties frequently complicate neurodevelopmental presentations \parencite{Maski2011}. Both parents' completion of TAFE diplomas indicates educational engagement, while Charles's active involvement in his Aboriginal community suggests access to cultural support networks.