% Chapters/01-Formulation.tex

\part{FOUNDATION}
\label{part:foundation}

\chapter{FORMULATION}
\label{cp:formulation}

\section{Preface}

The referral for Max, a 5-year-old child of Aboriginal Australian and Irish heritage, warrants systematic evaluation for neurodevelopmental conditions. Two primary diagnostic considerations emerge: Autism Spectrum Disorder (ASD; F84.0) as provisional diagnosis, and Intellectual Developmental Disorder (IDD; F70.0-F79.0) requiring further clarification.

\section{Autism Spectrum Disorder}

\subsection{Social Communication and Interaction Dynamics}

Max's presentation suggests potential ASD, pending comprehensive assessment. Regarding Criterion A (persistent deficits in social communication and interaction), several concerning features emerge \parencite{AmericanPsychiatricAssociation2022}. Social-emotional reciprocity deficits include lack of shared enjoyment, failure to seek comfort when hurt, and apparent preference for solitary engagement. Nonverbal communication shows inconsistent eye contact and joint attention (skills typically emerging by 12 months \parencite{Elsabbagh2010}, alongside reliance on leading his mother to desired objects rather than pointing or gestures. This protodeclarative communication delay represents a core early marker distinguishing autism from other developmental conditions \parencite{Mundy2009}. Relationship deficits manifest through parallel play preference, absence of friendships, and limited peer engagement despite regular preschool attendance.

\subsection{Restricted and Repetitive Behaviour Dynamics}

Criterion B features include stereotyped motor behaviours showing as repetitive lining up of cars and fascination with spinning wheels, occurring in approximately 70\% of young children later diagnosed with ASD \parencite{Leekam2011}. Insistence on sameness shows as rigid morning routines (requiring rice bubbles and grapes specifically) and significant distress when routines disrupted, intensifying during preschool years \parencite{Rodgers2012}. Restricted interests show as abnormal intensity focus on vehicles with possessiveness and rejection of alternatives. Sensory hyper-reactivity shows as pronounced auditory distress (hairdryer, vacuum cleaner, lawnmower) and tactile sensitivities during tooth brushing requiring physical restraint, aligning with findings that 69--95\% of children with ASD demonstrate atypical sensory processing \parencite{Tomchek2007}.

Criterion C (early developmental onset) appears met through concerns from infancy, including speech regression after initial word acquisition at 15 months. Criterion D (clinically significant impairment) appears satisfied across peer relationships, self-care, and adaptive functioning requiring teacher aide support. Criterion E (not better explained by IDD) requires careful evaluation, as cognitive functioning may substantially contribute to presentation.

\section{Intellectual Developmental Disorder}

Max's GMDS-ER assessment at 29.9 months yielded General Quotient of 69 (below first percentile), suggesting mild global developmental delay. However, this two-year-old assessment requires contemporary evaluation using age-appropriate measures examining both verbal and nonverbal abilities \parencite{AmericanPsychiatricAssociation2022}. Cognitive profiles in young children with ASD demonstrate instability \parencite{Munson2008}, and language-based measures may underestimate nonverbal reasoning in children with limited verbal output \parencite{Charman2011}.

Adaptive functioning shows documented difficulties across conceptual domains (limited vocabulary, word combinations rather than sentences at age 5), social domains (peer interaction difficulties, absence of reciprocal friendships), and practical domains (lack of toilet training indication, self-feeding regression, substantial preschool support required) \parencite{Tasse2012}. Systematic assessment using Vineland Adaptive Behaviour Scales, Third Edition \parencite{Sparrow2016} would provide essential quantitative data across contexts.

\section{Predisposing, Precipitating, Perpetuating, and Protective Factors}

\subsection{Predisposing Factors}

Prematurity (35 weeks, 3 days) and low birth weight (2645 grams) may have contributed to neurological vulnerability, though good condition at delivery suggests resilience. Late preterm infants show elevated neurodevelopmental difficulty rates \parencite{Johnson2011}. Neonatal jaundice requiring phototherapy and severe infant gastroesophageal reflux represent early biological stressors potentially affecting neurodevelopment and early parent-child interactions.

Family history of ASD (maternal nephew) suggests genetic vulnerability, with heritability estimates ranging from 37\% to over 90\% \parencite{Tick2016, Bai2019}. Speech regression (acquiring ``cat'' at 15 months, then ceasing verbal output) suggests atypical language trajectory. Sensory sensitivities across auditory and tactile modalities may have contributed to affect regulation difficulties.

Stephanie's postnatal depression following Max's birth potentially affected early attachment during critical developmental windows. While maternal depression can influence parent-infant interaction quality \parencite{Feldman2009}, effects vary considerably based on severity, duration, and treatment access (details unspecified in referral).

Max's Aboriginal heritage through his father requires respectful acknowledgment and integration into assessment and treatment planning. Aboriginal Australian children experience service access disparities, with IDD prevalence of 39 per 1,000 compared to 16 per 1,000 for non-Aboriginal children \parencite{Bourke2016}, differences likely reflecting systemic barriers rather than genetic factors. Cultural sensitivity and knowledge of sociostructural conditions prove essential \parencite{AmericanPsychiatricAssociation2022}. The family's Parramatta residence on Darug land suggests active connection to Charles's cultural identity, representing both protective factor and consideration for culturally responsive service provision.

\subsection{Precipitating and Perpetuating Factors}

Impending transition from preschool to primary school represents significant environmental change potentially affecting functioning, as educational transitions often challenge children with neurodevelopmental differences relying on predictable routines.

Ongoing sensory sensitivities create distress during routine activities, potentially maintaining heightened stress responses. Rigid routine reliance may receive inadvertent reinforcement through parental accommodation. Limited communicative repertoire constrains expression of needs and emotional states, potentially contributing to frustration and behavioral dysregulation. Current feeding patterns where Stephanie feeds Max, despite his demonstrated capacity for independence at 18 months, may reflect parental accommodation while limiting autonomy development. Teacher aide support, while appropriate, may inadvertently limit peer interaction if delivered primarily one-on-one.

\subsection{Protective Factors}

Max demonstrates pretend play skill acquisition through speech pathology intervention, indicating symbolic representation capacity and responsiveness to targeted teaching. Good sleep pattern (8pm--7am) represents significant strength \parencite{Maski2011}. Language re-acquisition following regression demonstrates neuroplasticity. Strong vehicle interests could serve as motivational engagement tools.

Both parents' TAFE diploma completion indicates educational engagement. Charles's active Aboriginal community involvement suggests access to cultural support networks. Stephanie's part-time employment provides financial stability while enabling parental availability. Current ABC preschool attendance with teacher aide support indicates access to educational services with appropriate accommodations. Family connection to Darug land and Aboriginal community provides cultural anchoring and potential access to culturally responsive support services.