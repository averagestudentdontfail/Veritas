\part{EPISTEOMOLOGY}
\label{part:epistemology}

\chapter{METHOD}

\section{Theoretical Positioning}

This study employed reflexive thematic analysis \citep{braun2019}, positioned within a contextualist epistemology that acknowledges both individual meaning-making and broader social-cultural contexts shaping experience. The analysis sought patterns of meaning across the dataset rather than quantifying explicit content, privileging interpretative depth over descriptive coverage, with themes understood as analytical constructs created through the interpretive process rather than emerging passively from data. As researcher, my own positive experiences with nature and interest in environmental psychology inevitably shaped both the interview and analytical process. Rather than positioning this as bias requiring elimination, I approached it as productive context enabling empathetic engagement while maintaining analytical awareness, attending throughout analysis to how my interpretations reflected particular theoretical lenses and personal experiences, considering alternative readings while making transparent the rationale for interpretations presented here.

\section{Participant Procedure}

The participant was a 34-year-old urban professional with self-reported regular nature engagement throughout childhood and continuing into adulthood, recruited through purposive sampling based on their history of diverse nature experiences and willingness to reflect on potential psychological impacts. Following informed consent assuring confidentiality, a semi-structured 45-minute interview explored the participant's nature experiences and perceived psychological impacts, prioritizing phenomenological attention to lived experience through prompts such as ``What happened during that experience?'' and ``How did you feel then?'' rather than ``Why?'' questions that might elicit intellectualized explanations divorced from experiential immediacy. The interview concluded with invitation for additional reflections and brief summary checking, providing opportunity to clarify misunderstandings or add dimensions not emerging through preceding questions. The audio-recorded interview was transcribed verbatim, producing approximately 6,800 words of text, with transcription capturing paralinguistic features such as pauses and laughter that might convey emotional significance.

\section{Analytical Protocol}

Analysis followed \citet{braun2006} reflexive approach through six recursive phases. Phase 1 involved repeated reading of the transcript while listening to audio, developing familiarity with both semantic content and emotional tone. Phase 2 generated 78 initial codes through systematic line-by-line examination, capturing both semantic content and latent meanings while maintaining sufficient surrounding context to preserve meaning. Examples included ``childhood nature as freedom,'' ``nature reducing mental clutter,'' ``feeling small in natural settings,'' and ``guilt about environmental impact.'' Phase 3 involved sorting codes into candidate themes through visual mapping, exploring relationships and patterns to identify potential clustering, generating approximately eight candidate themes with tentative boundaries. Phase 4 reviewed candidate themes for internal homogeneity and external heterogeneity, leading to collapsing two overlapping themes, subdividing one encompassing qualitatively different content, and eliminating one weakly supported candidate. Phase 5 defined and named each theme, developing preliminary analytical narratives specifying scope and boundaries while clarifying relationships among themes. Phase 6 involved selecting illustrative extracts and constructing the analytical narrative presented below, with extract selection aimed at providing vivid examples capturing theme essence while representing content range within themes. Throughout this process, I maintained awareness that theme identification represented one possible reading shaped by my theoretical understanding and experiences, considering alternative interpretations while presenting the structure judged most strongly supported by the data.