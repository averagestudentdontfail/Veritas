\part{ANALYSIS}
\label{part:analysis}

\chapter{FINDINGS}

Analysis identified four themes characterizing how this participant understood psychological impacts of nature experiences. Table~\ref{tab:thematic-structure} presents the thematic structure with brief definitions.

\begin{table}[htbp]
\centering
\caption{Thematic Structure Overview}
\label{tab:thematic-structure}
\begin{tabularx}{\textwidth}{>{\bfseries}l X}
\toprule
\textbf{Theme} & \textbf{Definition} \\
\midrule
1. Nature as Cognitive Restoration & Nature experiences providing mental relief from cognitive demands through shift in attentional mode \\
\addlinespace
2. Perspective Transformation Through Immersion & Immersive nature experiences producing lasting shifts in psychological perspective regarding self in relation to broader temporal and spatial contexts \\
\addlinespace
3. The Disconnect Between Environmental Values and Action & Tension between strong environmental concern and behavioural realities, sometimes intensified rather than resolved by nature experiences \\
\addlinespace
4. Nature Connection as Moral Motivation & Emotional investment in natural places motivating conservation-oriented actions despite recognized limitations of individual behaviour \\
\bottomrule
\end{tabularx}
\end{table}

\section{Theme 1: Nature as Cognitive Restoration}

The participant consistently described nature contact as providing mental relief from daily cognitive demands, characterizing urban work environments as producing what they termed ``mental clutter'' requiring constant attention management:

\begin{displayquote}
When I'm in the office all day, it's like my brain gets... cluttered. There's too much going on, too many things competing for attention. Even just stepping out to the park at lunch, there's something about being around trees that makes that feeling ease up.
\end{displayquote}

This restoration involved attentional shifts rather than mental emptiness. The participant distinguished between effortful attention required by work tasks and a more receptive, less demanding quality of attention experienced during nature contact, characterizing this not merely as relaxation but as specific cognitive replenishment:

\begin{displayquote}
After a walk in natural areas, I notice I can focus better. Tasks that felt overwhelming before seem more manageable. It's like my capacity to concentrate gets recharged somehow.
\end{displayquote}
The restorative quality appeared to emerge from what the participant termed nature's ``gentle'' engagement of attention through inherently fascinating features such as moving water, rustling leaves, and varied natural forms that captured interest without requiring directed mental effort. This contrasted sharply with urban environments described as demanding constant attention filtering and navigation of social stimuli. However, the participant acknowledged variability in these effects, noting that brief or interrupted nature contact sometimes provided less benefit than longer, more immersive experiences, suggesting the importance of both quality and duration for restoration to occur, though the participant struggled to articulate precise thresholds distinguishing more from less restorative encounters.

\section{Theme 2: Perspective Transformation Through Immersion}

Beyond immediate cognitive restoration, certain nature experiences, particularly those involving remote or expansive natural landscapes, produced shifts in psychological perspective that persisted beyond the experience itself, involving changes in how the participant situated themselves in relation to broader temporal and spatial scales:

\begin{displayquote}
Standing on that mountain overlook, looking out at this landscape that's been there for thousands of years... you just feel small. But it's not a bad small, not diminishing. It's more like... my everyday worries and stresses suddenly seem less important in perspective.
\end{displayquote}
This perspective shift involved several interrelated elements: experiencing themselves as embedded within larger natural and temporal contexts exceeding individual human concerns; generating feelings the participant characterized as both humbling and oddly comforting; and extending beyond immediate experience to influence subsequent emotional responses to everyday stressors:

\begin{displayquote}
For a while after that trip, when work stress started building up, I could sort of call back that feeling... that sense that this is all temporary, that there's this bigger world out there that just keeps going.
\end{displayquote}
The participant's account suggested that immersive nature experiences could serve as resources for psychological resilience, providing cognitive-emotional tools for managing stress and maintaining perspective during challenging periods. However, the participant noted that these perspective shifts appeared to attenuate over time without regular reinforcement through subsequent nature contact, suggesting the need for ongoing rather than one-time experiences. 

\clearpage

The participant also described these experiences as generating feelings of interconnection with the natural world that contrasted with their typical sense of separation from nature in urban daily life:

\begin{displayquote}
In those moments, you remember you're part of this. Not above it, not separate from it, but connected. We need these places, these ecosystems.
\end{displayquote}
This recognition of interconnection appeared to bridge psychological and environmental dimensions, representing both a subjective feeling state and a cognitive acknowledgment of ecological relationships, experienced as both comforting, providing a sense of belonging, and sobering, highlighting human vulnerability and dependency on functioning ecosystems.

\section{Theme 3: The Disconnect Between Environmental Values and Action}

Despite articulating strong environmental values and expressing concern about ecological degradation, the participant simultaneously described gaps between these values and their behavioural choices, generating psychological tension:

\begin{displayquote}
I care deeply about these issues. Climate change terrifies me. The loss of species, the destruction of habitats... it feels like a tragedy. But then I look at my own life and I'm still driving to work, still consuming things I don't really need, still living in ways that contribute to exactly what I'm worried about. There's this contradiction that's honestly uncomfortable.
\end{displayquote}
The participant identified several perceived barriers maintaining this gap: structural constraints of daily life such as transportation infrastructure and work requirements limiting behavioural options; social norms regarding consumption and convenience creating pressures toward environmentally problematic behaviours; and psychological distance between individual actions and environmental outcomes making connections feel abstract rather than immediate. The participant experienced this not as indifference but as a source of guilt and frustration:

\begin{displayquote}
It's like I'm part of the problem even though I don't want to be. And knowing that makes the problem worse because now I'm also dealing with feeling guilty, which is exhausting, but the guilt doesn't actually change what i do about it.
\end{displayquote}
Notably, the participant suggested that their nature experiences sometimes intensified this psychological tension rather than resolving it. Moments of connection with natural environments could highlight the severity of environmental degradation and the preciousness of what might be lost, making the participant more acutely aware of the gap between their values and actions: ``When I'm hiking in a beautiful forest, I sometimes think about how many places like this have been lost, how many more are threatened. And then I think about my own contribution to that loss, even indirectly, and it feels heavy.'' This theme revealed complexity in relationships between nature experiences and environmental behaviour that might not be captured by models assuming straightforward pathways from nature contact through positive attitudes to pro-environmental action.

\section{Theme 4: Nature Connection as Moral Motivation}

Despite the behavioural inconsistencies described in Theme 3, the participant identified ways that their ongoing nature experiences influenced environmental attitudes and motivated certain conservation-oriented actions through emotional rather than purely rational pathways:

\begin{displayquote}
I'm not logical about it, I'll admit. Like, I know intellectually that one individual changing their behaviour isn't going to solve climate change. But I also can't not care. When I think about those places I love... the beaches where I spent summers as a kid, the mountains where I've had those experiences... I want them to still be there.
\end{displayquote}
This motivation appeared grounded in what the participant described as a form of care or even love for specific natural places and the more abstract notion of wild nature generally, experienced as genuine and morally significant despite questions about consequentialist effectiveness:

\begin{displayquote}
There's something about spending time in nature that makes you want to protect it. Not in an abstract way, but personally. These places become meaningful to you. You develop relationships with them, almost. And when you care about something, you want to see it preserved.
\end{displayquote}
The participant linked this emotional investment to specific behavioural choices, even while acknowledging their limited scope, describing selecting products with environmental certifications when available, contributing financially to conservation organizations, and advocating for environmental policies. The participant attributed these actions at least partly to their ongoing nature experiences:

\begin{displayquote}
Would I do these things if I didn't have those experiences, if I wasn't regularly getting out into natural areas? Honestly, I'm not sure. I think the direct contact keeps it real for me, keeps it from being just an abstract issue.
\end{displayquote}
The emotional connection fostered through nature contact appeared to sustain environmental motivation even when structural barriers prevented more extensive behavioural change, suggesting that personal experiences might serve as counterweight to psychological distance that might otherwise characterize environmental issues encountered primarily through media coverage or scientific reports. However, the participant also expressed ambivalence about whether their individual actions constituted meaningful contribution to environmental protection or merely served to alleviate guilt, creating additional psychological complexity where conservation behaviours potentially functioned simultaneously as genuine expressions of environmental concern and as coping mechanisms for managing distress associated with ecological crisis.

\clearpage

\section{Relationships Among Themes}

The four themes demonstrated important interconnections suggesting that nature experiences operated through multiple pathways. Cognitive restoration (Theme 1) and perspective transformation (Theme 2) both contributed to overall psychological wellbeing the participant associated with nature experiences, potentially making nature experiences emotionally valued in ways that support the moral motivation described in Theme 4, creating feedback loops reinforcing continued nature engagement. However, the disconnect between values and action (Theme 3) complicated any straightforward pathway from nature experiences through wellbeing to environmental behaviour, revealing that psychological impacts of nature contact might be simultaneously beneficial for individual wellbeing yet potentially insufficient for generating comprehensive behavioural change without complementary changes in structural systems.